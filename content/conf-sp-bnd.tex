\chapter{Configuration Spaces of Manifolds with Boundary}
\label{chap.3}
\blkftnt{Based on a work-in-progress in collaboration with P. Lambrechts.}

We now extend the results of the previous chapter to compact manifolds with boundary which admit a ``surjective pretty model''~\cite{CordovaBulensLambrechtsStanley2015}.

For concreteness, let $(M, \partial M)$ be a compact manifold with boundary.
Given some integer $k \ge 0$, its $k$th (ordered) configuration space is:
\[ \Conf_{k}(M) \coloneqq \{ (x_{1}, \dots, x_{k}) \in M^{\times k} \mid x_{i} \neq x_{j} \; \forall i \neq j \}. \]

Suppose that the pair $(M, \partial M)$ has a surjective pretty model induced by $\psi : P \to Q$, where $P$ is a Poincaré duality CDGA and $\psi$ is surjective (see Section~\ref{cnfbnd.sec.surj-pretty-models} for the definitions).
In short, this means that $M$ has been obtained by removing from a closed manifold $N$ (of which $P$ is a model) a sub-polyhedron $X$ (of which $Q$ is a model).
Roughly, the pretty model construction is a way of encoding Poincaré--Lefschetz duality to obtain a model for $M = N - X$.
This situation occurs the case if $M$ is closed (in which case $Q = 0$), if both $M$ and $\partial M$ are $2$-connected and the boundary retracts rationally onto its half-skeleton, if $M$ is a disk bundle over a closed manifold, or if $M$ is obtained by removing the thickening of a high-codimensional subpolyhedron from a closed manifold (see Theorem~\ref{cnfbnd.thm.pretty-model}).
Then a model of $M$ is given by $A = P/I$, where $I = \im(\psi^{!})$ is an ideal of the Poincaré duality CDGA $P$.

We use the same construction as in Chapter~\ref{chap.2} to get a CDGA $\GGt{A}(k)$ (Equation~\eqref{cnfbnd.eq.perturbed-model}), which is somehow a ``perturbation'' of $\GG{A}(k)$.
We show that there is an isomorphism of graded vector spaces between $H^{*}(\GG{A}(k))$ and $H^{*}(\Conf_{k}(M))$ over $\Q$ (see Theorem~\ref{cnfbnd.thm.ga-same-cohom}), which generalizes the result of~\cite{LambrechtsStanley2008a}.
If $M$ is closed, then $I = 0$ and we recover the model considered in Chapter~\ref{chap.2}.
If $M$ is simply connected with simply connected boundary but does not admit a surjective pretty model, we have analogous results as soon as $\dim M \geq 7$ using what we dub ``Poincaré--Lefschetz duality models''.

Just like in the previous chapter, we consider a colored version of the Fulton--MacPherson compactification $\SFM_{M}(\varnothing, \underline{k})$.
This is a stratified space which contains $\Conf_{k}(\mathring{M})$ as its interior, and the inclusion is a weak equivalence.
If $M$ is framed, then the collection $\SFM_{M}(\varnothing, -) = \{ \SFM_{M}(\varnothing, \underline{k}) \}_{k \geq 0}$ is equipped with an action of the Fulton--MacPherson operad, just like in the previous chapter.

The first result of this chapter can be summarized as:
\begin{theoremintro}[See Theorem~\ref{cnfbnd.thm.model-ga}]
  \label{cnfbnd.thm.A}
  Let $M$ be a smooth, simply connected connected compact $n$-manifold with simply connected boundary of dimension at least $5$.
  Assume that either $M$ admits a surjective pretty model, or that $n \geq 7$ so that $M$ admits a Poincaré--Lefschetz duality model.
  Let $A$ be the model built either out of the surjective pretty model or the Poincaré--Lefschetz duality model.

  Then for all $k \geq 0$, the CDGA $\GGt{A}(\underline{k})$ is weakly equivalent to $\OmPA^{*}(\SFM_{M}(\varnothing, \underline{k}))$, and the equivalence is compatible with the action of the symmetric group $\Sigma_{k}$.

  The collection $\GGt{A}$ is a Hopf right comodule over the cooperad $\enV = H^{*}(\FM_{n})$.
  Moreover, if $M$ is framed, then the right Hopf comodule $(\GG{A}, \enV)$ is weakly equivalent to $(\OmPA^{*}(\SFM_{M}(\varnothing,-)), \OmPA^{*}(\FM_{n}))$.
\end{theoremintro}

\begin{corollaryintro}
  \label{cnfbnd.cor.I}
  Let $M$ be a manifold with boundary which satisfies the hypotheses of Theorem~\ref{cnfbnd.thm.A}.
  Then the real homotopy type of $\Conf_{k}(M)$ only depends on the real homotopy type of the pair $(M, \partial M)$.
\end{corollaryintro}

One can also consider colored configuration in the manifold $M$ and study the space, for $k,l \ge 0$:
\[ \Conf_{k,l}(M) \coloneqq \{ (x_{1}, \dots, x_{k}, y_{1}, \dots, y_{l}) \in \Conf_{k+l}(M) \mid x_{i} \in \partial M, y_{j} \in \mathring{M} \}. \]

Once again, we can consider its (colored) Fulton--MacPherson compactification $\SFM_{M}(\underline{k}, \underline{l})$.
This is a stratified space which contains $\Conf_{k,l}(M)$ as its interior, the inclusion being a homotopy equivalence.
When $M$ is framed, the collection $\SFM_{M}$ is equipped with an action of $\SFM_{n}$, an operad weakly equivalent to the Swiss-Cheese operad~\cite{Voronov1999}.
Willwacher~\cite{Willwacher2015a} has built a model $\SGraphs_{n}$ for $\SFM_{n}$ which is similar in spirit to Kontsevich's graph model $\Graphs_{n}$ (due to the non-formality of the Swiss-Cheese operad~\cite{Livernet2015}, it is not possible to go all the way to the cohomology $H^{*}(\SFM_{n})$).
We then have:

\begin{theoremintro}[See Theorem~\ref{cnfbnd.thm.sgraphs-sfm}]
  \label{cnfbnd.thm.B}
  Let $M$ be a smooth, simply connected manifold with a simply connected boundary of dimension at least $5$.
  Assume either that $M$ admits a surjective pretty model or that $n \geq 7$.
  Then we have an explicit model $\SGraphs_{R}(\underline{k}, \underline{l})$ of $\Conf_{k,l}(M)$ which is built out of the pretty (resp.\ PLD) model.
  It is compatible with the action of the symmetric groups $\Sigma_{k} \times \Sigma_{l}$.

  The collection $\SGraphs_{R}$ is a Hopf right comodule over Willwacher's~\cite{Willwacher2015a} model $\SGraphs_{n}$ for the Swiss-Cheese operad.
  Moreover, if $M$ is framed, then the Hopf right comodule $(\SGraphs_{R}, \SGraphs_{n})$ is weakly equivalent to $(\OmPA^{*}(\SFM_{M}), \OmPA^{*}(\SFM_{n}))$.
\end{theoremintro}

\paragraph{Outline}

In Section~\ref{cnfbnd.sec.backgr-recoll}, we recall some background on relative cooperads and comodules over them, pretty models, colored Fulton--MacPherson compactifications, and graph complexes.
In Section~\ref{cnfbnd.sec.pretty-nice}, we define Poincaré--Lefschetz duality models, a generalization of surjective pretty models, and we prove that any simply connected manifold with simply connected boundary of dimension at least $7$ admits a Poincaré--Lefschetz duality model.
In Section~\ref{cnfbnd.sec.model-conf-spac}, we describe the CDGA $\GGt{A}(k)$, and we prove that it computes the Betti numbers of $\Conf_{k}(M)$.
In Section~\ref{cnfbnd.sec.sgcn}, we compute the cohomology of the Swiss-cheese version of the graph complex twisted by a Maurer--Cartan element corresponding to the HKR isomorphism.
In Section~\ref{sec.model-right-hopf}, we show that $\GGt{A}(k)$ is actually a real model of $\Conf_{k}(M)$, and we prove that it is compatible with the action of the Fulton--MacPherson operad.
We also build a graph complex $\SGraphs_{R}^{c_{M},\szphi}(k,l)$ and we show that it is a real model for $\Conf_{k,l}(M)$, compatible with the action of the compactified colored Fulton--MacPherson operad.


\section{Background and recollections}
\label{cnfbnd.sec.backgr-recoll}

We mostly reuse the conventions of Section~\ref{cnf.sec.conventions}.

\subsection{Colored {(co)}operads and {(co)}modules}
\label{cnfbnd.sec.cooperads-comodules}

We deal with special types of two-colored operads, called relative operads~\cite{Voronov1999}, or Swiss-Cheese type operads.

\begin{definition}
  Given an operad $\PP$, a \textbf{relative operad over}\index{relative operad} $\PP$ is an operad in the category of right $\PP$-modules (see also Definition~\ref{sw.def.rel-op}).
  This is equivalent to an operad with two colors (traditionally called the ``closed'' color $\col{c}$ and the ``open'' color $\col{o}$) such that operations with a closed output may only have closed inputs and are given by $\PP$.
\end{definition}

We can encode the part of the operad with an open output as a \textbf{bisymmetric collection}\index{bisymmetric collection} $\QQ$, i.e.\ as a functor from the category of pairs of sets and pairs of bijections to dg-modules.
The first set in the pair corresponds to open inputs, and the second to closed inputs.
There is an identity $\eta_{\mathfrak{o}} \in \QQ(\{*\}, \varnothing)$, and the operadic composition structure maps are given by:
\begin{align*}
  \circ_{T} : \QQ(U, V/T) \otimes \PP(T)
  & \to \QQ(U,V)
  & T \subset V; \\
  \circ_{W,T} : \QQ(U/W, V) \otimes \QQ(W,T)
  & \to \QQ(U, V \sqcup T)
  & W \subset U.
\end{align*}

As mentioned in the definition, we can equivalently view $\QQ$ as an operad in the category of right $\PP$-modules.
The $\PP$-module in arity $U$ is given by $\QQ(U, -)$, and one checks that the operad structure maps $\circ_{W, -} : \QQ(U/W, -) \otimes \QQ(W, -) \to \QQ(U, -)$ are morphisms of right $\PP$-modules.

Given a (one-colored) cooperad $\CC$, a relative cooperad over $\CC$ is defined dually as a bisymmetric collection $\DD$ equipped with structure maps:
\begin{align}
  \circ^{\vee}_{T} : \DD(U,V)
  & \to \DD(U, V/T) \otimes \CC(W)
  & T \subset V; \\
  \circ^{\vee}_{W,T} : \DD(U, V \sqcup T)
  & \to \DD(U/W, V) \otimes \DD(W, T)
  & W \subset U.
\end{align}

Finally, a comodule over a relative $\CC$-cooperad $\DD$ is given by a bisymmetric collection $\NN$ equipped with structure maps:
\begin{align}
  \circ^{\vee}_{T} : \NN(U,V)
  & \to \NN(U, V/T) \otimes \CC(T)
  & T \subset V; \\
  \circ^{\vee}_{W,T} : \NN(U, V \sqcup T)
  & \to \NN(U/W, V) \otimes \DD(W, T)
  & W \subset U.
\end{align}

We can also define \textbf{relative Hopf cooperads} as relative cooperads in the category of CDGAs.

\subsection{Pretty models for compact manifolds with boundary}
\label{cnfbnd.sec.surj-pretty-models}

The cohomology of a compact manifold with boundary does not satisfy Poincaré duality; it satisfies Poincaré--Lefschetz duality instead.
Cordova Bulens, Lambrechts, and Stanley~\cite{CordovaBulensLambrechtsStanley2015,CordovaBulensLambrechtsStanley2015a} use this idea to define ``pretty rational models'' for Poincaré duality pairs such as $(M, \partial M)$.

The rough idea is that one can see $M$ as the complement in some closed manifold $N$ of the thickening of some subpolyhedron $K \subset N$.
One then takes a Poincaré duality model $P$ of $N$ and a model $Q$ of $K$, and formally ``kills'' the forms that are dual to homology classes on $K$ to obtain a model of $M$.

To be explicit, the starting data in the definition of a pretty model is:
\begin{itemize}
\item a Poincaré duality CDGA $P$ of formal dimension $n$ (in the rough picture above, it would be a model for $N$);
\item a connected CDGA $Q$ satisfying $Q^{\ge n/2 - 1} = 0$ (representing a model for $K \subset N$);
\item a morphism $\psi : P \to Q$ (representing the restriction $\Omega^{*}(N) \to \Omega^{*}(K)$).
\end{itemize}

The morphism $\psi$ makes $Q$ into a $P$-module.
Moreover, the dual $P^{\vee}$ (resp.\ $Q^{\vee}$) is a $P$-module (resp.\ $Q$-module) by letting $(x \cdot f)(y) = \pm f(xy)$, and so is the desuspension $P^{\vee}[-n]$ (resp.\ $Q^{\vee}[-n]$).

The Poincaré duality structure on $P$ induces an isomorphism of $P$-modules:
\begin{equation}
  \theta_{P} : P \tkrightarrow{\cong} P^{\vee}[-n]%
  \label{cnfbnd.eq.theta-p}
\end{equation}

given by $a \mapsto \varepsilon_{P}(a \cdot -)$.
The ``shriek map'' $\psi^{!}$\index{$\psi^{"!}$|see{shriek map}}\index{shriek map} is the morphism of $P$-modules defined by the composite map:
\begin{equation}
  \psi^{!} : Q^{\vee}[-n] \tkrightarrow{\psi^{\vee}[-n]} P^{\vee}[-n] \tkrightarrow[\cong]{\theta^{-1}_{P}} P.
\end{equation}
More concretely, given $\alpha \in Q^{\vee}[-n]$, $\psi^{!}(\alpha)$ is uniquely determined by:
\begin{equation}
  \forall x \in P, \; \varepsilon_{P}(\psi^{!}(\alpha) x) = \alpha(\psi(x)).
\end{equation}


One must further assume that $\psi\psi^{!}$ is ``balanced'', i.e.\ that for all $f,g \in Q^{\vee}[-n]$, one has $\psi\psi^{!}(f) \cdot g = f \cdot \psi\psi^{!}(g)$.
Note that, under our assumptions, this is automatically the case by degree reasons (in fact, we have $\psi\psi^{!} = 0$ because of the hypothesis that $Q^{\geq n/2-1} = 0$).

\begin{definition}
  The \textbf{mapping cone}\index{mapping cone} of a chain map $f : X \to Y$, denoted either by $\cone(f)$ or $Y \oplus_{f} X[1]$, is given as a graded vector space by $Y \oplus X[1]$, and the differential is given by $d(y,x) = (d_{Y}y + f(x), d_{X}x)$.
\end{definition}

Note that $\psi\psi^{!} = 0$ implies that $\cone(\psi\psi^{!}) = Q \oplus_{\psi\psi^{!}} Q^{\vee}[1-n] = Q \oplus Q^{\vee}[1-n]$ is simply given by the direct sum of $Q$ and $Q^{\vee}[1-n]$.
The balancedness assumption ensures that one can make the mapping cones $P \oplus_{\psi^{!}} Q^{\vee}[1-n]$ and $Q \oplus Q^{\vee}[1-n]$ into CDGAs, using the product of $P$ (resp.\ $Q$) and the $P$-module (resp. $Q$-module) structure of $Q^{\vee}[1-n]$, and by letting the product of two elements of $Q^{\vee}[1-n]$ be zero. In other words, the product is defined by:
\[ (x,\alpha) \cdot (y,\beta) \coloneqq (xy, x \cdot \beta + \alpha \cdot y). \]

\begin{definition}
The \textbf{pretty model}\index{pretty model} associated to $(P,Q,\psi)$ is the CDGA morphism:
\[ \lambda \coloneqq \psi \oplus \id : P \oplus_{\psi^{!}} Q^{\vee}[1-n] \rightarrow Q \oplus Q^{\vee}[1-n]. \]
\end{definition}

The pair $(M, \partial M)$ admits a \emph{surjective} pretty model if there exists a surjective morphism $\psi : P \to Q$ as above such that there is a commutative diagram of CDGAs:
\begin{equation}
  \begin{tikzcd}
    B \coloneqq P \oplus_{\psi^{!}} Q^{\vee}[1-n] \dar{\lambda \coloneqq \psi \oplus \id}
    & R \lar{f}[swap]{\sim} \rar{\sim}[swap]{g} \dar{\rho}
    & \Omega^{*}(M) \dar{\operatorname{res}} \\
    B_{\partial} \coloneqq Q \oplus Q^{\vee}[1-n]
    & R_{\partial} \lar{f_{\partial}}[swap]{\sim} \rar{\sim}[swap]{g_{\partial}}
    & \Omega^{*}(\partial M)
  \end{tikzcd}
  \label{cnfbnd.eq.zigzag}
\end{equation}
with each row a zigzag of quasi-isomorphisms.

\begin{theorem}[{Cordova Bulens, Lambrechts, and Stanley~\cite{CordovaBulensLambrechtsStanley2015,CordovaBulensLambrechtsStanley2015a}}]
  \label{cnfbnd.thm.pretty-model}
  The pair $(M, \partial M)$ admits a surjective pretty model in the following cases:
  \begin{itemize}
  \item If both $M$ and $\partial M$ are $2$-connected and the boundary $\partial M$ retracts rationally on its half-skeleton~\cite[Definition~6.1]{CordovaBulensLambrechtsStanley2015};
  \item If $M$ (resp.\ $\partial M$) is the associated disk bundle (resp.\ sphere bundle) of a vector bundle of even rank over a simply connected Poincaré duality space;
  \item If $M$ is obtained by removing a tubular neighborhood of a $2$-connected subpolyhedron $K$ from a $2$-connected closed manifold $N$ satisfying $2 \dim K + 3 \le \dim N = \dim M$.
  \end{itemize}
\end{theorem}

By the results from Section~\ref{cnf.sec.poinc-dual-models}, when $M$ is simply connected and $\partial M = \varnothing$, one can take a Poincaré duality model $P$ of $M$ and $Q = 0$ to obtain a surjective pretty model of $(M, \varnothing)$.

The CDGA $B_{\partial}$ is a Poincaré duality CDGA of formal dimension $n-1$, with augmentation $\varepsilon_{B_{\partial}} : B_{\partial} \to \K[1-n]$ given on
\[ (Q \oplus Q^{\vee}[1-n])^{n-1} = (Q^{0})^{\vee} \]
by evaluation on $1_{Q} \in Q^{0}$. In other words the volume form is given by $1_{Q}^{\vee}$.
We will also write $\varepsilon_{B} : B \to \K[-n]$ for the linear map given by $\varepsilon_{B}(x,y) = \varepsilon_{P}(x)$.

\begin{remark}
  \label{cnfbnd.rmk.epsilon-chain-map}
  The map $\varepsilon_{B}$ is not a chain map unless $Q = 0$.
  Indeed if $Q \neq 0$, then we have $d(1_{Q}^{\vee}) = \psi^{!}(1_{Q}^{\vee}) = \vol_{P}$, and thus $\varepsilon_{B} \circ d \neq 0$.
  Instead, there is a Stokes-like formula:
  \begin{equation}
    \varepsilon_{B}(d(x,\alpha)) = \varepsilon_{B_{\partial}}(\psi(x), \alpha), \quad \forall x \in P, \; \alpha \in Q^{\vee}[1-n],
  \end{equation}
  which means that $\varepsilon = (\varepsilon_{B}, \varepsilon_{B_{\partial}})$ defines a chain map $\varepsilon : \cone(\psi \oplus \id) \to \R[-n+1]$.
\end{remark}

Let $I \subset P$ be the image of $\psi^{!}$, which is an ideal.
Then when $\psi$ is surjective, the projection $P \oplus_{\psi^{!}} Q^{\vee}[1-n] \to A \coloneqq P/I$ is a quasi-isomorphism.
Thus if $(M, \partial M)$ admits a surjective pretty model, then $M$ admits a model which is a quotient of a Poincaré duality CDGA.\@
It is an open conjecture whether this is the case for all compact manifolds with boundary.

There is an interpretation of Poincaré--Lefschetz duality in this case.
The kernel $K \coloneqq \ker \psi \subset P$ is equal to $\ker(\psi \oplus \id)$, which is a model for
\[ \Omega^{*}(M, \partial M) \coloneqq \hoker(\Omega^{*}(M) \to \Omega^{*}(\partial M)). \]

Recall the isomorphism $\theta_{P} : P \to P^{\vee}[-n]$ from Equation~\eqref{cnfbnd.eq.theta-p}.
It induces dual isomorphisms:
\begin{align}
  \label{cnfbnd.eq.lefschetz}
  \theta_{P} : P/I
  & \tkrightarrow{\cong} K^{\vee}[-n],
  & \vartheta_{P} : K
  & \tkrightarrow{\cong} (P/I)^{\vee}[-n],
\end{align}
that are both induced by the non-degenerate pairing $(P/I) \otimes (\ker \psi) \to \R[-n]$, $x \otimes y \mapsto \varepsilon_{P}(xy)$.

\begin{example}
  \label{cnfbnd.exa.model-dn}
  Consider the manifold $M = D^{n}$ with boundary $\partial M = S^{n-1}$.
  Intuitively, we can see $M$ as a sphere with a (thick) point removed.
  Thus consider the Poincaré duality CDGA $P = H^{*}(S^{n}) = S(\vol_{n}) / (\vol_{n}^{2})$ (with a generator of degree $n$), and let $Q = H^{*}(\mathrm{pt}) = \R$.
  Let also $\psi : P \to Q$ be the augmentation.

  We get as a model for $D^{n}$ the mapping cone $B = P \oplus_{\psi^{!}} Q^{\vee}[1-n]$, which is spanned by three generators $1_{P}$, $1_{Q}^{\vee}$, and $\vol_{n}$.
  All the nontrivial products vanish, and $d(1_{Q}^{\vee}) = \vol_{n}$.
  The quotient $A = P/I$ is isomorphic to $H^{*}(D^{n}) = \R$, and $B_{\partial} = Q \oplus Q^{\vee}[1-n]$ is isomorphic to $H^{*}(\partial D^{n}) = H^{*}(S^{n-1})$ -- both manifolds being formal.
\end{example}

\subsection{Compactified configuration spaces}
\label{cnfbnd.sec.comp-conf-spac}

Recall the Fulton--MacPherson operad $\FM_{n}$ from Section~\ref{cnf.sec.little-disks}, defined using compactified configuration spaces.
Recall also its right modules $\FM_{M}$, defined when $M$ is a closed framed manifold.
These constructions generalize to manifolds with boundary.
Given a manifold with boundary $(M, \partial M)$ and finite sets $U, V$, define the colored configuration spaces:\index{ConfUVM@$\Conf_{U,V}(M)$}
\begin{equation}
  \Conf_{U,V}(M) \coloneqq \{ c \in \Conf_{U \sqcup V}(M) \mid c(U) \subset \partial M, \; c(V) \subset \mathring{M} \}.
\end{equation}

If $M$ is a compact manifold, then this configuration space can be compactified into $\SFM_{M}(U,V)$.\index{SFMM@$\SFM_{M}$}
This is a stratified manifold of dimension $(n-1) \cdot \# U + n \cdot \# V$.
Note that $\SFM_{M}(\varnothing, \{*\}) \cong M$, as a point may be infinitesimally close to the boundary of $M$.

Similarly if $M = \mathbb{H}^{n} \subset \R^{n}$ is the upper half-space, then $\Conf_{U,V}(M)$, after modding it out by the group of translations preserving $\mathbb{H}^{n}$ and positive dilatations, can be compactified into a space $\SFM_{n}(U,V)$~\cite{Voronov1999}\index{SFMn@$\SFM_{n}$}.
It is a stratified manifold of dimension $n \cdot \# V + (n-1) \cdot \# U - n$ as soon as $\#U + 2 \#V \ge 2$, and it is reduced to a point otherwise.
One has homeomorphisms $\SFM_{n}(U, \varnothing) \cong \FM_{n-1}(U)$ and $\SFM_{n}(\underline{1}, \underline{1}) \cong D^{n-1}$, and homotopy equivalences $\SFM_{n}(\varnothing, V) \simeq \FM_{n}(V)$.

In this way, one obtains a relative operad $\SFM_{n}$ over $\FM_{n}$, weakly equivalent to the Swiss-Cheese operad~\cite{Voronov1999} where operations with no open inputs are allowed (see~\cite[Section~3]{HoefelLivernet2012} or Remark~\ref{rmk.sc-sc-vor} for an explanation of the difference).
If $M$ is framed, then the collection $\SFM_{M}$ also assemble into a right $\SFM_{n}$-module using insertion of configurations:
\begin{align}
  \circ_{T} : \SFM_{M}(U, V/T) \times \FM_{n}(T)
  & \to \SFM_{M}(U,V)
  & T \subset V; \\
  \circ_{W,T} : \SFM_{M}(U/W, V) \times \SFM_{n}(W,T)
  & \to \SFM_{M}(U, V \sqcup T)
  & W \subset U.
\end{align}

The spaces $\SFM_{n}(U,V)$ are clearly semi-algebraic sets.
Moreover, $M$ itself can be endowed with a semi-algebraic structure.
Indeed, it can be triangulated and is $C^{1}$-diffeomorphic to some simplicial complex in some $\R^{N}$.
Each simplex is a semi-algebraic set, and there is a finite number of them (because $M$ is compact), hence $M$ itself is semi-algebraic.
It follows that the spaces $\SFM_{M}(U,V)$ are semi-algebraic too.
We can thus apply the theory of PA forms (see Section~\ref{cnf.sec.little-disks}) and obtain relative ``almost'' cooperads and comodules $\OmPA^{*}(\SFM_{n})$ and $\OmPA^{*}(\SFM_{M})$.

Moreover, the projections
\begin{equation}
  p_{U,V} : \SFM_{M}(U \sqcup I, V \sqcup J) \to \SFM_{M}(U,V)
\end{equation}
which forget some of the points are semi-algebraic bundles.
Just like in Chapter~\ref{chap.2}, we follow the proof from~\cite[Section~5.9]{LambrechtsVolic2014} step by step, and we work by induction on the number of points forgotten (because SA bundles can be composed, see~\cite[Proposition~8.5]{HardtLambrechtsTurchinVolic2011}).
If we forget a terrestrial point, then this essentially follows from the fact that the projections $\FM_{\partial M}(U \sqcup I) \to \FM_{\partial M}(U)$ are SA bundles.
For aerial points, let us give an idea of the key point of the proof for the projection $p : \SFM_{M}(\varnothing, \underline{2}) \to \SFM_{M}(\varnothing, \underline{1})$ which forgets the point $2$.
Recall that $\SFM_{M}(\varnothing, \underline{1})$ is homeomorphic to $M$.
The fiber of $p$ over a point $x$ of the interior of $M$ is given by $M$ with an open disk centered around $x$.
The fiber over a point $x$ of the boundary has three types of points: either point $2$ is far away from point $1$, or it is infinitesimally close to point $1$.
The set of configurations from the first case is given by $M$ with a half-disk around point $1$ removed.
The set of configurations from the second case is given by the space $\SFM_{n}(\varnothing, \underline{2})$ which specify how the two points are infinitesimally arranged.
The intersection of these two sets is given by $\SFM_{n}(\varnothing, \underline{1})$, i.e.\ point $2$ is infinitesimally close to $1$ but ``almost'' leaves it.
In other words, the fiber is:
\begin{equation}
  p^{-1}(x) = M - \{ \text{half-disk around } x\} \cup_{\SFM_{n}(\varnothing,\underline{1})} \SFM_{n}(\varnothing, \underline{2}),
\end{equation}
which is also homeomorphic to $M$ with an open disk removed, see Figure~\ref{fig.fiber-proj}.

\begin{figure}[htbp]
  \centering
  \def\svgwidth{0.7\textwidth}
  \import{fig/}{fiber-proj.pdf_tex}
  \caption{The fiber of $\SFM_M(\varnothing, 2) \to \SFM_M(\varnothing, 1) = M$ above a point of the boundary: $(M - \{\text{half-disk}\}) \cup_{\SFM_{n}(\varnothing,1)} \SFM_{n}(\varnothing, 2)$}
  \label{fig.fiber-proj}
\end{figure}

\subsection{Graphs and the Swiss-Cheese operad}
\label{cnfbnd.sec.extens-swiss-cheese}

The Swiss-Cheese operad is not formal~\cite{Livernet2015,Willwacher2017a}, thus there can be no quasi-iso\-mor\-phism between $H^{*}(\SFM_{n})$ and $\OmPA^{*}(\SFM_{n})$.
Nevertheless, there is a model $\SGraphs_{n}$\index{SGraphsn@$\SGraphs_{n}$} of $\SFM_{n}$, due to Willwacher~\cite{Willwacher2015a}, which is similar in spirit to the cooperad $\Graphs_{n}$.
The cohomology $H^{*}(\SFM_{n}) \cong H^{*}(\FM_{n}) \otimes H^{*}(\FM_{n-1})$ splits as a Voronov product (see~\cite{Voronov1999} and Section~\ref{sw.sec.homology-dd_n-sc}), and $\SGraphs_{n}$ is a way of intertwining $\Graphs_{n}$ and $\Graphs_{n-1}$ in a way that corrects the lack of formality.

\begin{remark}
  Our notations differ slightly from the notations of~\cite{Willwacher2015a}.
  We call $\SGraphs_{n}$ what is called $\Graphs_{n}^{1}$ there, i.e.\ the space of operations with output of ``type $1$'' (which corresponds to ``open'').
  This is a relative Hopf cooperad over $\Graphs_{n}$ (which would be $\Graphs_{n}^{2}$ in Willwacher's paper, the space of operations with output of ``type $2$'', i.e.\ closed).
\end{remark}

Let us assume from the start that $n \geq 3$ to avoid some difficulties that arise when $n = 2$.
Note that in the $n = 2$ case we have found a different model (Chapter~\ref{chap.1}) for $\SFM_{2}$ which is instead inspired by Tamarkin's proof of the formality of the little $2$-disks operad, obtained by intertwining the operad of parenthesized permutations and the operad of parenthesized chord diagrams.

The idea is to construct a relative Hopf cooperad $\SGra_{n}$ over $\Gra_{n}$, with two types of vertices: aerial ones, corresponding to closed inputs, and terrestrial ones, corresponding to open inputs.
Edges are oriented, and the source of an edge may only be an aerial vertex.
More concretely, one defines:\index{SGran@$\SGra_{n}$}
\begin{equation}
  \SGra_{n}(U,V) \coloneqq S(e_{vu})_{u \in U, v \in V} \otimes S(e_{vv'})_{v, v' \in V}
\end{equation}
where the generators all have degree $n-1$ and the cooperad structure is given by formulas similar to Equation~\eqref{cnf.eq.coop-gra-n}.

A monomial in $\SGra_{n}(U,V)$ can be seen as a directed graph with two kinds of vertices: aerial and terrestrial.
The set $U$ is the set of terrestrial vertices, and the set $V$ is the set of aerial vertices.
Note that unlike graphs in $\Gra_{n}$, double edges (of the type $e_{vu}^{2}$ or $e_{vv'}^{2}$) and loops (also known as tadpoles, of the type $e_{vv}$) are allowed.

This allows us to produce a first morphism $\omega'' : \SGra_{n} \to \OmPA^{*}(\SFM_{n})$.\index{womegaprime@$\omega'(\Gamma)$}
One can define, for $v, v' \in V$, $\omega'(e_{vv'}) \coloneqq p_{vv'}^{*}(\vol_{n-1})$, where
\[ \vol_{n-1} \in \OmPA^{n-1}(\SFM_{n}(\varnothing, \{v,v'\})) \simeq \OmPA^{n-1}(\FM_{n}(\{v,v''\})) \simeq \OmPA^{n-1}(S^{n-1}). \]
Recall that $\SFM_{n}(\{u\}, \{v\})$ is homeomorphic to $D^{n-1}$, and we write $\overline{\vol}^{h}_{n-1}$ for the $(n-1)$-form on $\SFM_{n}(\{u\}, \{v\})$ obtained by pulling back the volume form of $S^{n-1}$ along the map $D^{n-1} \to S^{n-1}$ given by the hyperbolic geodesic (see~\cite[Equation~(8)]{Willwacher2015a}).
Then for $u \in U$ and $v \in V$, define $\omega'(e_{vu}) \coloneqq p_{vu}^{*}(\overline{\vol}^{h}_{n-1})$.

If $\Gamma \in \SGra_{n}(U,V)$ is a graph, let\index{cgamma@$c_{\Gamma}$}
\begin{equation}
  \label{cnfbnd.eq.konts-coeff}
  c(\Gamma) \coloneqq \int_{\SFM_{n}(U,V)} \omega'(\Gamma).
\end{equation}
Note that these are analogous to the coefficients that appear in Kontsevich's universal $L_{\infty}$ formality morphism $T_{\mathrm{poly}} \to D_{\mathrm{poly}}$, defined for $n = 2$~\cite{Kontsevich2003}.

The cooperad $\SGra_{n}$ is then twisted with respect to the sum of the Maurer--Cartan element $\mu \in \Def(\hoLie_{n} \to \Gra_{n})$ with the Maurer--Cartan element defined by $c$, to obtain a relative Hopf cooperad $\Tw \SGra_{n}$\index{TwSGran@$\Tw \SGra_{n}$} over $\Tw \Gra_{n}$.

Concretely, $\Tw \SGra_{n}(U,V)$ is spanned by graphs with four types of vertices: they can be either aerial or terrestrial, and either internal or external.
Internal terrestrial vertices are of degree $1-n$ and indistinguishable among themselves, and internal aerial vertices are of degree $-n$ and indistinguishable among themselves.
Edges remain oriented and of degree $n-1$.
External terrestrial vertices are in bijection with $U$, and external aerial vertices are in bijection with $V$.
The cooperad structure maps collapse subgraphs, and the product glues graphs along external vertices.

\begin{figure}[htbp]
  \centering
  \begin{tikzpicture}
    \node (0) {};
    \node (o) [right = 3cm of 0] {};
    \draw[dotted] (0) to (o);
    \node[extv] (u) [right = 0.5cm of 0] {$u$};
    \node[intv] (i1) [right = 1cm of u] {};
    \node[extv] (v1) [above = 1cm of u] {$v_{1}$};
    \node[extv] (v2) [right = 1cm of v1] {$v_{2}$};
    \node[intv] (i2) [right = 1cm of v2] {};
    \draw[->] (v1) edge (u) edge (i1) edge (v2);
    \draw[->] (v1) to (v2);
    \draw[->] (i2) edge (v2) edge (i1);
  \end{tikzpicture}
  \caption{A colored graph in $\Tw \SGra_{n}(\{u\}, \{v_{1}, v_{2}\})$.
    Terrestrial vertices are drawn on a dotted line to distinguish them, even though they are not ordered.
    This graph is of degree $5(n-1) - n - (n-1)$.}
  \label{cnfbnd.fig.exa-sgra_n}
\end{figure}

The differential has several summands:
\begin{itemize}
\item A first summand (corresponding to $\mu$) contracts edges between two aerial vertices, with at least one being internal.
\item A second summand is given by contracting subgraphs $\Gamma' \subset \Gamma$ with at most one external vertex, which must be terrestrial, to obtain $\Gamma / \Gamma'$, with coefficient $c(\Gamma')$.
  One should note that in $\Gamma/\Gamma'$, the new vertex representing the collapsed subgraph is terrestrial, even if $\Gamma'$ is fully aerial.
\item Finally, a third summand is given by forgetting some internal vertices and keeping a subgraph $\Gamma' \subset \Gamma$, with coefficient $c(\Gamma / \Gamma')$.
\end{itemize}

In the second and third cases, if the graph $\Gamma / \Gamma'$ contains an edge whose source is terrestrial, then the summand is defined to be zero (see~\cite[§6.4.2.2]{Kontsevich2003}).

One then checks that there is an extension $\omega : \Tw \SGra_{n} \to \OmPA^{*}(\SFM_{n})$, given by a formula analogous to Equation~\eqref{cnf.eq:4}.
It remains to mod out by the bi-ideal of graphs with internal components to obtain a Hopf cooperad $\SGraphs_{n}$, and to check that $\omega$ factors through the quotient.

\begin{theorem}[{Willwacher~\cite{Willwacher2015a}}]
  The morphism $\omega : \SGraphs_{n} \to \OmPA^{*}(\SFM_{n})$\index{womega@$\omega(\Gamma)$} is a quasi-isomorphism of relative Hopf cooperads.
\end{theorem}

\subsection{Graph complexes}
\label{cnfbnd.sec.graph-complexes}

There is a dg-module of particular interested appearing in Kontsevich's proof of the formality of $\FM_{n}$.
Define\index{fGCn@$\fGC_{n}$}
\begin{equation}
  \fGC_{n} \coloneqq \Tw \Gra_{n}(\varnothing)[-n]
\end{equation}
to be the full graph complex.
It is spanned by graphs containing only internal vertices, with a degree shift: if $\gamma \in \fGC_{n}$ is a graph with $k$ edges and $l$ vertices, then its degree $k(n-1) - l n + n$.
The differential is given by contracting edges.
Its suspension $\fGC_{n}[n]$ is a CDGA, and the product is the disjoint union of graphs.
Since the differential cannot create new connected components, this a free CDGA, and we have:
\begin{equation}
  \fGC_{n} = S(\GC_{n}[n])[-n]
\end{equation}
where $\GC_{n}$\index{GCn@$\GC_{n}$} is the submodule of connected graph.
The variants $\GC_{n}^{2}$ (where bivalent vertices are allowed but univalent vertices forbidden) and $\GC_{n}^{\circlearrowright}$ (where loops are allowed) are defined similarly.

\begin{remark}
  Our notations are slightly nonstandard.
  It is often the dual complex $\GC_{n}^{\vee}$ which is called the graph complex, with a differential that is dually given by creating internal vertices.
  This dual is a Lie algebra using insertion of graphs.
\end{remark}

The homology of this dg-module is particularly hard to compute.
It is known that $H^{0}(\GC_{2}^{\vee})$ is isomorphic to the Grothendieck--Teichmüller Lie algebra $\mathfrak{grt}_{1}$ and that $H^{<0}(\GC_{2}^{\vee}) = 0$~\cite{Willwacher2014}; it is a conjecture that $H^{1}(\GC_{2}) = 0$.
The vanishing of $H^{*}(\GC^{2,\vee}_{n})$ in some degrees shows that the operad of little $n$-disks is intrinsically formal as a Hopf cooperad~\cite{FresseWillwacher2015}.
And this homology computes the homotopy groups of the space of rational automorphisms of the little $n$-disks operad~\cite{FresseTurchinWillwacher2017}.

There is also a Swiss-Cheese version of that graph complex appearing in the description of Willwacher's model for the Swiss-Cheese operad:\index{fSGCn@$\fSGC_{n}$}\index{SGCn@$\SGC_{n}$}
\begin{equation}
  \label{cnfbnd.eq.fsgcn}
  \fSGC_{n} \coloneqq \Tw \SGra_{n}(\varnothing, \varnothing)[-n] = S(\SGC_{n}[n])[-n]
\end{equation}
is the full Swiss-Cheese graph complex, a symmetric algebra on its subcomplex of connected graphs.
Its dual $\SGC_{n}^{\vee}$ is a Lie algebra using insertion of graphs (and the Lie algebra $\GC_{n}^{\vee}$ acts on $\SGC_{n}$), and is sometimes also denoted $\mathrm{KGC}_{n}$.
The Kontsevich integrals $c \in \SGC_{n}^{\vee}$ define a Maurer--Cartan element in this Lie algebra.

\section{Poincaré--Lefschetz duality models}
\label{cnfbnd.sec.pretty-nice}

\subsection{Motivation}
\label{cnfbnd.sec.motivation}

In Section~\ref{cnfbnd.sec.surj-pretty-models}, we recalled the definition of pretty models.
If $M$ admits a surjective pretty model (see Theorem~\ref{cnfbnd.thm.pretty-model}) induced by $\psi : P \to Q$, then we get a diagram as in Equation~\eqref{cnfbnd.eq.zigzag}:
\begin{equation}
  \label{cnfbnd.eq.zigzag2}
  \begin{tikzcd}
    {} &
    K \coloneqq \ker \lambda
    \ar[d, hook]
    \ar[dl, bend right, dashed, leftrightarrow, "\text{\tiny non degen.}" {sloped,above}, "\text{\tiny pairing}" {sloped,below}]
    \\
    A \coloneqq P/I
    &
    B \coloneqq P \oplus_{\psi^{!}} Q^{\vee}[1-n] \ar[d, "\lambda = \psi \oplus \id", two heads] \ar[l, "\pi", "\sim" swap, two heads]
    & R \ar[l, "f", "\sim" swap] \ar[r, "\sim", "g" swap] \ar[d, "\rho", two heads]
    & \OmPA^{*}(M) \dar{\operatorname{res}} \\
    {}
    & B_{\partial} \coloneqq Q \oplus Q^{\vee}[1-n]
    & R_{\partial} \lar{f_{\partial}}[swap]{\sim} \rar{\sim}[swap]{g_{\partial}}
    & \OmPA^{*}(\partial M)
  \end{tikzcd}
\end{equation}
Recall that $I \subset P$ is the image of $\psi^{!} : Q^{\vee}[1-n] \to P$.
We consider the diagonal cocycle $\Delta_{P} \in P^{\otimes 2}$ as in Equation~\eqref{cnf.eq.def-delta}, and we implicitly view it as an element of $B^{\otimes 2}$ using the inclusion $B \subset P$.
We let $\Delta_{A} \in A^{\otimes 2}$ be the image of $\Delta_{P}$ under the projection $\pi : B \to A$.
Let $K$ be the kernel of $\psi$, which is also the kernel of $\lambda = \psi \oplus \id$.

While they are quite useful, we do not know whether all manifolds with boundary admit a surjective pretty model.
We will deal with a more general notion, which we dub ``Poincaré--Lefschetz duality models'' (or ``PLD models'' for short), and which are available as soon as $\dim M \geq 7$ and $M$ and $\partial M$ are simply connected.
The goal is to obtain a diagram similar to Equation~\eqref{cnfbnd.eq.zigzag2} without assuming that $B$ (resp.\ $B_{\partial}$) is of the form $P \oplus_{\psi^{!}} Q^{\vee}[1-n]$ (resp.\ $Q \oplus Q^{\vee}[1-n]$).

Let us first motivate our definition.
It is always possible to find a surjective model $\rho$ for $M$:\footnote{We can also assume that the maps $f$, $f_{\partial}$ factor through the sub-CDGAs of trivial forms in order to integrate along the fibers of the canonical projections, see \cite[Appendix~C]{CamposWillwacher2016}.}
\begin{equation}
  \label{cnfbnd.eq.surj-mod-rho}
  \begin{tikzcd}
    R \ar[d, two heads, "\rho"] \ar[r, "f", "\sim" swap] & \OmPA^{*}(M) \ar[d, "\operatorname{res}"] \\
    R_{\partial} \ar[r, "f_{\partial}", "\sim" swap] & \OmPA^{*}(\partial M)
  \end{tikzcd}
\end{equation}

We can still define a chain map $(\varepsilon, \varepsilon_{\partial}) : \cone(\rho) \to \R[-n+1]$ by $\varepsilon(x) = \int_{M} f(x)$ and $\varepsilon_{\partial}(y) = \int_{\partial M} f_{\partial}(y)$.
If we let $K_{R} = \ker(\rho)$, this yields dual maps $\theta_{R} : R \to K_{R}^{\vee}[-n]$ and $\vartheta_{R} : K_{R} \to R^{\vee}[-n]$, which are quasi-isomorphisms by Poincaré--Lefschetz duality.
However, they are not necessarily isomorphisms, contrary to the maps $\theta_{B}, \vartheta_{B}$ defined in the setting of surjective pretty models.
This prevents us from carrying out the arguments of Section~\ref{cnfbnd.sec.model-conf-spac}.
The idea of this section is to adapt the proofs of~\cite{LambrechtsStanley2008} in order to obtain a surjective model of $(M, \partial M)$ for which $\theta$ is surjective and $\vartheta$ is injective.

If the morphism $\theta_{R}$ were surjective (which is equivalent to $\vartheta_{R}$ being injective), then we would obtain an induced isomorphism $R / \ker \theta \cong K^{\vee}[-n]$, and the quotient map $R \to R / \ker \theta_{R}$ would be a quasi-isomorphism by the 2-out-of-3 property.
We would thus get an actual isomorphism between a model of $M$ and a model for the homology of $M$.

\subsection{Definition and existence}
\label{cnfbnd.sec.definition}

\begin{definition}
  A \textbf{Poincaré--Lefschetz duality pair} (or ``\textbf{PLD pair''} for short) of formal dimension $n$ is a CDGA morphism $\lambda : B \to B_{\partial}$ between two connected CDGAs, equipped with a chain map\footnote{Recall that this is equivalent to the data of two linear maps $\varepsilon_{B} : B^{n} \to \R$ and $\varepsilon_{B_{\partial}} : B_{\partial}^{n-1} \to \R$ satisfying the ``Stokes formula'' $\varepsilon_{B}(dx) = \varepsilon_{B_{\partial}}(\lambda(x))$ and $\varepsilon_{B_{\partial}}(dy) = 0$.} $\varepsilon : \cone(\lambda) \to \R[1-n]$, and such that:
  \begin{itemize}
  \item The pair $(B_{\partial}, \varepsilon_{B_{\partial}})$ is a Poincaré duality CDGA of formal dimension $n-1$;
  \item Let $K \coloneqq \ker \lambda$, and let $\theta_{B} : B \to K^{\vee}[-n]$ be defined by $\theta_{B}(b)(k) = \varepsilon(bk)$; then we require $\theta_{B}$ to be surjective and to be a quasi-isomorphism.
  \end{itemize}
\end{definition}

If $(B \xrightarrow{\lambda} B_{\partial}, \varepsilon)$ is a PLD pair, then
\begin{itemize}
\item the CDGA $B$ is quasi-isomorphic to its quotient $A \coloneqq B / \ker \theta_{B}$;
\item the map $\theta : A \to K^{\vee}[-n]$ induced by $\theta_{B}$ is an isomorphism of $B^{\otimes 2}$-modules;
\item Equivalently, the pairing $A^{i} \otimes K^{n-i} \to \R$, given by $a \otimes k \mapsto \varepsilon_{B}(ak)$, is non-degenerate for all $i \in \mathbb{Z}$.
\end{itemize}

\begin{example}
  A surjective pretty model is an example of a PLD pair.
\end{example}

\begin{definition}
  A \textbf{Poincaré--Lefschetz duality model}\index{Poincaré--Lefschetz duality model}\index{PLD model} of $(M, \partial M)$ is a PLD pair $\lambda : B \to B_{\partial}$ which is a model of the inclusion $\partial M \subset M$, in the sense that we can fill out the following diagram:
  \[ \begin{tikzcd}
      B \ar[d, "\lambda"] & R \ar[d, "\rho"] \ar[l, dashed, "\sim" swap] \ar[r, "\sim"] & \OmPA^{*}(M) \ar[d, "\text{res}"] \\
      B_{\partial} & R_{\partial} \ar[l, dashed, "\sim" swap] \ar[r, "\sim"] & \OmPA^{*}(\partial M)
    \end{tikzcd}
  . \]
\end{definition}

\begin{remark}
  In that case, $H^{*}(A) \cong H^{*}(M)$, $H^{*}(K) \cong H^{*}(M, \partial M)$, and the isomorphism $\theta$ is given by Poincaré--Lefschetz duality $H^{*}(M) \cong H_{n-*}(M, \partial M)$.
\end{remark}

\begin{proposition}
  \label{cnfbnd.prop.more-general-model}
  Let $M$ be a simply connected $n$-manifold with simply connected boundary, and assume that $n \geq 7$.
  Then $(M, \partial M)$ admits a PLD model.
\end{proposition}

\begin{proof}
  We start with  some surjective model $\rho : R \to R_{\partial}$ as in Equation~\eqref{cnfbnd.eq.surj-mod-rho}, and we will build a PLD model out of it.

  We keep the terminology of~\cite{LambrechtsStanley2008}.
  We can find a surjective quasi-iso\-mor\-phism $g_{\partial} : R_{\partial} \to B_{\partial}$, where $B_{\partial}$ is a Poincaré duality CDGA, by~\cite{LambrechtsStanley2008}.
  We let $K_{R}$ be the kernel of $\rho''' \coloneqq g_{\partial} \circ \rho$ and we define the chain map $\varepsilon_{R} : \cone(\rho') \to \R[-n+1]$.
  Let $\OO \coloneqq \ker \vartheta_{R} \subset K_{R}$ be the ideal of ``orphans'', i.e.\
  \begin{equation}
    \OO \coloneqq \ker \vartheta_{R} = \{ y \in K_{R} \mid \forall x \in R, \; \vartheta_{R}(y)(x) = \varepsilon_{R}(xy) = 0 \}.
  \end{equation}
  We could consider (for a moment) the new short exact sequence:
  \begin{equation}
    0 \to (K \coloneqq K_{R}/\OO) \to (B \coloneqq R/\OO) \tkrightarrow{\lambda} B_{\partial} \to 0.
  \end{equation}
  There is an induced chain map $\varepsilon_{B} : \cone(\lambda) \to \R[-n+1]$, and $\vartheta_{B} : K \to B^{\vee}[-n]$ is injective because we killed all the orphans.
  Thus we do obtain an isomorphism $B / \ker \theta_{B} \cong K^{\vee}[-n]$ induced by $\theta_{B}$.

  The problem is that the ideal $\OO$ is not necessarily acyclic, thus $\lambda$ is not necessarily a model of $(M, \partial M)$ anymore.
  Indeed, by Poincaré--Lefschetz duality, all we know is that a cycle $o \in \OO$ is always the boundary of some element $z \in K$; but one may not always choose $z \in \OO$.
  The idea, just like in~\cite{LambrechtsStanley2008}, is to extend the CDGA $R$ by acyclic cofibrations (over $B_{\partial}$) in order to make the ideal of orphans acyclic.

  Thanks to our connectivity assumptions on the manifold and its boundary, we can assume that the model $\rho : R \to R_{\partial}$ of $(M, \partial M)$ and the chain map $\varepsilon_{R} : \cone(\rho') \to \R[-n+1]$ satisfy:
  \begin{itemize}
  \item $\rho$ is surjective, hence so is $\rho' \coloneqq g_{\partial} \circ \rho$, and we let $K_{R} = \ker(\rho')$;
  \item both $R$ and $R_{\partial}$ are of finite type;
  \item both $R$ and $R_{\partial}$ are simply connected, i.e.\ $R^{0} = R_{\partial}^{0} = \R$ and $R^{1} = R_{\partial}^{1} = 0$;
  \item we have $R_{\partial}^{2} \subset \ker d$ and $K^{2} \subset \ker d$;\footnote{Since we require $\rho$ to be surjective, it is possible that not all elements of $R^{2}$ are cycles, because some of the classes from $H^{2}(\partial M)$ may need to be killed. Check Example~\ref{cnfbnd.exa.model-dn} when $M = D^{3}$.}
  \item the morphisms $\theta_{R}$, $\vartheta_{R}$ induced by $\varepsilon_{R}$ are quasi-isomorphisms.
  \end{itemize}

  We say that the pair $(\rho,\varepsilon_{R})$ is a ``good'' pair if it satisfies all these assumptions.
  Let us say that its orphans are $k$-half-acyclic if $H^{i}(\OO) = 0$ for $n/2 +1 \leq i \leq k$.
  This condition is void when $k = n/2$, and if $k = n+1$ then Poincaré--Lefschetz duality implies that $\OO$ is acyclic (see \cite[Proposition~3.6]{LambrechtsStanley2008}).

  We will now work by induction (starting at $k = n/2$) and we assume that we are given a good pair $(\rho,\varepsilon_{R})$ whose orphans are $(k-1)$-half-acyclic.
  We will build an extension:
  \begin{equation}
    \begin{tikzcd}
      0 \rar
      & K_{R} \rar \dar[hookrightarrow]{\sim}
      & R \rar{\rho} \dar[hookrightarrow]{\sim}
      & R_{\partial} \rar \dar{=}
      & 0
      \\
      0 \rar
      & \hat{K}_{R} \rar
      & \hat{R} \rar{\hat{\rho}}
      & R_{\partial} \rar
      & 0
    \end{tikzcd}
  \end{equation}
  and an extension $\hat{\varepsilon}_{R} : \cone(\hat{\rho'}) \to \R[-n+1]$ of $\varepsilon_{R}$ such that $(\hat{\rho},\hat{\varepsilon})$ is good and its orphans are $k$-half-acyclic.

  We follow closely~\cite[Sections~4 and~5]{LambrechtsStanley2008}, adapting the proof where needed.
  Let $l = \dim(\OO^{k} \cap \ker d) - \dim(d(\OO^{k-1}))$ and choose $l$ linearly independent cycles $\alpha_{1}, \dots, \alpha_{l} \in \OO^{k}$ such that
  \begin{equation}
    \OO^{k} \cap \ker d = d(\OO^{k-1}) \oplus \langle \alpha_{1}, \dots, \alpha_{l} \rangle.
  \end{equation}
  These are the obstructions to $\OO$ being $k$-half acyclic.
  Because $\vartheta_{R}$ is a quasi-isomorphism and $\vartheta_{R}(\alpha_{i}) = 0$, there exists $\gamma'_{i} \in K^{k-1}$ such that $d\gamma'_{i} = \alpha_{i}$.

  Let $m \coloneqq \dim H^{*}(R) = \dim H^{*}(K_{R})$, and choose cycles $h_{1}, \dots, h_{m} \in R$ such that $([h_{1}], \dots, [h_{m}])$ is a basis of $H^{*}(R)$.
  By Poincaré--Lefschetz duality there exists cycles $h'_{1}, \dots, h'_{m}$ in $K_{R}$ such that $\varepsilon(h_{i} h'_{j}) = \delta_{ij}$ (and these form a basis for $H^{*}(K)$).
  Let
  \begin{equation}
    \gamma_{i} \coloneqq \gamma'_{i} - \sum_{j} \varepsilon(\gamma'_{i} h_{j}) h'_{j} \in K^{k-1},
  \end{equation}
  and let $\Gamma$ be the subspace of $K^{k-1}$ generated by the $\gamma_{i}$.
  Then a proof similar to the proof of~\cite[Lemma~4.1]{LambrechtsStanley2008} shows that $d\gamma_{i} = \alpha_{i}$, and if $y \in R$ is a cycle, then $\varepsilon(\gamma_{i} y) = \vartheta(\gamma_{i})(y) = 0$.

  Now let
  \begin{equation}
    \hat{R} \coloneqq \bigl( R \otimes S(c_{1}, \dots, c_{l}, w_{1}, \dots, w_{l}), dc_{i} = \alpha_{i}, dw_{i} = c_{i} - \gamma_{i} \bigr),
  \end{equation}
  where the $c_{i}$ and the $w_{i}$ are new variables of degrees $k-1$ and $k-2$.

  Extend $\rho$ to $\hat{\rho} : \hat{R} \to R_{\partial}$ by declaring that $\hat{\rho}(c_{i}) = \hat{\rho}(w_{i}) = 0$.
  It follows that
  \begin{equation}
    \hat{K}_{R} \coloneqq \ker \hat{\rho} = (K_{R} \otimes S(c_{i}, w_{j})_{1 \leq i, j \leq l}, d).
  \end{equation}
  It is not hard to see that $R \hookrightarrow \hat{R}$ is an acyclic cofibration (compare with~\cite[Lemma~4.2]{LambrechtsStanley2008}), and so is $K \hookrightarrow \hat{K}$.
  Finally, since all the $\gamma_{i}$ and $\alpha_{i}$ are in $\ker \hat{\rho}$, one can extend $\varepsilon_{R}$ to $\hat{\varepsilon}_{R} : \cone(\hat{\rho}) \to \R[-n+1]$ by formulas identical to~\cite[Equation~4.5]{LambrechtsStanley2008}, which works because $n \geq 7$.
  It is clear that $(\hat{\rho}, \hat{\varepsilon}_{R})$ is still a good pair.
  We let $\hat{\theta}_{R}$, $\hat{\vartheta}_{R}$ be the quasi-isomorphisms induced by the pairing, and we finally let $\hat{\OO} \coloneqq \ker \hat{\vartheta}_{R}$.

  It remains to check that $\hat{\OO}$ is $k$-half-acyclic, knowing that $\OO$ is $(k-1)$-half-acyclic.
  First, we can reuse the proofs of~\cite[Lemmas~5.2 and~5.3]{LambrechtsStanley2008} to check that if $i > n-k+2$ then $\OO^{i} \subseteq \hat{\OO}^{i}$, and if $i \in \{ k-2, k-1 \}$ then $\hat{\OO}^{i} \cap \ker d \subset \OO^{i} \cap \ker d$.
  The only difference is that instead of the hypothesis $d(R^{2}) = 0$, we instead have $d(K_{R}^{2}) = 0$.

  Now, just like \cite{LambrechtsStanley2008}, we have several cases to check.
  If we have $k \geq n/2 + 2$, or if we have $n$ odd and $k = (n+1)/2 + 1$, then the proof goes through unchanged up to slight changes of notation.
  However, if $n$ is even and $k = n/2 + 1$, more significant adaptations are needed, even if the idea is the same.
  We'd like to check that $\hat{\OO}^{k} \cap \ker d \subset d(\hat{\OO}^{k-1})$.
  We already know that
  \begin{equation}
    \hat{\OO}^{k} \cap \ker d \subset \OO^{k} \cap \ker d = d(\OO^{k-1}) \oplus \langle \alpha_{i} \rangle.
  \end{equation}
  Since $\OO^{k-1} \subset \hat{\OO}^{k-1}$ (the proof is identical to~\cite[Lemma~5.7]{LambrechtsStanley2008}), it is sufficient to check that $\hat{\OO}^{k-1} \cap \langle \alpha_{i} \rangle = 0$.
  Let us write $j : K_{R} \to R$ for the inclusion, $\bar{R} \coloneqq R / \OO$, $\bar{K}_{R} \coloneqq K_{R} / \OO$, $n = 2m$ and $k = n/2 + 1 = m + 1$.
  Then we have long exact sequences in cohomology and morphisms between them:
  \begin{equation}
    \begin{tikzcd}
      0 \rar
      & H^{m}(R) \rar{\pi}
      & H^{m}(\bar{R}) \rar{\delta}
      & H^{m+1}(\OO) \rar
      & 0
      \\
      0 \rar
      & H^{m}(K) \rar{\pi} \uar{j}
      & H^{m}(\bar{K}_{R}) \rar{\delta} \uar{\bar{\jmath}}
      & H^{m+1}(\OO) \rar \uar{=}
      & 0
    \end{tikzcd}
  \end{equation}

  The space $H^{m+1}(\OO)$ is generated by the classes of the $\alpha_{i}$.
  We obtain a section $\sigma : H^{m+1}(\OO) \to H^{m}(\bar{K}_{R})$ of $\delta$ by letting $\sigma([\alpha_{i}]) \coloneqq [\gamma_{i}]$.

  Suppose that we have some nonzero element $\alpha \in \langle \alpha_{i} \rangle$; we'd like to show that it is not in $\hat{\OO}^{k}$, i.e.\ that it is not an orphan in $\hat{K}_{R}$.
  The pairing $H^{m}(\bar{K}_{R}) \otimes H^{m}(\bar{R}) \to \R$ induced by $\bar{\varepsilon}_{R}$ is non-degenerate, and $\sigma(\alpha) \neq 0$, hence there exists some $\beta \in H^{m}(\bar{R})$ such that $\varepsilon(\sigma(\alpha)\beta) \neq 0$.
  But as we saw before, for any $[h] = \sum x_{j} [h_{j}] \in H^{m}(R)$, $\varepsilon_{R}(\sigma(\alpha) h) = 0$, because $\varepsilon_{R}(\gamma_{i} h_{j}) = 0$.
  It follows that $\varepsilon_{R}(\sigma(\alpha) \sigma(\delta(\beta))) = \varepsilon_{R}(\sigma(\alpha) \beta) \neq 0$.
  If we write $\delta(\beta) = \sum_{j} \beta_{j} \gamma_{j} \in H^{m+1}(\OO)$, we can let $w = \sum_{j} \beta_{j} w_{j} \in \hat{R}$, and then by definition $\hat{\varepsilon}_{R}(\alpha w) = \varepsilon_{R}(\sigma(\alpha) \sigma(\delta(\beta))) \neq 0$, thus $\alpha$ is not an orphan.
  This completes the proof that $\hat{\OO}^{k-1} \cap \langle \alpha_{i} \rangle = 0$, and thus that $\hat{\OO}^{k} \cap \ker d \subset d(\hat{\OO}^{k-1})$.
  We have thus covered all the cases to prove that if $\OO$ is $(k-1)$-half-acyclic, then $\hat{\OO}$ is $k$-half-acyclic.

  By induction, we obtain a good pair $(\hat{\rho} : \hat{R} \to R_{\partial}, \hat{\varepsilon}_{R})$ whose ideal of orphans is $(n+1)$-half acyclic (and hence actually acyclic), obtained by a sequence of acyclic cofibrations from $R$.
  It then remains to define $B \coloneqq \hat{R} / \hat{\OO}$ and $\varepsilon_{B}$ to be the map induced by $\hat{\varepsilon}_{R}$ on the quotient to prove Proposition~\ref{cnfbnd.prop.more-general-model}.
\end{proof}

Given a PLD model of $(M, \partial M)$, we obtain a diagram (similar to~\eqref{cnfbnd.eq.zigzag2}):
\begin{equation}
  \label{cnfbnd.eq.nice-diagram}
  \begin{tikzcd}
    {} &
    K \coloneqq \ker \lambda
    \ar[d, hook]
    \ar[dl, bend right, dashed, leftrightarrow, "\text{\tiny non degen.}" {sloped,above}, "\text{\tiny pairing}" {sloped,below}]
    \\
    A \coloneqq B / \ker \theta_{B} &
    B \ar[l, "\sim" swap, "\pi", two heads] \ar[d, "\lambda", two heads] &
    R \ar[l,"g", "\sim" swap] \ar[d, "\rho", two heads] \ar[r, "f" swap, "\sim"] &
    \OmPA^{*}(M) \ar[d, "\operatorname{res}"]
    \\
    {} &
    B_{\partial} &
    R_{\partial} \ar[l, "g_{\partial}" "\sim" swap] \ar[r, "f_{\partial}" swap, "\sim"] &
    \OmPA^{*}(\partial M)
  \end{tikzcd}
\end{equation}

We also see in the proof that $\varepsilon_{B}$ satisfies by construction $\varepsilon_{B} g = \int_{M} f(-)$.
We also have $\varepsilon_{B_{\partial}} g_{\partial} = \int_{\partial M} f_{\partial}(-)$ by construction (see Remark~\ref{cnf.rmk.commute}).

\subsection{Diagonal classes}
\label{cnfbnd.sec.diagonal-classes}

Let $(B \xrightarrow{\lambda} B_{\partial}, \varepsilon)$ be a Poincaré--Lefschetz duality pair, with $K = \ker \lambda$.
Recall that we write $\theta : (A = B / \ker \theta_{B}) \to K^{\vee}[-n]$ for the isomorphism of $B^{\otimes 2}$-modules induced by the surjective $\theta_{B} : B \to K^{\vee}[-n]$.

The multiplication $\mu_{K} : K^{\otimes 2} \to K$ can then be dualized into a morphism of $B^{\otimes 2}$-modules $\delta : A \to A^{\otimes 2}[-n]$, and we let\index{diagonal class@$\Delta_{A}$, diagonal class}
\begin{equation}
  \Delta_{A} \coloneqq \delta(1_{A}) \in (A \otimes A)^{n}
\end{equation}
be a representative of the diagonal class.
Graded commutativity of $\mu_{K}$ implies that $\Delta_{A}^{21} = (-1)^{n} \Delta_{A}$.
The fact that $\delta$ is a morphism of $B^{\otimes 2}$-modules imply that $\Delta_{A}$ satisfies
\begin{equation}
  \forall a \in A, \; \delta(a) = (a \otimes 1) \Delta_{A} = \Delta_{A} (1 \otimes a).
\end{equation}

\begin{remark}
  If the PLD model comes from a surjective pretty model, then we check that this diagonal class is indeed the image of $\Delta_{P} \in P^{\otimes 2} \subset B^{\otimes 2}$ under the projection $\pi : B \to A$.
\end{remark}

The duality between $A$ and $K$ can also be turned into a cocycle $\Delta_{AK} \in A \otimes K$.
Let $\{ a_{i} \}$ be some basis of $A$, and let $\{a_{i}^{*}\}$ be the dual basis of $K$, satisfying $\varepsilon_{B}(a_{i} a_{i}^{*}) = \delta_{ij}$.
Then we define:\index{DeltaAK@$\Delta_{AK}$}
\begin{equation}
  \label{cnfbnd.def.delta-ak}
  \Delta_{AK} \coloneqq \sum \pm a_{i} \otimes a_{i}^{*} \in (A \otimes K)^{n}.
\end{equation}
It is not hard to check that $\Delta_{A}$ is the image of $\Delta_{AK}$ under the composite map $1 \otimes \pi \circ \iota : A \otimes K \to A \otimes B \to A \otimes A$, and by definition:
\begin{equation}
  \forall a \in A, \sum_{(\Delta_{AK})} \varepsilon_{B}(a \Delta_{AK}'') \Delta_{AK}' = a.
\end{equation}

Note also that, under the multiplication map $\mu_{A} : A \otimes A \to A$, we have $\mu_{A}(\Delta_{A}) = \sum \pm \pi(a_{i}a_{i}^{*}) = \chi(A) \pi(\vol_{K})$ is equal to the Euler characteristic of $B$ multiplied by the volume form $\vol_{K} \in K$ (the only element of $B$ satisfying $\varepsilon_{B}(\vol_{K}) = 1$, representing the fundamental class of $(M, \partial M)$).
But by degree reasons, $\vol_{K} \in K \subset B$ is in the kernel of $\theta_{B} : B \to K^{\vee}[-n]$ when $\partial M \neq \varnothing$, hence $\pi(\vol_{K}) = 0$ by definition.
In other words,
\begin{equation}
  \label{cnfbnd.eq.mu-delta-zero}
  \mu_{A}(\Delta_{A}) = 0 \text{ if } \partial M \neq \varnothing.
\end{equation}

\section{The model and its cohomology}
\label{cnfbnd.sec.model-conf-spac}

From now on, let us assume that $M$ is a smooth, simply connected manifold with nonempty boundary.
We also fix a PLD model of $M$ as in Equation~\eqref{cnfbnd.eq.nice-diagram}.

\subsection{The dg-module model of \texorpdfstring{$\Conf_{k}(M)$}{Conf\_k(M)}}
\label{cnfbnd.sec.dgmodel}

Given a finite set $V$ and an element $v \in V$, define the canonical injection $\iota_{v} : A \to A^{\otimes V}$\index{iota@$\iota_{v}$} by
\begin{equation}
  \iota_{v}(a) \coloneqq 1 \otimes \dots \otimes 1 \otimes \underbrace{a}_{v} \otimes 1 \otimes \dots \otimes 1.
\end{equation}

\begin{definition}
  \label{cnfbnd.def.ga}
  Define a symmetric collection of CDGAs $\GG{A}$\index{GA@$\GG{A}$, Lambrechts--Stanley CDGA} by:
  \begin{equation*}
    \GG{A}(V) \coloneqq \bigl( A^{\otimes V} \otimes \enV(V) / (\iota_{v}(a) \cdot \omega_{vv''} = \iota_{v'}(a) \cdot \omega_{vv'}), \; d(\omega_{vv'}) = (\iota_{v} \cdot \iota_{v'})(\Delta_{A}) \bigr)
  \end{equation*}
  with the obvious actions of the symmetric groups.
\end{definition}

This definition also makes sense when $\partial M = \varnothing$, and it yields the symmetric collection of CDGAs considered in Section~\ref{cnf.sec.model}.

When $M$ is a closed manifold, as soon as its Euler characteristic $\chi(M)$ vanishes, then there is a structure of Hopf right $\enV$-comodule on $\GG{A}$ (Proposition~\ref{cnf.prop.comodule-eA}), with cocomposition structure maps characterized by (compare with Equation~\eqref{cnf.eq.coop-gra-n}):
\begin{equation}
  \begin{aligned}
    \circ^{\vee}_{T}(\omega_{vv'})
    & = 1 \otimes \omega_{vv'},
    & \text{if } \{v, v'\} \subset T; \\
    \circ^{\vee}_{T}(\omega_{vv'})
    & = \omega_{[v][v']} \otimes 1,
    & \text{if } \{v,v'\} \not \subset T; \\
    \circ_{T}^{\vee}(\iota_{v}(a))
    & = \iota_{[v]}(a)
    & \text{for } a \in A, \; v \in V;
  \end{aligned}
\end{equation}
where $[v] \in V/T$ is the class of $v$ in the quotient.

\begin{proposition}
  \label{cnfbnd.prop.ga-comod}
  If $\partial M \neq \varnothing$, then the symmetric collection of CDGAs $\GG{A}$ forms a right Hopf $\enV$-comodule, with the same formulas.
\end{proposition}
\begin{proof}
  Comparing with the proof of Proposition~\ref{cnf.prop.comodule-eA}, we see that almost all the arguments are the same.
  The only difficulty is to check that the cocomposition is compatible with the differential, which required that the Euler characteristic vanished in the boundaryless case.

  It is immediate to check that $d(\circ_{T}^{\vee}(\iota_{v}(a))) = \circ_{T}^{\vee}(d(\iota_{v}(a))) = \iota_{[v]}(da)$, thus it suffices to check that the same equality holds on the generators $\omega_{vv'}$.
  If either $v \not\in T$ or $v' \not\in T$, this is again immediate; hence it suffices to check that
  \[ d(\circ_{T}^{\vee}(\omega_{vv'})) = \circ_{T}^{\vee}(d(\omega_{vv'})) \textnormal{ for } v,v' \in T \]

  The LHS of that equation always vanishes.
  On the other hand, the RHS is equal to $\iota_{*}(\mu_{A}(\Delta_{A}))$, where $\mu_{A} : A \otimes A \to A$ is the product.
  But by Equation~\eqref{cnfbnd.eq.mu-delta-zero}, $\mu_{A}(\Delta_{A}) = 0$.
\end{proof}

\begin{example}
  \label{cnfbnd.exa.model-dn-deux}
  Recall from Example~\ref{cnfbnd.exa.model-dn} the model for $(D^{n}, S^{n-1})$, with $A = \R$, and $\Delta_{A} = 0$.
  It follows that in this case, $\GG{A}$ is isomorphic to $\enV$ seen as a Hopf right comodule over itself.
  This is not surprising, given that $\FM_{n}$ is formal as an operad, and hence as a module over itself, and that $\SFM_{D^{n}}(\varnothing, -)$ is weakly equivalent to $\FM_{n}$ as a right $\FM_{n}$-module.
\end{example}

\subsection{Computing the homology}
\label{sec.computing-homology}

We now prove that $\GG{A}$ has the right cohomology, in the spirit of~\cite{LambrechtsStanley2008a} and using the methods of~\cite{CordovaBulensLambrechtsStanley2015a} to deal with manifolds with boundary.
From then on and until the end of this section, we fix some integer $k \geq 0$.
We can work over $\Q$ in this section.

The general idea goes as follows.
If $W$ is a manifold with boundary, and $X \subset W$ is a sub-polyhedron, then by~\cite{CordovaBulensLambrechtsStanley2015a}, it suffices to know a CDGA model of the square of inclusions
\begin{equation}
  \label{eq.square}
  \begin{tikzcd}
    \partial W \ar[hook]{r} & W \\
    \partial_{W}X \coloneqq X \cap \partial W \ar[hook]{r} \ar[hook]{u} & X \ar[hook]{u}
  \end{tikzcd}
\end{equation}
to obtain a complex computing the cohomology of $W - X$.

Therefore, to compute the cohomology of $\Conf_{k}(M)$, we need to find such models for $W = M^{k}$ and $X = \Delta_{(k)} \coloneqq \bigcup_{1 \le i, j \le k} \Delta_{ij}$\index{Deltak@$\Delta_{(k)}$}, where $\Delta_{ij} \coloneqq \{ x \in M^{k} \mid x_{i} = x_{j} \}$\index{Deltaij@$\Delta_{ij}$}.
Since the sub-polyhedron $\Delta_{(k)}$ can be decomposed into the sub-polyhedra $\Delta_{ij}$, we can use the techniques of~\cite{LambrechtsStanley2008a} to further simplify the description of the dg-module model as a ``total cofiber'' indexed by graphs, which will be isomorphic to $\GG{A}$.

Let us now give the details.
Let $E = \{ (i,j) \mid 1 \leq i < j \leq k \}$ be a set of pairs, and let $\Gamma$ be the poset of subsets of $E$ ordered by reverse inclusion.
We can see an element $\gamma \in \Gamma$ as a graph on $k$ vertices, with an edge between $i$ and $j$ iff $(i,j) \in \gamma$.
In particular $\varnothing \in \Gamma$ is the ``empty'' graph with no edges (but $k$ vertices).
Using this point of view, we can define the ``zeroth homotopy group'' $\pi_{0}(\gamma)$ of a graph $\gamma \in \Gamma$, which is a partition of $\{1, \dots, k\}$.

We obtain a functor $\nabla$\index{$\nabla(\gamma)$} from $\Gamma$ to the category of topological spaces defined by
\begin{equation}
  \gamma \mapsto \nabla(\gamma) \coloneqq \bigcap_{e \in E_{\gamma}} \Delta_{e} \subset M^{k},
\end{equation}
where $\Delta_{(i,j)}$ is simply the small diagonal $\Delta_{ij}$.
Note that $\nabla(\varnothing) = M^{k}$, and that if $\gamma' \supset \gamma$ then there is an inclusion $\nabla(\gamma') \subset \nabla(\gamma)$.
The space $\nabla(\gamma)$ is homeomorphic to the product $M^{\pi_{0}(\gamma)}$, and under these homeomorphisms, the inclusion $\nabla(\gamma') \subset \nabla(\gamma)$ is the cofibration induced by iterations of the diagonal map $M \to M \times M$.
We thus obtain that:
\begin{equation}
  \label{cnfbnd.eq.delta-colim}
  \Delta_{(k)} = \bigcup_{1 \leq i < j \leq k} \Delta_{ij} = \colim_{\gamma \in \Gamma} \nabla(\gamma) = \colim_{\gamma \in \Gamma} M^{\pi_{0}(\gamma)},
\end{equation}
and this is in fact a homotopy colimit.

We will first aim to build a CDGA model for the square~\eqref{eq.square} (with $W = M^{k}$ and $X = \Delta_{(k)}$) out of the diagram~\eqref{cnfbnd.eq.zigzag2}.
The previous description of $\Delta_{(k)}$ as a homotopy colimit tells us that a model for $\Delta_{(k)}$ is given by $\lim_{\gamma \in \Gamma^{\mathrm{op}}} B^{\otimes \pi_{0}(\gamma)}$, where the maps in the diagram are induced by iterations the multiplication $\mu_{B}$ of $B$.
The inclusion $\Delta_{(k)}$ is modeled by the canonical map from $B^{\otimes k} = B^{\otimes \pi_{0}(\varnothing)}$ to the colimit.

It remains to find a model for $\partial_{W} X = \Delta_{(k)} \cap \partial(M^{k})$ and models for the inclusion maps.
The morphism $B \to B_{\partial}$ is surjective, hence $B_{\partial}$ is isomorphic to $B / K$ where $K \coloneqq \ker (B \to B_{\partial})$.
We get:

\begin{lemma}
  \label{cnfbnd.lem.square}
  For all $i \geq 0$, the left-hand side square is a CDGA model for the right-hand side square, where the horizontal maps are the diagonal maps:
  \[ \begin{tikzcd}[cramped]
      B \ar[two heads, d] & B^{\otimes i} \ar[two heads, d] \ar[l, "\mu_{B}^{(i)}" swap] \\
      B / K & B^{\otimes i} / K^{\otimes i} \ar[l, "\mu_{B}^{(i)}" swap]
    \end{tikzcd}
    \text{ is a model for }
    \begin{tikzcd}[cramped]
      M \ar[r, hook, "\delta"] & M^{i} \\
      \partial M \ar[hook, u] \ar[r, hook, "\delta"] & \partial \bigl( M^{i} \bigr) \ar[hook, u]
    \end{tikzcd} \]
\end{lemma}
\begin{proof}
  The idea is the same as in~\cite[Proposition~5.1]{CordovaBulensLambrechtsStanley2015a}.
  We work by induction.

  The case $i = 1$ is obvious (as $B_{\partial} \cong B/K$), and they prove the case $i = 2$.
  Now let use assume that the proposition is true for a given $i \geq 2$.
  There is a diagram, where all the inclusions are either induced by diagonal maps or induced by $\partial M \subset M$:
  \[ \begin{tikzcd}[cramped]
      M \times M^{i} && M \ar[ll, hook] \\
      \partial(M \times M^{i}) \ar[u, hook] & M \times \partial(M^{i}) \ar[l, hook] \\
      (\partial M) \times M^{i} \ar[u, hook] & (\partial M) \times (\partial(M^{i})) \ar[l, hook] \ar[u, hook] \ar[ul, phantom, "\text{\tiny (ho.\ pushout)}"] & \partial M \ar[l, hook] \ar[uu, hook]
    \end{tikzcd}. \]
  The diagram of the proposition is the ``outer'' diagram, and the bottom left square is a (homotopy) pushout.

  Let $P$ be the (homotopy) pullback in CDGAs of
  \[ B_{\partial} \otimes B^{\otimes i} \to B_{\partial} \otimes (B^{\otimes i} / K^{\otimes i}) \gets B \otimes (B^{\otimes i} / K^{\otimes i}). \]
  Then the induction hypothesis and the fact that homotopy pushouts of spaces become homotopy pullbacks of models imply that the following diagram is a CDGA model of the previous one (where the maps are either induced by $\mu_{B} : B^{\otimes 2} \to B$ or $\lambda : B \to B_{\partial}$):
  \[ \begin{tikzcd}[cramped]
      B \otimes B^{\otimes i} && B \ar[ll, leftarrow] \\
      P \ar[u, leftarrow] & B \otimes (B^{\otimes i} / K^{\otimes i}) \ar[l, leftarrow] \\
      B_{\partial} \otimes B^{\otimes i} \ar[u, leftarrow] & B_{\partial} \otimes (B^{\otimes i} / K^{\otimes i}) \ar[l, leftarrow] \ar[u, leftarrow] \ar[ul, phantom, "\text{\tiny (ho.\ pullback)}"] & B_{\partial} \ar[l, leftarrow] \ar[uu, leftarrow]
    \end{tikzcd} \]

  Now, as in the proof \cite[Lemma~5.3]{CordovaBulensLambrechtsStanley2015a}, it is clear that the natural map $B^{\otimes (i+1)} \to P$ is surjective and that its kernel is $K^{\otimes(i+1)}$, in other words that $P \cong B^{\otimes(i+1)} / K^{\otimes(i+1)}$.
  The proposition then follows immediately.
\end{proof}

The space $\Delta_{(k)} \cap \partial(M^{k})$ admits a description as a colimit similar to the one Equation~\ref{cnfbnd.eq.delta-colim}.
Indeed, a point of $M^{k}$ is in the boundary iff one of the coordinates is in the boundary of $M$.
Now if a point is in both $\nabla(\gamma)$ and $\partial(M^{k})$, then at least one of the coordinates (say $x_{i}$) is in the boundary, and thus all the points indexed by some $j$ in the same connected component as $i$ in $\gamma$ is also in the boundary.
We then obtain:
\begin{equation}
  \Delta_{(k)} \cap \partial(M^{k}) = \colim_{\gamma \in \Gamma} \colim_{\varnothing \subsetneq S \subset \pi_{0}(\gamma)} (\partial M)^{S} \times M^{\pi_0(\gamma) - S}.
\end{equation}
For a fixed $\gamma$, the inner colimit is precisely the image of $\partial(M^{\pi_{0}(\gamma)})$ under the diagonal embedding $M^{\pi_{0}(\gamma)} \hookrightarrow M^{k}$.
Combining this with the previous lemma, we then obtain:

\begin{proposition}
  A model for the square~\eqref{eq.square}, with $W = M^{k}$ and $X = \Delta_{(k)}$, is given by:
  \[ \begin{tikzcd}
      B^{\otimes k} \ar[r, two heads, "\alpha_{k}"] \ar[d, "\xi_{k}"] & B^{\otimes k} / K^{\otimes k} \ar[d] \\
      \lim_{\gamma \in \Gamma^{\mathrm{op}}} R^{\otimes \pi_{0}(\gamma)} \ar[r, "\beta_{k}"] & \lim_{\gamma \in \Gamma^{\mathrm{op}}} R^{\otimes \pi_{0}(\gamma)} / K^{\otimes \pi_{0}(\gamma)}
    \end{tikzcd}
    \qed \]
\end{proposition}

The map $\alpha_{k}$ is surjective, thus we have a canonical quasi-isomorphism $\ker \alpha_{k} = K^{\otimes k} \qiso \hoker \alpha_{k}$.
Therefore, by \cite[Proposition~3.1]{CordovaBulensLambrechtsStanley2015a}:
\begin{corollary}
  \label{cnfbnd.cor.cone-cohom}
  The cohomology of the cone
  \[ \cone \bigl( (\hoker \beta_{k})^{\vee}[-nk] \tkrightarrow{\bar{\xi}_{k}} (K^{\otimes k})^{\vee}[-nk] \bigr) \]
  of the map induced by $\xi_{k}$ on the kernels is isomorphic, as a graded vector space, to the cohomology of $\Conf_{k}(M) = M^{k} - \Delta_{(k)}$.
  \qed
\end{corollary}

Armed with this corollary, we can now define a cubical diagram $C_{\bullet}$ (see~\cite[Section~7]{LambrechtsStanley2008a}) whose total cofiber computes the cohomology of $\Conf_{k}(M)$ as a graded vector space.
Given $\gamma \in \Gamma$, the chain complex $C_{\gamma}$ is defined to be $(K^{\otimes \pi_{0}(\gamma)})^{\vee}$; note in particular that $C_{\varnothing} = (K^{\otimes k})^{\vee}$.
If $\gamma' \supset \gamma$, then the map $C_{\gamma' \supset \gamma} : C_{\gamma'} \to C_{\gamma}$ is induced by the dual of the multiplication of $K$.

Recall that the total cofiber\index{TotCof@$\TotCof$, total cofiber} of $C_{\bullet}$ is a representative of the homotopy colimit of $C$, given by the chain complex~\cite[Definition~7.2]{LambrechtsStanley2008a}:
\begin{equation}
  \TotCof C_{\bullet} \coloneqq \biggl( \bigoplus_{\gamma \in \Gamma} C_{\gamma} \cdot y_{\gamma}, \; D \biggl),
\end{equation}
where $y_{\gamma}$ is some variable of degree $- \# E_{\gamma}$, $\deg(x \cdot y_{\gamma}) = \deg(x) + \deg(y_{\gamma})$, and
\begin{equation}
  D(x \cdot y_{\gamma}) = \pm (dx) \cdot y_{\gamma} + \sum_{e \in E_{\gamma}} \pm (C_{\gamma' \supset \gamma} x) \cdot y_{\gamma - e}.
\end{equation}

\begin{proposition}
  The total cofiber of $C_{\bullet}$ computes the cohomology of $\Conf_{k}(M)$ as a graded vector space, up to suspension by $nk$.
\end{proposition}
\begin{proof}
  The proof is almost identical to the proof of~\cite[Theorem~9.2]{LambrechtsStanley2008a}.
  First define an auxiliary cubical diagram $C'_{\bullet}$ just like $C_{\bullet}$, except for $C_{\varnothing}$ which we set equal to $(\hoker \beta_{k})^{\vee}$.
  The maps $C'_{\gamma \supset \varnothing}$ are induced by the inclusions $K^{\otimes \pi_{0}(\gamma)} \to B^{\otimes \pi_{0}(\gamma)}$, which go through the definition of $\hoker \beta_{k}$.
  By Poincaré--Lefschetz duality, $(\hoker \beta_{k})^{\vee}$ is a model for $\Delta_{(k)} = \bigcup_{\gamma \in \Gamma} \nabla(\gamma)$, each $C_{\gamma}$ is a model for $\nabla(\gamma)$, and the maps between the $C'_{\gamma}$ are models for the inclusions by Lemma~\ref{cnfbnd.lem.square}.
  We thus obtain that $\TotCof C'_{\bullet}$ is acyclic by~\cite[Proposition~9.1]{LambrechtsStanley2008a} and the homotopy invariance of total cofibers.

  The morphism $C'_{\gamma} \to C_{\gamma}$ is given by the identity if $\gamma \neq \varnothing$, and it is given by $\bar{\xi}_{k}^{\vee}$ if $\gamma = \varnothing$.
  This yields a morphism of cubical diagrams $C'_{\bullet} \to C_{\bullet}$.
  Define $C''_{\bullet}$ to be the object-wise mapping cone of $C'_{\bullet} \to C_{\bullet}$, so that there is a short exact sequence
  \[ 0 \to C'_{\bullet} \to C_{\bullet} \to C''_{\bullet} \to 0. \]
  For $\gamma \neq \varnothing$, the map $C'_{\gamma} \to C_{\gamma}$ is the identity, hence $C''_{\gamma}$ is acyclic.
  It follows that $\TotCof C''_{\bullet}$ is quasi-isomorphic to the cone of $\bar{\xi}_{k}^{\vee} : C'_{\varnothing} \to C_{\varnothing}$, which computes the cohomology of $\Conf_{k}(M)$ by Corollary~\ref{cnfbnd.cor.cone-cohom}.

  There is a long exact sequence between the homologies of the total cofibers $C'_{\bullet}$, $C_{\bullet}$, and $C''_{\bullet}$: total cofibers commute with mapping cones up to homotopy, because both are types of homotopy colimits.
  The proposition then follows from the fact that $H_{*}(\TotCof C'_{\bullet}) = 0$.
\end{proof}

\begin{theorem}
  \label{cnfbnd.thm.ga-same-cohom}
  Let $M$ be a simply connected manifold with simply connected boundary and which admits a Poincaré--Lefschetz duality model.
  Let $A$ be the model of $M$ obtained from this PLD model (see Section~\ref{cnfbnd.sec.pretty-nice}).
  Then there is an isomorphism of graded vector spaces between $H^{*}(\GG{A}(k); \Q)$ and $H^{*}(\Conf_{k}(M); \Q)$.
\end{theorem}
\begin{proof}
  There is in fact an isomorphism of dg-modules
  \[ \GG{A}(k) \cong (\TotCof C_{\bullet})[-nk]. \]
  It is induced by the Poincaré--Lefschetz isomorphism $\theta : A \to K^{\vee}[-n]$.
  We use the crucial fact that the multiplication of $K$ is dual, under this isomorphism, to the map $A \to A^{\otimes 2}$ defined by $a \mapsto (a \otimes 1) \Delta_{A} = (1 \otimes a) \Delta_{A}$ (which is true by definition of $\Delta_{A}$ in our setting).
  The proof is then similar to the proof of Lemma~\ref{cnf.lemma.cmp-felix-thomas}.
\end{proof}

In other words, $\GG{A}(k)$ is a dg-module model of $\Conf_{k}(M)$.
Unfortunately, in general if $\partial M \neq \varnothing$ then $\GG{A}(k)$ is \emph{not} an actual model of $\Conf_{k}(M)$: the algebra structure is not the correct one.

\subsection{The perturbed model}
\label{cnfbnd.sec.perturbed-model}

We define in the next section a ``perturbed'' version of $\GG{A}(V)$, which will be the actual model for $\Conf_{k}(M)$, and we will prove that it is isomorphic to $\GG{A}(V)$ as a dg-module.

Choose some section $s : B_{\partial} \to B$ of $\rho$.
Using the Poincaré duality of $B_{\partial}$, this section $s$ corresponds to some element $\sigma_{B} \in B_{\partial} \otimes B$ of degree $n-1$.
Note that, by definition, the image of $\sigma_{B}$ in $B_{\partial}^{\otimes 2}$ is the diagonal class of $B_{\partial}$.

The map $s$ does not commute, in general, with the differential, though for any $x \in B_{\partial}$, one has $ds(x) - s(dx) \in K = \ker \rho$.
It follows that in general $d \sigma_{B} \neq 0$, but $d \sigma_{B} \in B_{\partial} \otimes K$.
We in fact have that $d \sigma_{B}$ is a representative of the image of $\Delta_{AK}$ under the map $B \otimes K \to B_{\partial} \otimes K$.

Let\index{sigmaA@$\sigma_{A}$}
\begin{equation}
  \sigma_{A} = (1 \otimes \pi)(\sigma_{B}) \coloneqq \sum_{(\sigma_{A})} \sigma' \otimes \sigma'' \in B_{\partial} \otimes A.
\end{equation}

\begin{remark}
  We are going to use Sweedler's notation as above extensively.
  If the element $\sigma$ (or any other tensor) appears multiple times in an equation, we are going to write it as follows:
  \[ \sigma \otimes \sigma = \sum \sigma'_{1} \otimes \sigma''_{1} \otimes \sigma'_{2} \otimes \sigma''_{2} \]
  and so on if it appears more than twice.
\end{remark}

We define the perturbed model to be:\index{GGAt@$\GGt{A}(V)$}
\begin{equation}
  \label{cnfbnd.eq.perturbed-model}
  \GGt{A}(V) \coloneq \bigl( A^{\otimes V} \otimes S(\tilde\omega_{vv'})_{v,v' \in V} / J, d \tilde\omega_{vv'} = (\iota_{v} \cdot \iota_{v'})(\Delta_{A}) \bigr)
\end{equation}
where the ideal of relations is generated by $\tilde\omega_{vv'}^{2} = 0$ and, for all $b \in B$ and all subsets $T \subset V$ of cardinality at least two:
\begin{equation}
  \sum_{v \in T} \pm \bigl( \iota_{v}(\pi(b)) \cdot \prod_{v \neq v' \in T} \tilde\omega_{vv'} \bigr) + \sum \pm \varepsilon_{\partial} \bigl( \rho(b) \prod_{v \in T} \sigma'_{v} \bigr) \prod_{v \in T} \iota_{v}(\sigma''_{v}).
\end{equation}

Note in particular that we have
\begin{equation}
  \tilde\omega_{12} - (-1)^{n} \tilde\omega_{21} + \sum \pm \varepsilon_{\partial}(\sigma'_{1} \sigma'_{2}) \sigma''_{1} \otimes \sigma''_{2} = 0 \in \GGt{A}(\underline{2}),
\end{equation}
with the last summand being a cycle.
The relations for $\# T = 2$ are perturbations of the symmetry relation of $\GG{A}(V)$, and the relations for $\# T = 3$ are perturbations of the Arnold relations in $\enV(V)$ (where, by ``perturbation'', we informally mean that the perturbed relation is the sum of the standard relation and terms which have a strictly lower number of generators $\tilde\omega_{vv'}$).

\begin{example}
  Let us consider $M = S^{1} \times [0,1]$ (even though it doesn't satisfy our assumptions about connectivity).
  We can find a PLD model for $M$ where:
  \begin{itemize}
  \item $B_{\partial} = H^{*}(\partial M)$ is four-dimensional, generated by $1$, $t$, $d\varphi$, and $t d\varphi$ with $t$ being idempotent;
  \item $A = H^{*}(M) = H^{*}(S^{1})$ is two-dimensional, generated by $1$ and $d\varphi$;
  \item $K = H^{*}(M, \partial M)$ is two-dimensional, generated by $dt$ and $dt \wedge d\varphi$.
  \end{itemize}

  Then one of the nontrivial relation in $\GGt{A}(2)$ is given by $(d\varphi \otimes 1) \tilde\omega_{21} + (1 \otimes d\varphi) \tilde\omega_{12} + (d\varphi \otimes d\varphi) = 0$.
  To obtain some intuition about this relation, consider that $\Conf_{2}(M) \simeq \Conf_{2}(\R^{2} - \{0\})$ is homotopy equivalent to $\Conf_{3}(\R^{2})$.
  The element $d\varphi$ correspond to the two points are rotating around the origin.
  Thus we can identify $\tilde\omega_{12}$ with $\omega_{12} \in H^{*}(\Conf_{3}(\R^{2}))$, $d\varphi \otimes 1$ with $\omega_{13}$, and $1 \otimes d\varphi$ with $\omega_{23}$.
  The perturbed relation in $\GGt{A}(2)$ is then nothing but the usual Arnold relation in $\enV[2](3) = H^{*}(\Conf_{3}(\R^{2}))$.
\end{example}

\begin{proposition}
  There is an isomorphism of dg-modules between $\GG{A}(V)$ and $\GGt{A}(V)$.
\end{proposition}

\begin{proof}
  Let us fix $V = \{ 1, \dots, k \}$ for some $k \geq 0$.
  Consider the standard basis of $\enV(k)$ given by monomials of the type:
  \[ \omega_{i_{1}j_{1}} \dots \omega_{i_{r}j_{r}}, \]
  with $1 \leq i_{1} < \dots < i_{r} \leq r$ and $i_{l} < j_{l}$ for all $l$.
  By choosing some basis $\{ a_{1}, \dots, a_{m} \}$ of $A$, we obtain a basis of $\GG{A}(k)$ by labeling the last element of each connected component of such a monomial by some $a_{i}$.

  We claim that if we replace all the $\omega_{ij}$ by $\tilde\omega_{ij}$ in this basis, then we obtain a basis of $\GGt{A}(V)$.
  Using the perturbed Arnold relations, it's clear that any element of $\GGt{A}(V)$ can be written as a linear combination of these elements.
  Moreover, using the same argument that proves that there is no nontrivial relation between the elements of the standard basis of $\enV(k)$, we can prove that there is non nontrivial relation between the elements of our claimed basis.

  There is thus a linear isomorphism $\GG{A}(V) \to \GGt{A}(V)$ which is defined on the basis by replacing all the $\omega_{ij}$ by $\tilde\omega_{ij}$.
  It's then clear that this map preserves the internal differential of $A$ and the part of the differential which splits an $\omega_{i_{l}j_{l}}$ (which can be written down explicitly in the basis: it merely splits a connected component into two).
\end{proof}

\begin{corollary}
  There is an isomorphism of graded vector spaces $H^{*}(\Conf_{k}(M)) \cong H^{*}(\GGt{A}(k))$.
\end{corollary}
\begin{proof}
  This follows immediately from Theorem~\ref{cnfbnd.thm.ga-same-cohom}
\end{proof}

\begin{remark}
  It is often the case that $\GG{A}$ and $\GGt{A}$ are actually equal.
  For example, if $M$ is obtained by removing a point from a closed manifold, then $B_{\partial} = H^{*}(S^{n-1}) = S(v)/(v^{2})$, and $\sigma_{A} = v \otimes 1$.
  Then in all the ``corrective terms'' in the definition of $\GGt{A}(V)$, the power of $v$ is at least two, hence the term vanishes.
\end{remark}

\begin{example}
  \label{cnfbnd.exa.model-dn-trois}
  If $M = D^{n}$ and we use the PLD model of Example~\ref{cnfbnd.exa.model-dn}, then $\GGt{A} = \GG{A}$ (see Example~\ref{cnfbnd.exa.model-dn-deux}).
\end{example}

\section{Computation of Swiss-Cheese graph cohomology}
\label{cnfbnd.sec.sgcn}

In this section, we prove statements related to the vanishing of the homology of a twist of $\SGC_{n}^{\vee}$ (see Section~\ref{cnfbnd.sec.graph-complexes}).
We assume that $n \geq 3$ throughout the section.

Elements of $\SGC_{n}$ are given by linear combinations of graphs; we identify an element in the dual with a (possibly infinite) sum of graphs using the dual basis.
Let $\delta$ be the differential in $\SGC_{n}^{\vee}$: it is given by ``splitting'' an aerial vertex (in all possible ways) in two vertices connected by an edge.
Then $\SGC_{n}^{\vee}$ is a (complete) dg-Lie algebra, with the bracket given by $[ \gamma, \gamma' ] = \gamma \circ \gamma'' - \pm \gamma' \circ \gamma$, where $\gamma \circ \gamma'$ is defined by the sum over the terrestrial vertices of $\gamma$ of the insertion of $\gamma'$ into that terrestrial vertex and reconnecting the incident edges in all possible ways.

Recall the Kontsevich integrals $c \in \SGC_{n}^{\vee}$ (see Equation~\eqref{cnfbnd.eq.konts-coeff}).
It is well-known (see e.g.~\cite[Section~6.1]{Willwacher2015a}) that $c$ starts with a linear part $c_{0}$ corresponding to the Hochschild--Kostant--Rosenberg map, which we now describe.
Let $\gamma_{j}$\index{gammaj@$\gamma_{j}$}, for $j \geq 0$, be the graph with one aerial vertex, $j$ terrestrial vertex, and an edge from the aerial vertex to each terrestrial vertex.

\begin{equation}
  \gamma_{j} \coloneqq
  \underbrace{
  \begin{gathered}
      \begin{tikzpicture}
        \node (0) {};
        \node (o) [right = 4.5cm of 0] {};
        \draw[dotted] (0) to (o);
        \node[intv] (i1) [right = 0.5cm of 0] {};
        \node[intv] (i2) [right = 0.5cm of i1] {};
        \node[intv] (i3) [right = 1.5cm of i2] {};
        \node[intv] (i4) [right = 0.5cm of i3] {};
        \node[intv] (j) [above right = 1cm of i2] {};
        \draw[->] (j) edge (i1) edge (i2) edge (i3) edge (i4);
      \end{tikzpicture}
    \end{gathered}
    }_{j \text{ vertices}}
\end{equation}

\begin{definition}
  \label{cnfbnd.def.c0}
  The linear part $c_{0} \in \SGC_{n}^{\vee}$\index{c0@$c_{0}$} of $c$ is given by $c_{0}(\gamma_{j}) = \frac{1}{j!}$ and $c_{0}(\gamma) = 0$ if $\gamma$ does not have exactly one aerial vertex.
\end{definition}

The element $c_{0}$ itself is a Maurer--Cartan element in $SGC_{n}^{\vee}$.
In other words, $c - c_{0}$, which vanishes on all graphs with exactly one aerial vertex, is a Maurer--Cartan in the twisted Lie algebra
\begin{equation}
  \SGC_{n}^{\vee, c_{0}} \coloneqq \bigl( \SGC_{n}^{\vee}, \delta + [c_{0}, -] \bigr).
\end{equation}
We'd like to prove that $c$ is gauge equivalent to $c_{0}$, or equivalently that $c-c_{0}$ is gauge trivial in the twisted Lie algebra.

\begin{proposition}
  \label{cnfbnd.prop.h-sgcn}
  For each $i \in \mathbb{Z}$, the dimension of $H^{i}(\SGC_{n}^{\vee, c_{0}})$ is at most equal to the dimension of $H^{i}(\GC_{n}^{\vee}) \oplus H^{i}(\GC_{n-1}^{\vee})$.
\end{proposition}

This is to be compared with a similar result for $n = 2$ at the beginning of~\cite[Section~5]{Willwacher2015a}, stating that $H^{*}(\SGC_{2}^{\vee, c_{0}}) \cong H^{*}(\GC_{2}^{\vee})$.
This result relies in turn on~\cite[Appendix~F]{Willwacher2016}, and we will use similar proof techniques.

\begin{proof}[Proof of Proposition~\ref{cnfbnd.prop.h-sgcn}]
  Call an aerial vertex a ``wedge''\index{wedge} if it is bivalent and connected to two terrestrial vertices.
  For example, the only aerial vertex of $\gamma_{2}$ is a wedge.
  We define a complete decreasing filtration on $\SGC_{n}^{\vee,c_{0}}$ by the number of aerial vertices that are \emph{not} wedges.
  On the $\EE^{0}$ page, the differential is given by the bracket $\gamma \mapsto [\gamma_{2}, -]$, which can be represented as follows:
  \[
    \begin{gathered}
      \begin{tikzpicture}
        \node (0) {};
        \node (o) [right = 3cm of 0] {};
        \draw[dotted] (0) to (o);
        \node[intv] (i) [right=1.5cm of 0] {};
        \draw[<-] (i) edge +(-1,1) edge +(-0.5,1) edge +(0.5,1) edge +(1,1);
      \end{tikzpicture}
    \end{gathered}
    \mapsto
    \begin{gathered}
      \begin{tikzpicture}
        \node (0) {};
        \node (o) [right = 3cm of 0] {};
        \draw[dotted] (0) to (o);
        \node[intv] (i1) [right=1cm of 0] {};
        \draw[<-] (i1) edge +(-1,1) edge +(-0.5,1);
        \node[intv] (i2) [right=1cm of i1] {};
        \draw[<-] (i2) edge +(1,1) edge +(0.5,1);
        \node[intv] (i) at ($(i1)+(.6,.7)$) {};
        \draw[->] (i) edge (i1) edge (i2);
      \end{tikzpicture}
    \end{gathered}
  \]
  This roughly corresponds to the differential in $\Graphs_{n-1}^{\vee}$.
  Note that the creations of ``dead ends'' in $\gamma \circ \gamma_{2}$ (i.e.\ when all the incoming edges at a vertex are put on one side of the wedge) are canceled with the summands coming from $\gamma_{2} \circ \gamma$.

  Our goal is to prove that on $\EE^{1} = H(\EE^{0}, d^{0})$, there only remains univalent terrestrial vertices (``hairs'') and no wedges.
  The computation is conceptually similar to the one right after~\cite[Claim~1]{Willwacher2016}, except that instead of computing the Hochschild homology of a symmetric algebra, we compute the $\en[n-1]$-cohomology of a symmetric algebra with trivial bracket.

  The $\EE^{0}$ page splits in a direct sum as follows.
  Given some $\gamma \in \EE^{0}$, we consider the ``index'' $\operatorname{idx}(\gamma)$ obtained by removing all wedges and cutting off the terrestrial vertices, keeping a half-edge.
  We obtain in this way a possibly disconnected graph with only aerial vertices, each one having a certain number of half-edges attached to it.
  The only restriction is that a zero-valent vertex may not have two half-edges attached to it (otherwise it would be a wedge).
  Then $\EE^{0}_{\Gamma} \coloneqq \{ \gamma \in \EE^{0} \mid \operatorname{idx}(\gamma) = \Gamma \}$ is a subcomplex, and that $\EE^{0} = \bigoplus_{\Gamma} \EE^{0}_{\Gamma}$.

  We'd like to compute $H^{*}(\EE^{0}_{\Gamma}, d^{0})$.
  There are two cases to consider.
  If $\Gamma = \varnothing$, then $(\EE^{0}_{\varnothing}, d^{0})$ is isomorphic to $\GC_{n-1}^{\vee}$ in the following way.
  If $\gamma \in \GC_{n-1}^{\vee}$, then we can view it as an element of $\EE^{0}_{\varnothing}$ by viewing all its vertices as terrestrial, and by replacing all its edges by wedges.
  The differential $d^{0}$ is then the differential of $\GC_{n-1}^{\vee}$.

  Now if $\Gamma \neq \varnothing$, then the actual shape of $\Gamma$ doesn't matter, only the number of half-edges does.
  If we number the half-edges as $h_{1}, \dots, h_{k}$, then we see that we are actually computing the $\Graphs_{n-1}^{\vee}$-cohomology of the symmetric algebra $S(h_{1}, \dots, h_{k})$ and taking the summand where each half-edge appears exactly once.
  Since $\Graphs_{n-1}^{\vee} \simeq \en[n-1]$, we have $H^{*}(\EE^{0}_{\Gamma}) = H^{*}_{\en[n-1]}(S(h_{1}, \dots, h_{k}); \R)^{(1, \dots, 1)}$.
  Applying the higher HKR theorem, this cohomology is given by a symmetric algebra on a shift of the generators.
  Going back to $\EE^{0}_{\Gamma}$, under this identification, $H^{*}(\EE^{0}_{\Gamma}, d^{0})$ is given by graphs where the code is $\Gamma$,there are no wedges, and each terrestrial vertex is univalent.

  In other words, we can identify $\EE^{1}$ as a direct sum of $H^{*}(\GC_{n-1}^{\vee})$ and a version of the hairy graph complex (as a vector space) where each vertex may have multiple hairs.
  The differential $d^{1}$ is given by the part of the differential of $\SGC_{n}^{\vee,c_{0}}$ which raises the number of non-wedge vertices by exactly one, i.e. $d^{1} = \delta + [\gamma_{0}, -] + [ \gamma_{1}, -] + \sum_{j \geq 3} [\gamma_{j}, -]$.

  We can now apply the same technique which appears in the proof of~\cite[Claim~2]{Willwacher2016}.
  The differential $d^{1}$ splits into three parts: $d^{1,-1} = [\gamma_{0},-]$ which decreases the number of terrestrial vertices by $1$, $d^{1,0} = \delta + [\gamma_{1}, -]$ which keeps it constant, and $d^{1,\geq 1} = \sum_{j \geq 3} [\gamma_{j}, -]$ which increases it.
  Let $\EE^{1'}$ be the quotient of $\EE^{1}$ by graphs with  terrestrial vertices as well as graphs in the image of $d^{1,-1}$ (i.e.\ graphs with univalent aerial vertices connected to another aerial vertex).
  The map $d^{1,0}$ induces a differential on $\EE^{1'}$.
  We can then reuse the proof of~\cite[Claim~2]{Willwacher2016} to prove that $(\EE^{1}, d^{1})$ is quasi-isomorphism to the direct sum of $H^{*}(\GC_{n-1}^{\vee})$ and the quotient $(\EE^{1'}, d^{1,0})$.

  Then $(\EE^{1'}, d^{1,0} = \delta)$ has the same cohomology as $\GC_{n}^{\vee}$, using an argument identical to the one in~\cite[Appendix~F.2.1]{Willwacher2016}.
  Thus, the $\EE^{2}$ page is a direct sum of $H^{*}(\GC_{n}^{\vee})$ and $H^{*}(\GC_{n-1}^{\vee})$.
  It follows that the dimension of the cohomology of $\SGC_{n}^{\vee,c_{0}}$ is at most the dimension of the cohomology of $\GC_{n}^{\vee} \oplus \GC_{n-1}^{\vee}$.
\end{proof}

Recall that $H^{*}(\GC_{n}^{\vee})$ and $H^{*}(\GC_{n-1}^{\vee})$ contain some special classes, the circular graphs (or loops, but this terminology clashes with our terminology for edges between a vertex and itself), see e.g.\ \cite[Proposition~3.4]{Willwacher2014}.
We depict these circular graphs in Figure~\ref{cnfbnd.fig.circular}.

\begin{figure}[htbp]
  \centering
  \raisebox{-.5\height}{\tikz{
    \node[intv] (i1) {};
    \draw (i1) edge [loop above] (i1);
  }}
  \;
  \raisebox{-.5\height}{\tikz{
    \node[intv] (i1) {};
    \node[intv, right={1 of i1}] (i2) {};
    \draw (i1) to[bend left] (i2);
    \draw (i2) to[bend left] (i1);
  }}
  \;
  \raisebox{-.5\height}{\tikz[scale=0.5]{
    \node[intv] (i1) at (0:1) {};
    \node[intv] (i2) at (120:1) {};
    \node[intv] (i3) at (240:1) {};
    \draw (i1) to (i2);
    \draw (i2) to (i3);
    \draw (i3) to (i1);
  }}
  \;
  \raisebox{-.5\height}{\tikz{
    \node (o) {};
    \draw[dotted] (o) -- ($(o)+(1.5,0)$);
    \node[intv, right={0.5 of o}] (i1) {};
    \node[intv, above={1 of i1}] (j1) {};
    \draw[->] (j1) edge[bend left] (i1) edge[bend right] (i1);
  }}
  \raisebox{-.5\height}{\tikz{
    \node (o) {};
    \draw[dotted] (o) -- ($(o)+(3,0)$);
    \node[intv, right={0.5 of o}] (i1) {};
    \node[intv, right={2 of i1}] (i2) {};
    \node[intv] (j1) at ($(i1)+(0.5,1)$) {};
    \node[intv] (j2) at ($(i2)+(-0.5,1)$) {};
    \draw[->] (j1) edge (i1) edge (i2);
    \draw[->] (j2) edge (i1) edge (i2);
  }}
  \raisebox{-.5\height}{\tikz[yscale=0.7]{
    \node (o) {};
    \draw[dotted] (o) -- ($(o)+(4,0)$);
    \node[intv, right={0.5 of o}] (i1) {};
    \node[intv, right={1 of i1}] (i2) {};
    \node[intv] (j1) at ($(i1)+(0.6,1)$) {};
    \draw[->] (j1) edge (i1) edge (i2);
    \node[intv, right={1 of i2}] (i3) {};
    \node[intv] (j2) at ($(i2)+(0.6,1)$) {};
    \draw[->] (j2) edge (i2) edge (i3);
    \node[intv, above={1 of i2}] (j3) {};c
    \draw[->] (j3) edge[bend right] (i1) edge[bend left] (i3);
  }}
  \caption{Circular graphs in $\GC_n$ and $\GC_{n-1}$}
  \label{cnfbnd.fig.circular}
\end{figure}

\begin{corollary}
  The Maurer--Cartan element $c$ is gauge equivalent to $c_{0}$ in $\SGC_{n}^{\vee}$.
\end{corollary}

\begin{proof}
  According to~\cite[Proposition~2.2.3]{FresseWillwacher2015}, the cohomology of $\GC_{n}^{\vee}$ is given by a sum of the circular graphs and a part which vanishes in degrees $* > -n$ (we work with cohomological conventions, so the degree is reversed).
  The integral defining $c$ vanishes on all the circular graphs for degree reasons: the total degree of a circular graph with $k$ ``edges'' (either an edge between two aerial vertices or a wedge between two terrestrial vertices) is $-k < 0$.
  It then follows from Proposition~\ref{cnfbnd.prop.h-sgcn} that $c - c_{0}$ lives in a Lie subalgebra of $\SGC_{n}^{\vee, c_{0}}$ whose cohomology vanishes in degrees $* > -(n - 1)$.
  Hence, by obstruction theory,\footnote{The $\EE^{1}$ page of the spectral sequence computing $\pi_{*} \operatorname{Map}_{\cdga}((S(\SGC_{n}[1]), \delta_{c_{0}}), \R) = \pi_{*} \mathsf{MC}_{\bullet}(\SGC_{n}^{\vee,c_{0}})$ vanishes in the right degrees.} the Maurer--Cartan element $c - c_{0}$ is gauge trivial.
\end{proof}

The definition of the cooperad $\SGraphs_{n}$ (cf.\ Section~\ref{cnfbnd.sec.extens-swiss-cheese}) depends on the choice of this Maurer--Cartan element, $c$ or $c_{0}$.

\begin{definition}
  \label{cnfbnd.def.sgraphs0}
  The cooperad $\SGraphs_{n}^{c_{0}}$ is defined similarly to $\SGraphs_{n}$, but the Maurer--Cartan element $c_{0}$ is used in place of $c$ in the definition of the differential.
\end{definition}

By choosing a path object for $\SGC_{n}^{\vee}$, we can turn the gauge equivalence between $c$ and $c_{0}$ into:
\begin{corollary}
  \label{cnfbnd.cor.sgraphs-c0}
  There is a zigzag of quasi-isomorphism of cooperads:
  \begin{equation*}
    \SGraphs_{n} \qiso* \cdot \qiso \SGraphs_{n}^{c_{0}}.
    \qed
  \end{equation*}
\end{corollary}

Let us also record the following observation.
The comodule $\SGraphs_{n}^{c_{0}}(\varnothing, -)$ is not isomorphic to $\Graphs_{n}$ seen as a comodule over itself, because its graphs can contain internal terrestrial vertices.
However, $\SFM_{n}(\varnothing, \{*\})$ is a point, and there is a weak equivalence of $\FM_{n}$-modules given by:
\begin{equation}
  \FM_{n}(U) \cong \SFM_{n}(\varnothing, \{*\}) \times \FM_{n}(U) \tkrightarrow{\circ_{U}} \SFM_{n}(\varnothing, U).
\end{equation}

This is modeled by the following proposition.
Define the map
\begin{equation}
  \nu : \SGraphs_{n}^{c_{0}}(\varnothing, V) \to \Graphs_{n}(V)
\end{equation}
as follows.
Given some $\Gamma \in \SGraphs_{n}^{c_{0}}(\varnothing, V)$, if $\Gamma$ only has univalent terrestrial vertices, then $\nu(\Gamma)$ is the graph with these univalent vertices and their incident edges removed.
Otherwise, if $\Gamma$ has terrestrial vertices of valence greater than one, then $\nu(\Gamma) = 0$.

\begin{proposition}
  \label{cnfbnd.prop.qiso-sc}
  The maps $\nu : \SGraphs_{n}^{c_{0}}(\varnothing, V) \to \Graphs_{n}(V)$ defines a quasi-isomorphism of Hopf right $\Graphs_{n}$-modules.
\end{proposition}
\begin{proof}
  For a given $V$, the map is the composite of two quasi-isomorphisms of CDGAs (hence it is itself a CDGA map and a quasi-isomorphism):
  \begin{itemize}
  \item The comodule structure map
    \[ \circ_{V}^{\vee} : \SGraphs_{n}^{c_{0}}(\varnothing, V) \to \SGraphs_{n}^{c_{0}}(\varnothing, \{*\}) \otimes \Graphs_{n}(V), \]
    which is a quasi-isomorphism because it models the weak equivalence $\FM_{n}(\varnothing, \{*\}) \times \FM_{n}(V) \to \SFM_{n}(\varnothing, V)$;
  \item The map $\SGraphs_{n}^{c_{0}}(\varnothing, \{*\}) \to \R$ given by $c_{0}$, tensored with the identity of $\Graphs_{n}(V)$.
    This is a quasi-isomorphism because $\SGraphs_{n}^{c_{0}}(\varnothing, \{*\})$ is a model for $\SFM_{n}(\varnothing, \{*\})$, which is a point, and $c_{0}$ is a nontrivial cocycle.
  \end{itemize}

  The fact that $\SGraphs_{n}^{c_{0}}$ is a relative operad over $\Graphs_{n}$ also shows that this is a morphism of $\Graphs_{n}$-comodules.
\end{proof}

\section{A model of right Hopf comodules}
\label{sec.model-right-hopf}

\subsection{The propagator}
\label{cnfbnd.sec.propagator}

The proof of Theorem~\ref{cnf.thm.A} in the case of closed manifolds relied on the existence of a propagator, see Section~\ref{cnf.sec.propagator}.
Let us briefly recall its construction in this case, before going back to the case $\partial M \neq \varnothing$.
The propagator $\varphi$ is an $(n-1)$-form on $\FM_{M}(\underline{2})$ whose differential was the pullback of the diagonal class.
Moreover, recall that the projection $\partial \FM_{M}(\underline{2}) \to M$ is an $S^{n-1}$-bundle, and the restriction of the propagator to the boundary is a global angular form for this bundle.
If $M$ is framed then this bundle is trivial (with $\circ_{1} : M \times S^{n-1} \to \FM_{M}(\underline{2})$ being an isomorphism of bundles), and one can further assume that $\varphi$ is equal to $1 \times \vol_{n-1}$.

This propagator was constructed by considering a global angular form of the previous bundle, pulling it back to a tubular neighborhood of the boundary inside $\FM_{n}(\underline{2})$, multiplying it by a bump function, and then extending it by zero outside of the tubular neighborhood (see~\cite[Proposition~7]{CamposWillwacher2016}).
Moreover, it can be chosen so that it belongs to the subalgebra $\Omega^{*}_{\mathrm{triv}}(\FM_{M}(\underline{2})) \subset \OmPA^{*}(\FM_{M}(\underline{2}))$ of ``trivial'' PA forms (see~\cite[Appendix~C]{CamposWillwacher2016}), which implies that it can be integrated along the fibers of the canonical projections $p_{V}$.

Let us now return to the case $\partial M \neq \varnothing$.
Consider the double $D = M \cup_{\partial M} \bar{M}$, a closed manifold.
One can construct a propagator $\psi \in \Omega_{\mathrm{triv}}(\FM_{D}(\underline{2}))$.
There is an ``inclusion'' map $i : \SFM_{M}(\varnothing, \underline{2}) \to \FM_{D}(\underline{2})$ (which is not actually injective on the set of configurations $x \in \SFM_{M}(\underline{2})$ where $p_{1}(x) = p_{2}(x) \in \partial M$, just like the standard map $\SFM_{n}(\varnothing, \underline{2}) \to \FM_{n}(\underline{2})$ is not injective).
Moreover, there is a ``mirror'' map $\tau_{1} : \SFM_{M}(\varnothing, \underline{2}) \to \FM_{D}(\underline{2})$ which is defined similarly, except that the first point of the configuration is sent to its mirror in $\bar{M} \subset D$.
We then define the propagator on $M$ to be:
\begin{equation}
  \label{cnfbnd.eq.propagator}
  \varphi \coloneqq \frac{1}{2} \bigl( i^{*}\psi - \tau_{1}^{*}\psi \bigr).
\end{equation}
One can write down the cohomology of $H^{*}(D)$ in terms of the cohomology of $M$ and $\partial M$ and check on a basis that $[d \varphi] \in H^{*}(M \times M)$ is indeed the diagonal class of $M$.
Note that this propagator is not symmetrical: if the point $1$ (the ``origin'' of the propagator) becomes infinitesimally close to $\partial M$, then the two contributions cancel out and $\varphi$ become zero; this is not the case for the point $2$ (the ``target'').

Consider the space:
\begin{equation}
  E = \{ x \in \SFM_{M}(\varnothing,\underline{2}) \mid p_{1}(x) = p_{2}(x) \}
  \label{cnfbnd.eq.bdl-propagator}
\end{equation}
which is an $S^{n-1}$-bundle $\mathring{E}$ when restricted over the interior of $M$.
Given that $\psi$ is a global angular form on $\partial \FM_{D}(\underline{2})$, we check that $\varphi$ is a global angular form on this $S^{n-1}$-bundle.

\begin{remark}
  \label{cnfbnd.rmk.triv-forms}
  We go back to the construction of the diagram of Equation~\eqref{cnfbnd.eq.nice-diagram} and replace $\OmPA^{*}(-)$ with its quasi-isomorphic subalgebra $\Omega^{*}_{\mathrm{triv}}(-)$, and then build a PLD model from that.
  We can then compose with the inclusion $\Omega^{*}_{\mathrm{triv}}(-) \to \OmPA^{*}(-)$ and we obtain a diagram:
  \[ \begin{tikzcd}
      R \ar[d, two heads, "\rho"] \ar[r, "\sim", "f" swap] & \Omega^{*}_{\mathrm{triv}}(M) \ar[r, hook, "\sim"] \ar[d, "\text{res}"] & \OmPA^{*}(M) \ar[d, "\text{res}"] \\
      R_{\partial} \ar[r, "\sim", "f_{\partial}" swap] & \Omega^{*}_{\mathrm{triv}}(M) \ar[r, hook, "\sim"] & \OmPA^{*}(M)
    \end{tikzcd} \]
  This allows to assume that if $x \in R$ (resp.\ $y \in R_{\partial}$), then $g(x)$ (resp.\ $g_{\partial}(y)$) is a trivial form on $M$ (resp.\ $\partial M$) -- compare with Remark~\ref{cnf.rmk.triv-forms}.
\end{remark}

Let us fix the differential of the resulting propagator.
Recall the diagonal class $\Delta_{AK} \in A \otimes K$ from Equation~\eqref{cnfbnd.def.delta-ak}, which gets sent to $\Delta_{A} \in A \otimes A$ under $K \subset B \xrightarrow{\pi} A$.
Then we have:

\begin{lemma}
  \label{cnfbnd.lem.delta-r}
  There exists an element $\Delta_{R} \in R \otimes R$ such that $(\pi \otimes \id) g^{\otimes 2}(\Delta_{R}) = \Delta_{AK}$ and $\mu_{R}(\Delta_{R}) = 0$.
\end{lemma}
\begin{proof}
  The proof is identical to the proof of Proposition~\ref{cnf.lemma.zigzag-a-r-omega}, recalling that $R \to B$ and $B \to A$ are both surjective, and that $\mu_{A}(\Delta_{A}) = 0$ when $\partial M \neq 0$ (Equation~\eqref{cnfbnd.eq.mu-delta-zero}).
\end{proof}

\begin{definition}
  \label{cnfbnd.def.delta-b}
  For convenience, define a cocycle $\Delta_{B}$\index{DeltaB@$\Delta_{B}$} in $B^{\otimes 2}$ by:
  \begin{equation*}
    \Delta_{B} \coloneqq g^{\otimes2}(\Delta_{R}).
  \end{equation*}
\end{definition}

Finally, we'd like to fix the value of the propagator on the subspace $\SFM_{M}(\{2\}, \{1\})$ of $\SFM_{M}(\varnothing, \{1,2\})$.
In order to define the inclusion
\begin{equation}
  \label{cnfbnd.eq.def-j-incl}
  j : \SFM_{M}(\{2\}, \{1\}) \hookrightarrow \SFM_{M}(\varnothing, \{1,2\}),
\end{equation}
we choose some inward pointing vector field on $\partial M$ and we use it to infinitesimally push the second point into the interior of $M$.

Recall the element $\sigma_{B} \in B_{\partial} \otimes B$ defined in Section~\ref{cnfbnd.sec.perturbed-model}, with $d \sigma_{B} \in B_{\partial} \otimes K$ being cohomologous to the diagonal class of $M$ with the first factor restricted to $\partial M$.
Using the fact that we have a surjective quasi-isomorphism (thanks to the five lemma):
\begin{equation}
  \cone(R_{\partial} \otimes \ker \rho) \to \cone(B \otimes K),
\end{equation}
we can find some element $\sigma_{R} \in R_{\partial} \otimes R$ with $(g_{\partial} \otimes g)(\sigma_{R}) = \sigma_{B}$ and $d \sigma_{R} \in R_{\partial} \otimes \ker \rho$.
Then $j^{*} \varphi - (f_{\partial} \otimes f)(\sigma_{R})$ is an exact form, by checking on cohomology.

\begin{proposition}
  \label{cnfbnd.prop.propagator}
  There exists a form $\varphi \in \OmPA^{n-1}(\SFM_{M}(\varnothing,\underline{2}))$\index{propagator@propagator $\varphi$} such that:
  \begin{itemize}
  \item $d\varphi$ is the pullback of $(f \otimes f)(\Delta_{R})$ along the product of the projections $\SFM_{M}(\underline{2}) \to M \times M$;
  \item its restriction to the bundle $\mathring{E}$ from Equation~\eqref{cnfbnd.eq.bdl-propagator} is a global angular form;
  \item if moreover $M$ is framed we choose $\varphi$ so that $\varphi|_{\mathring{E}} = 1 \times \vol_{n-1}$;
  \item $j^{*} \varphi = \varphi|_{\SFM_{M}(\{2\},\{1\})} = (f_{\partial} \otimes f)(\sigma_{R})$;
  \item for any $x \in R$, one has $(p_{2})_{*}(p_{1}^{*}(x) \varphi) = 0$.
  \end{itemize}
\end{proposition}
\begin{proof}
  We've already seen how to construct a propagator satisfying the first three properties, using the mirror $D = M \cup_{\partial M} \bar{M}$.

  The difference $j^{*} \varphi - (f_{\partial} \otimes f)(\sigma_{R})$ is an exact form, say $d \xi$ where $\xi$ is a trivial form on $\SFM_{M}(\{1\}, \{2\})$.
  The application $j^{*}$ is surjective (because $\SFM_{M}(\{1\}, \{2\})$ is a submanifold with corners of $\SFM_{M}(\varnothing, \{1,2\})$), hence there exists some $\hat{\xi}$ such that $j^{*} \hat{\xi} = \xi$.
  It then suffices to replace $\varphi$ by $\varphi - d \hat{\xi}$, which still satisfies the first three properties.

  Finally, for the last property, we reuse the proof technique of~\cite[Lemma~3]{CattaneoMneev2010}.
  Namely, we consider the chain map:
  \begin{align*}
    r : \Omega^{*}_{\mathrm{triv}}(M) & \to R \\
    x & \mapsto \sum_{(\Delta_{R})} \bigl( \int_{M} x \Delta' \bigr) \Delta'' + \sum_{(\sigma_{R})} (\int_{\partial M} x|_{\partial M} \sigma') \sigma'',
  \end{align*}
  satisfying $f = f \circ r \circ f$, and the homotopy:
  \begin{align*}
    h : \Omega^{*}_{\mathrm{triv}}(M) & \to \Omega^{*-1}_{\mathrm{triv}}(M) \\
    x & \mapsto (p_{1})_{*}(p^{*}_{2}(x) \wedge \varphi).
  \end{align*}

  Then we replace $h$ by $h' \coloneqq (\id - fr) \circ h \circ (\id - fr)$ in order to obtain a new homotopy satisfying $h''''' \circ f = 0$.
  Let us write down the integral kernel of $h'$.
  In order to simplify notations, we write $\int_{i,j,\dots} (-)$ to indicate that we push forward along the projection which forgets the points $i, j, \dots$, we write $\Delta_{ij} = p_{ij}^{*}(\Delta_{R})$ and $\sigma_{ij} = p_{ij}^{*}(\sigma_{R})$.
  Then the new propagator is:
  \begin{align*}
    \tilde\varphi_{12}
    & \coloneq \varphi_{12} - \int_{3} \varphi_{23} \Delta_{13} - \int_{3} \varphi_{23} \sigma_{13} - \int_{3} \varphi_{13} \Delta_{23} - \int_{3} \varphi_{13} \sigma_{23} \\
    & + \int_{3,4} \Delta_{24} \varphi_{34} \Delta_{13} + \int_{3,4} \sigma_{24} \varphi_{23} \Delta_{13} + \int_{3,4} \Delta_{24} \varphi_{34} \sigma_{13} + \int_{3,4} \sigma_{42} \varphi_{34} \sigma_{13}.
      \qedhere
  \end{align*}

  By construction, this new propagator $\tilde\varphi$ satisfies the last property.
  To check that the corrective term $\tilde\varphi - \varphi$ is a cycle, one needs to apply the Stokes formula and check that all the terms cancel out, which follows from the general properties of the diagonal class, the fact that $\varphi|_{\mathring{E}}$ is a global angular form, and that $j^{*}\varphi = \sigma$.
  Similarly to check that $j^{*}(\tilde\varphi - \varphi) = 0$ and that $(\tilde\varphi - \varphi)|_{\mathring{E}}$, one need to use the functoriality property of integral along fibers~\cite[Proposition~8.9]{HardtLambrechtsTurchinVolic2011} and check that all the terms cancel out.
\end{proof}

\subsection{Colored labeled graphs}
\label{cnfbnd.sec.colored-labeled-graphs}

We now introduce a colored version of the graph comodule $\Graphs_{R}$ considered in Section~\ref{cnf.sec.label-graphs-stat}.
Recall the cocycle $\Delta_{R}$ from Lemma~\ref{cnfbnd.lem.delta-r}.
We also define $\Delta_{R,R_{\partial}} = (\id \otimes \rho)(\Delta_{R}) \in R \otimes R_{\partial}$\index{DeltaRpartial@$\Delta_{R,R_{\partial}}$} and $\Delta_{R, \Omega^{*}(\partial M)} \coloneqq (\id \otimes g_{\partial})(\Delta_{R,R_{\partial}})$ for notational convenience.

\begin{definition}
  The CDGA of $(R, \Omega_{\mathrm{triv}}^{*}(\partial M))$-labeled graphs on the finite sets $U$, $V$ is:
  \[ \SGra_{R, \Omega^{*}(\partial M)}(U,V) \coloneqq \bigl( (\Omega_{\mathrm{triv}}^{*}(\partial M))^{\otimes U} \otimes R^{\otimes V} \otimes \SGra_{n}(U,V), d \bigr) \]
  with differential defined by $de_{vv''''} = \iota_{vv'}(\Delta_{R})$ for $v, v' \in V$, and $de_{vu} = \iota_{vu}(\Delta_{R,\Omega^{*}(\partial M)})$ for $u \in U$ and $v \in V$.
\end{definition}

\begin{proposition}
  The bisymmetric collection $\SGra_{R, \Omega^{*}(\partial M)}$ is a Hopf right comodule over $\SGra_{n}$.
\end{proposition}

\begin{proof}
  The proof is identical to the proof of Proposition~\ref{cnf.prop.gra-tad-comod}.
  Recall that we do not need additional assumptions on the Euler characteristic as soon as we assume $\partial M \neq \varnothing$ (see Lemma~\ref{cnfbnd.lem.delta-r}).
\end{proof}

This comodule has a graphical description similar to $\SGra_{n}$.
The difference is that aerial points (corresponding to the interior of $M$) are labeled by $R$, while terrestrial points (corresponding to the boundary) are labeled by $R_{\partial}$.

Given a graph $\Gamma$, the differential $d \Gamma$ is a sum over the set of edges of $\Gamma$.
For each summand, one removes the edge from the graph and multiplies the endpoints of the edge by either $\Delta_{R}$ (if both endpoints are aerial) or $\Delta_{R,\partial}$ (if one endpoint is aerial and the other terrestrial).
We call this ``splitting'' the edge, see Figure~\ref{cnfbnd.fig.d-split}.
Recall that we write $\Delta_{R} = \sum_{(\Delta_{R})} \Delta_{R}' \otimes \Delta_{R}'' \in R^{\otimes 2}$.

\begin{figure}[htbp]
  \centering
  \[ \begin{gathered}
      \begin{tikzpicture}
        \node[unkv, label = {\scriptsize $x$}] (1) {};
        \node[unkv, label = {\scriptsize $y$}] (2) [right = 1cm of 1] {};
        \node (x1) [left = 0.8cm of 1] {};
        \node (y1) [above left = 1cm of 1] {};
        \node (z1) [below left = 1cm of 1] {};
        \node (x2) [right = 0.8cm of 2] {};
        \node (y2) [above right = 1cm of 2] {};
        \node (z2) [below right = 1cm of 2] {};
        \draw (1) -- (2);
        \draw (x1) -- (1); \draw (y1) -- (1); \draw(z1) -- (1);
        \draw (x2) -- (2); \draw (y2) -- (2); \draw(z2) -- (2);
      \end{tikzpicture}
    \end{gathered}
  \mapsto
  \sum_{(\Delta_R)}
  \begin{gathered}
    \begin{tikzpicture}
        \node[unkv, label = {\scriptsize $x \Delta_{R}'$}] (1) {};
        \node[unkv, label = {\scriptsize $y \Delta_{R}''$}] (2) [right = 1cm of 1] {};
        \node (x1) [left = 0.8cm of 1] {};
        \node (y1) [above left = 1cm of 1] {};
        \node (z1) [below left = 1cm of 1] {};
        \node (x2) [right = 0.8cm of 2] {};
        \node (y2) [above right = 1cm of 2] {};
        \node (z2) [below right = 1cm of 2] {};
        \draw (x1) -- (1); \draw (y1) -- (1); \draw(z1) -- (1);
        \draw (x2) -- (2); \draw (y2) -- (2); \draw(z2) -- (2);
      \end{tikzpicture}
    \end{gathered}
  \]
  \caption{The splitting differential of $\SGra_{R,\Omega^{*}(\partial M)}$ (a gray vertex can be of any kind).}
  \label{cnfbnd.fig.d-split}
\end{figure}

The product glues graphs along their vertices (multiplying the labels), and the comodule structure collapses subgraphs, multiplying the labels and applying $\rho$ to them if necessary.

\begin{proposition}
  \label{cnfbnd.prop.omega-prime-colored}
  There is a morphism of bisymmetric collections $\omega' : \SGra_{R, \Omega^{*}(\partial M)} \to \OmPA^{*}(\SFM_{M})$\index{womegaprime@$\omega'(\Gamma)$} characterized by
  \begin{align*}
    \omega'(\iota_{v}(x))
    & = p_{v}^{*}(g(x))
    & v \in V, \; x \in R \\
    \omega'(\iota_{u}(x))
    & = p_{u}^{*}(x)
    & u \in U, \; x \in \Omega_{\mathrm{triv}}^{*}(\partial M) \\
    \omega'(e_{vv'})
    & = p_{vv'}^{*}(\varphi)
    & v, v' \in V \\
    \omega'(e_{vu})
    & = p_{vu}^{*}(j^{*}(\varphi))
    & u \in U, \; v \in V
  \end{align*}
  where $j : \SFM_{M}(\underline{1}, \underline{1}) \to \SFM_{M}(\varnothing,\underline{2})$ is the inclusion.\footnote{Recall that one must choose a collar $\partial M \times [0,1) \subset M$ in order to define this inclusion.}
  Moreover if $M$ is framed, then together with Willwacher's morphism~\cite{Willwacher2015a}, this defines a Hopf right comodule morphism
  \[ \omega' : (\SGra_{R, \Omega^{*}(\partial M)}, \SGra_{n}) \to (\OmPA^{*}(\SFM_{M}), \OmPA^{*}(\SFM_{n})). \]
\end{proposition}
\begin{proof}
  The proof is identical to the proof of Proposition~\ref{cnf.prop.zigzag-gra-r}.
\end{proof}

Let us define a Maurer--Cartan element $c_{\varphi} \in \SGC_{n}^{\vee} \otimes \Omega_{\mathrm{triv}}^{*}(\partial M)$.
Let $\gamma \in \SGC_{n}$ be a connected graph, let $I$ be the set of its terrestrial vertices and $J$ of its aerial vertices.
Then $\gamma$ defines a form $\omega'(\gamma)$ on $\SFM_{M}(I,J)$, which we can pull back along the composition map $\circ_{I,J} : \partial M \times \SFM_{n}(I,J) \to \SFM_{M}(I,J)$ and then integrate on the $\SFM_{n}(I,J)$ factor, in other words:
\begin{equation}
  c_{\varphi}(\gamma) \coloneqq (p_{\partial M})_{*}(\circ_{I,J}^{*}(\omega'(\gamma))).
\end{equation}

\begin{remark}
  For all $\gamma$, $\omega'(\gamma)$ is a trivial form (in the sense of~\cite[Appendix~C]{CamposWillwacher2016}), hence $c_{\varphi}(\gamma)$ is obtained by pushing forward a trivial form along a trivial bundle and is therefore trivial itself.
\end{remark}

\begin{definition}
  The \textbf{twisted colored labeled graph comodule} $\Tw \SGra_{R, \Omega^{*}(\partial M)}$\index{TwSGraR@$\Tw \SGra_{R, \Omega^{*}(\partial M)}$} is the Hopf right $(\Tw \SGra_{n})$-comodule obtained by twisting $\SGra_{R, \Omega^{*}(\partial M)}$ with respect to the Maurer--Cartan element $c_{\varphi}$.
\end{definition}

Let us give a graphical description of the module $\Tw \SGra_{R, \Omega^{*}(\partial M)}(U,V)$.
It is spanned by graphs with four types of vertices: on the one hand, either aerial or terrestrial, and on the other hand either external or internal.
Recall that aerial vertices are labeled by $R$ while terrestrial ones are labeled by $R_{\partial}$.
External aerial vertices are in bijection with $V$, while external terrestrial vertices are in bijection with $U$.
Internal aerial vertices are of degree $-n$, while internal terrestrial vertices are of degree $-(n-1)$.
Both kinds of internal vertices are indistinguishable among themselves.
Finally, edges are of degree $n-1$, and the source of an edge may only be aerial.

\begin{figure}[htbp]
  \centering
    \begin{tikzpicture}[baseline = 0.5cm]
    \node (0) {};
    \node (o) [right = 3cm of 0] {};
    \draw[dotted] (0) to (o);
    \node[extv, label = {above left}:{\scriptsize $x_{1}$}] (u) [right = 0.5cm of 0] {$u$};
    \node[intv, label = {\scriptsize $x_{2}$}] (i1) [right = 1cm of u] {};
    \node[extv, label = {\scriptsize $y_{1}$}] (v1) [above = 1cm of u] {$v_{1}$};
    \node[extv, label = {\scriptsize $y_{2}$}] (v2) [right = 1cm of v1] {$v_{2}$};
    \node[intv, label = {\scriptsize $y_{3}$}] (i2) [right = 1cm of v2] {};
    \draw[->] (v1) edge (u) edge (i1) edge (v2);
    \draw[->] (v1) to (v2);
    \draw[->] (i2) edge (v2) edge (i1);
  \end{tikzpicture}
  \quad ($x_{1}, x_{2} \in R_{\partial}$ and $y_{1}, y_{2}, y_{3} \in R$)
  \caption{A colored labeled graph in $\Tw \SGra_{R, \Omega^{*}(\partial M)}(\{u\}, \{v_1, v_2\})$.}
  \label{cnfbnd.fig.exa-col-lab-graph}
\end{figure}

The product glues graphs along external vertices (multiplying the labels).
The comodule structure maps collapse subgraphs, and the label of the collapsed subgraph is the product of all the labels inside that subgraph (applying $\rho : R \to R_{\partial}$ as needed).
Finally, the differential has several parts:
\begin{itemize}
\item A first part comes from $\SGra_{R, \Omega^{*}(\partial M)}$: it splits edges between vertices of any type, and then multiplies the endpoints of the removed edge by either $\Delta_{R}$ or $\Delta_{R,\partial}$.
\item A second part contracts edges between two aerial vertices, one of them being internal, and multiplies the labels of the endpoints in the process.
  This includes dead ends, i.e.\ edges connected to a univalent internal vertex.
\item A third part contracts a subgraph $\Gamma'$ with at most one external vertex.
  The label of the contracted subgraph is the product of all the labels inside it multiplied by $c_{\varphi}(\bar{\Gamma}')$, where $\bar{\Gamma}'$ is the subgraph with the labels removed.
  If the result contains a ``bad edge'' (whose source is terrestrial), then the summand vanishes.
\end{itemize}

\begin{remark}
  Note that there are several differences compared to the description of the differential from Section~\ref{cnfbnd.sec.extens-swiss-cheese}: dead ends are contractible, and the last part of the differential which forgets some internal vertices is not present.
  This comes from the fact that in the definition of the twisting of a right comodule over a cooperad, the Maurer--Cartan element of the deformation complex can only act ``from the right'' on the comodule.

  Dead ends aren't contractible in $\Tw \SGra_{n}$  because the contraction of a dead end appears twice in the differential, one ``from the left'' and one ``from the right'', and they cancel each other (see~\cite[Appendix~I.3]{Willwacher2014}).
  This cancellation does not occur in $\Tw \SGra_{R, \Omega^{*}(\partial M)}$.
  The last part of the differential on $\Tw \SGra_{n}$ came exclusively from the left action of the deformation complex on $\SGra_{n}$, and so cannot appear in $\Tw \SGra_{R, \Omega^{*}(\partial M)}$.
\end{remark}

\begin{remark}
  \label{cnfbnd.rmk.stokes}
  One should not forget that when a subgraph $\Gamma' \subset \Gamma$ with only internal vertices is contracted (the third part), the result may be a terrestrial vertex even though the subgraph contains only aerial vertices, see Figure~\ref{cnfbnd.fig.collapse-aerial-terrestrial}.
\end{remark}

\begin{figure}[htbp]
  \centering
  \begin{tikzpicture}[baseline = 0.5cm]
    \node (0) {};
    \node (o) [right = 1.5cm of 0] {};
    \draw[dotted] (0) to (o);
    \node (u) [right = 0.5cm of 0] {};
    \node[intv, label = {\scriptsize $x$}] (i1) [above = 0.5cm of u] {};
  \end{tikzpicture}
  $\mapsto$
  \begin{tikzpicture}[baseline = 0.5cm]
    \node (0) {};
    \node (o) [right = 1.5cm of 0] {};
    \draw[dotted] (0) to (o);
    \node[intv, label = {\scriptsize $\rho(x)$}] (u) [right = 0.5cm of 0] {};
  \end{tikzpicture}
  \caption{Collapsing an aerial vertex into a terrestrial vertex}
  \label{cnfbnd.fig.collapse-aerial-terrestrial}
\end{figure}

\begin{proposition}
  The morphism $\omega' : \SGra_{R, \Omega^{*}(\partial M)} \to \OmPA^{*}(\SFM_{M})$ of Proposition~\ref{cnfbnd.prop.omega-prime-colored} extends to a morphism of bisymmetric collections $\omega : \Tw \SGra_{R, \Omega^{*}(\partial M)} \to \OmPA^{*}(\SFM_{M})$\index{womega@$\omega(\Gamma)$}, given on a graph $\Gamma \in \SGra_{R, \Omega^{*}(\partial M)}(U \sqcup I, V \sqcup J) \subset \Tw \SGra_{R, \Omega^{*}(\partial M)}(U,V)$ by:
  \[ \omega(\Gamma) \coloneqq (p_{U,V})_{*}(\omega'(\Gamma)) = \int_{\mathrlap{\SFM_{M}(U \sqcup I, V \sqcup J) \to \SFM_{M}(U,V)}} \quad \omega'(\Gamma). \]
  Moreover if $M$ is framed then, together with Willwacher's morphism~\cite{Willwacher2015a}, this assignment defines a morphism of Hopf right comodules
  \[ \omega : (\Tw \SGra_{R, \Omega^{*}(\partial M)}, \Tw \SGra_{n}) \to (\OmPA^{*}(\SFM_{M}), \OmPA^{*}(\SFM_{n})). \]
\end{proposition}
\begin{proof}
    First note that we have chosen the propagator $\varphi$ so that it is a trivial form, and we have assumed that the morphisms $R \to \OmPA^{*}(M)$ and $R_{\partial} \to \OmPA^{*}(\partial M)$ factor through the sub-CDGAs of trivial forms.
  Hence for any graph $\Gamma$, $\omega'(\Gamma)$ is a trivial form and can be integrated along the fiber of $p_{U,V}$.

  We can now reuse the proof of Proposition~\ref{cnf.prop.mph-tw-om}.
  The difference is the description of the decomposition of the fiberwise boundary of $p_{U,V}$ used to show that $\omega$ is a chain map through the application of Stokes' formula.
  This description is very similar to the one implicitly used by~\cite{Willwacher2015a} (see also~\cite[Section~5.2.1]{Kontsevich2003}) with some variations accounting from the fact that no normalization is done to compactify $M$ (see the discussion before the proof of Lemma~\ref{cnf.lemma.int-partial-omega}).
  More concretely, the boundary of $\SFM_{M}(U,V)$ is given by:
  \[ \partial \SFM_{M}(U,V) = \bigcup_{\hspace{-2em} \mathrlap{T \in \BF'(V)}} \im (\circ_{T}) \cup \bigcup_{\hspace{-2em} \mathrlap{(W,T) \in \BF''(U;V)}} \im (\circ_{W,T}), \]
  where:
  \begin{align*}
    \BF'(V) & \coloneqq \{ T \subset V \mid \# T \geq 2 \}, \\
    \BF''(U;V) & \coloneqq \{ (W,T) \mid W \subset U, \; T \subset V, \; 2 \cdot \#T + \#W \geq 2 \}.
  \end{align*}
  Note that in the description of $\partial \SFM_{n}(U,V)$, there is an additional condition $W \cup T \subsetneq U \cup V$.
  Indeed in $\SFM_{n}$ the normalization by the affine group prevents the points from becoming infinitesimally close all at once; in $\SFM_{M}$, no such normalization occurs.

  Then the fiberwise boundary of the canonical projection $p_{U,V}$ is given by:
  \[ \SFM^{\partial}_{M}(U,V) =  \bigcup_{\hspace{-2em} \mathrlap{T \in \BF'(V,J)}} \im (\circ_{T}) \cup \bigcup_{\hspace{-2em} \mathrlap{(W,T) \in \BF''(U, I; V, J)}} \im (\circ_{W,T}) \subset \SFM_{M}(U \sqcup I, V \sqcup J), \]
  where
  \begin{align*}
    T \in \BF'(V,J) \subset \BF'(V \sqcup J) & \iff \#(T \cap J) \leq 1, \\
    (W,T) \in \BF''(U,I; V,J) \subset \BF''(U \sqcup I, V \sqcup J) & \iff V \cap T = \varnothing, \#(U \cap W) \leq 1.
  \end{align*}

  One can then check that the boundary faces of that decomposition correspond to the summands of the differential.
\end{proof}

\begin{definition}
  Define the \textbf{full colored graph complex}\index{fSGCR@$\fSGC_{R}$}
  \[ \fSGC_{R} \coloneqq \Tw \SGra_{R, \Omega^{*}(\partial M)}(\varnothing, \varnothing)[-n]. \]
\end{definition}

This is the (shifted) CDGA of colored, labeled graphs with only internal vertices.
The product is the disjoint union of graphs, thus $\fSGC_{R}[n]$ is free as an algebra, generated by the graded module $\SGC_{R}[n]$ of connected graphs.
Each $\Tw \SGra_{R}(U,V)$ is a module over the CDGA $\fSGC_{R}[n]$ by adding connected components.

\begin{definition}
  The \textbf{colored partition function} $\SZphi : \fSGC_{R}[n] \to \R$\index{ZSphi@$\SZphi$} is the CDGA morphism given by the restriction in empty arity
  \[ \SZphi \coloneqq \omega|_{(\varnothing,\varnothing)} : \Tw \SGra_{R, \Omega^{*}(\partial M)}(\varnothing, \varnothing) \to \OmPA^{*}(\SFM_{M}(\varnothing, \varnothing)) = \OmPA^{*}(\{*\}) = \R. \]
  Let $\R^{S}_{\varphi}$\index{RSphi@$\R^{S}_{\varphi}$} be the one-dimensional $\fSGC_{R}[n]$-module induced by $\SZphi$.
\end{definition}

\begin{remark}
  \label{cnfbnd.rmk.maurer-cartan}
  Since $\fSGC_{R}[n]$ is a CDGA, the dual module $\SGC_{R}^{\vee}$ is naturally an $\hoLie_{n}$-algebra.
  The differential of $\fSGC_{R}$ cannot create more than two connected components, thus $\SGC_{R}^{\vee}$ is actually a $\Lie_{n}$-algebra.
  The differential blows up vertices (like in $\GC_{n}^{\vee}$) and joins pairs of vertices by an edge, while the Lie bracket joins two graphs by an edge.
  The algebra morphism $\SZphi$ is uniquely determined by its restriction $\szphi$\index{zsphi@$\szphi$} to $\SGC_{R}$, which can be seen as a Maurer--Cartan element in the $\Lie_{n}$-algebra $\SGC_{R}^{\vee}$.
\end{remark}

\begin{definition}
  The \textbf{reduced colored labeled graph comodule} $\SGraphs_{R, \Omega^{*}(\partial M)}^{c_{\varphi}, \szphi}$\index{SGraphsRphi@$\SGraphs_{R, \Omega^{*}(\partial M)}^{c_{\varphi},\szphi}$} is the bisymmetric collection given in each arity by:
  \[ \SGraphs_{R, \Omega^{*}(\partial M)}^{c_{\varphi},\szphi}(U, V) \coloneqq \R^{S}_{\varphi} \otimes_{\fSGC_{R}[n]} \Tw \SGra_{R, \Omega^{*}(\partial M)}(U,V). \]
\end{definition}

\begin{proposition}
  The bisymmetric collection $\SGraphs_{R, \Omega^{*}(\partial M)}^{c_{\varphi},\szphi}$ forms a Hopf right comodule over $\SGraphs_{n}$, and the map $\omega : \Tw \SGra_{R, \Omega^{*}(\partial M)} \to \OmPA^{*}(\SFM)$ factors through a Hopf right comodule morphism:\index{womega@$\omega(\Gamma)$}
  \[ \omega : (\SGraphs_{R,\Omega^{*}(\partial M)}^{c_{\varphi},\szphi}, \SGraphs_{n}) \to (\OmPA^{*}(\SFM_{M}), \OmPA^{*}(\SFM_{n})). \]
\end{proposition}
\begin{proof}
  Identical to the proof of Proposition~\ref{cnf.prop.mph-1}.
\end{proof}

We now show that our comodule is quasi-isomorphic to a simpler one, in the spirit of Corollary~\ref{cnfbnd.cor.sgraphs-c0}.
Recall the circular graph from Figure~\ref{cnfbnd.fig.circular}.
We will need an explicit name for the circular graph of length $1$ in $\GC_{n-1}$, say:\index{l@$\ell_1$}
\begin{equation}
  \label{cnfbnd.eq.ell}
  \ell_1 \coloneqq
  \raisebox{-.5\height}{
    \begin{tikzpicture}
      \node (o) {};
      \draw[dotted] (o) -- ($(o)+(1.5,0)$);
      \node[intv, right={0.5 of o}] (i1) {};
      \node[intv, above={1 of i1}] (j1) {};
      \draw[->] (j1) edge[bend left] (i1) edge[bend right] (i1);
    \end{tikzpicture}
  }
\end{equation}

\begin{lemma}
  \label{cnfbnd.lem.cphi-cm}
  The Maurer--Cartan element $c_{\varphi} \in \SGC_{n}^{\vee} \otimes \Omega_{\mathrm{triv}}^{*}(\partial M)$ is gauge equivalent to\index{cM@$c_{M}$}
  \begin{equation*}
    c_{M} \coloneqq c_{0} \otimes 1 + \ell_1^{\vee} \otimes e_{\partial M},
  \end{equation*}
  where:
  \begin{itemize}
  \item $c_{0} \in \SGC_{n}^{\vee}$ is the linear part of $c$ (Definition~\ref{cnfbnd.def.c0});
  \item $\ell_1 \in \SGC_{n}$ is the circular graph of length $1$ in $\GC_{n-1}$ (Equation~\eqref{cnfbnd.eq.ell});
  \item $e_{\partial M} \coloneqq \chi(\partial M) \cdot \vol_{\partial M}$\index{edm@$e_{\partial M}$} is the Euler class of $\partial M$ (and $\vol_{\partial M} = g_{\partial}(\vol_{R_{\partial}})$ where $\vol_{R_{\partial}}$ is a fixed cocycle satisfying $f_{\partial}(\vol_{R_{\partial}}) = \vol_{B_{\partial}}$).
  \end{itemize}
\end{lemma}

\begin{proof}
  Let us first show that $c_{\varphi}$ agrees with $c_{M}$ on circular graphs, i.e.\ that $c_{\varphi}$ vanishes on circular graphs except for $\ell_1$, on which it is equal to $e_{M}$.
    The aerial circular graphs which produce nontrivial classes have odd length,\footnote{Depending on the parity of $n$, they are congruent to either $1$ or $3$ modulo $4$.} hence they contain a vertex with in and out valences of $1$.
    One can construct the propagator $\psi$ on $D = M \cup_{\partial M} \bar{M}$ (see Section~\ref{cnfbnd.sec.propagator}) so that $\varphi$ exhibits symmetry properties that make $c_{\varphi}$ vanish on these graphs, by an argument similar to Kontsevich's trick~\cite[Lemma~2.1]{Kontsevich1994}.
    Moreover, using a similar argument, we see that $c_{\varphi}$ vanishes on the circular graphs from $\GC_{n-1}$, because apart from the first one, they all contain a copy of the following graph (which represent a ``bivalent'' terrestrial vertex, the ``outside'' vertices may be the same one):
  \begin{center}
    \begin{tikzpicture}
      \node (o) {};
      \draw[dotted] (o) -- ($(o)+(4,0)$);
      \node[intv, right={0.5 of o}] (i1) {};
      \node[intv, right={1 of i1}] (i2) {};
      \node[intv] (j1) at ($(i1)+(0.6,1)$) {};
      \draw[->] (j1) edge (i1) edge (i2);
      \node[intv, right={1 of i2}] (i3) {};
      \node[intv] (j2) at ($(i2)+(0.6,1)$) {};
      \draw[->] (j2) edge (i2) edge (i3);
    \end{tikzpicture}
  \end{center}
  Thus when computing the integral defining $c$, one can consider the automorphism of $\SFM_{n}(\{u_{1}, u_{2}, v\}, \varnothing)$ which maps $v$ to its symmetric with respect to the barycenter of $u_{1}$ and $u_{2}$ inside $\R^{n-1} \times \{ 0 \}$, which changes the sign of the integral.
  The only exception is $\ell_1$, for which we can explicitly compute $c_{M}(\ell_1) = e_{M}$ using the fourth property of $\varphi$ in Proposition~\ref{cnfbnd.prop.propagator} and the standard fact that the Euler class $e_{\partial M}$ is the image of the diagonal class of $\partial M$ under the multiplication map.

  It then follows that $c_{\varphi} - c_{M}$ is a Maurer--Cartan element in the twisted Lie algebra $\SGC_{n}^{\vee, c_{0}} \otimes \Omega^{*}(\partial M).$
  According to Proposition~\ref{cnfbnd.prop.h-sgcn}, if we forget about circular graphs the cohomology of $\SGC_{n}^{\vee, c_{0}}$ is concentrated below degree $-(n-1)$, and the cohomology of $\partial M$ below degree $n-1$.
  It follows that $c_{\varphi} - c_{M}$ lives in a Lie subalgebra of $\SGC_{n}^{\vee, c_{0}} \otimes \Omega^{*}(\partial M)$ whose cohomology vanishes in positive degrees.
  Thus, by obstruction theory, $c_{\varphi} - c_{M}$ is gauge trivial and we have proved the proposition.
\end{proof}

Recall the cooperad $\SGraphs_{n}^{c_{0}}$ from Definition~\ref{cnfbnd.def.sgraphs0}.
We can define a Hopf right comodule $\SGraphs_{R, \Omega^{*}(\partial M)}^{c_{M}, \szphi}$ over this cooperad, by twisting $\SGra_{R, \Omega^{*}(\partial M)}$ with respect to $c_{M}$ instead of $c_{\varphi}$.

\begin{corollary}
  There is zigzag of quasi-isomorphisms of Hopf right comodules:
  \[
    (\SGraphs_{R,\Omega^{*}(\partial M)}^{c_{\varphi},\szphi}, \SGraphs_{n}) \qiso* \cdot \qiso (\SGraphs_{R, \Omega^{*}(\partial M)}^{c_{M}, \szphi}, \SGraphs^{c_{0}}_{n}).
    \qed
  \]
\end{corollary}

Moreover, we can now get rid of the $\Omega^{*}(\partial M)$ and label the terrestrial vertices by elements of $R_{\partial}$ instead.
This was not possible before vecause $c_{\varphi}$ could produce forms that were not in the image of $g_{\partial} : R_{\partial} \qiso \Omega^{*}(\partial M)$, but this is not the case for $c_{M} \otimes 1$.
We thus define a Hopf right comodule $(\SGraphs_{R,R_{\partial}}^{c_{M}, \szphi}, \SGraphs_{n}^{c_{0}})$, and we obtain:
\begin{lemma}
  There is a quasi-isomorphism of Hopf right comodules induced by $g_{\partial} : R_{\partial} \qiso \Omega^{*}(\partial M)$:
  \[ (\SGraphs_{R,R_{\partial}}^{c_{M}, \szphi}, \SGraphs_{n}^{c_{0}}) \qiso (\SGraphs_{R, \Omega^{*}(\partial M)}^{c_{0}, \szphi}, \SGraphs^{c_{0}}_{n}).
    \qed \]
\end{lemma}

\subsection{Labeled graphs and proof of Theorem~\ref{cnfbnd.thm.A}}
\label{cnfbnd.sec.labeled-graphs}

We now introduce a variant $\Graphs^{\zphi}_{R}$ of the graph comodule from Chapter~\ref{chap.2}.
Recall that $j^{*} \varphi = \varphi|_{\SFM_{M}(\{2\},\{1\})}$ is equal to $(g_{\partial} \otimes g)(\sigma_{R})$, where $\sigma_{R} = \sum \sigma' \otimes \sigma'' \in R_{\partial} \otimes R$ was defined in Proposition~\ref{cnfbnd.prop.propagator}.

\begin{definition}
  \label{cnfbnd.def.graphs-phi}
  The CDGA of reduced $R$-labeled graphs $\Graphs^{\zphi}_{R}(V)$\index{GraphsRphi@$\Graphs_{R}^{\zphi}$} on the finite set $V$ is spanned by directed graphs with external (corresponding to $V$) vertices and internal vertices, with all the vertices labeled by $R$.
  The algebra structure glues graphs along external vertices, and the comodule structure collapses subgraphs.
  A connected component with only internal vertices is identified with the real number given by the partition function $\SZphi$, by viewing the component as having no terrestrial vertices.
  The differential has three parts:
  \begin{itemize}
  \item Splitting any edge and multiplying the labels of the endpoints by $\Delta_{R}$;
  \item Contracting an edge between an internal vertex and another vertex, multiplying the labels;
  \item Removing one internal vertex $u$.
    Let $x \in R$ be the label of $u$, and let $v_{1}, \dots, v_{k}$ be the vertices connected to $u$, with respective labels $y_{1}, \dots, y_{k} \in R$.
    Then in the differential, the vertex $u$ and its incident edges are removed, the label of $v_{i}$ becomes $y_{i} \sigma''_{i}$, and the new graph is multiplied by $\varepsilon_{\partial}(\rho(x) \sigma_{1}' \dots \sigma_{k}') \in \R$.
    See Figure~\ref{cnfbnd.fig.third-part-diff} for an example.
  \end{itemize}
\end{definition}

\begin{figure}[htbp]
  \centering
  \begin{equation*}
    \begin{gathered}
      \begin{tikzpicture}
        \node[intv, label={$x$}] (i) {};
        \node[extv, label={$y_{1}$}] (u1) at ($(i)+(-2,-1)$) {1};
        \node[extv, label={$y_{2}$}, right={1 of u1}] (u2) {2};
        \node[extv, label={$y_{3}$}, right={1 of u2}] (u3) {3};
        \draw[->] (i) edge (u1) edge (u2) edge (u3);
      \end{tikzpicture}
    \end{gathered}
    \mapsto
    \varepsilon_{\partial}(\rho(x) \sigma'_{1} \sigma'_{2} \sigma'_{3})
    \cdot
    \begin{gathered}
      \begin{tikzpicture}
        \node[extv, label={$y_{1} \sigma''_{1}$}] (u1) {1};
        \node[extv, label={$y_{2} \sigma''_{2}$}, right={1 of u1}] (u2) {2};
        \node[extv, label={$y_{3} \sigma''_{3}$}, right={1 of u2}] (u3) {3};
      \end{tikzpicture}
    \end{gathered}
  \end{equation*}
  \caption{The third part of the differential in $\Graphs^\varphi_R$}
  \label{cnfbnd.fig.third-part-diff}
\end{figure}

Roughly speaking, in this third part of the differential, the removed vertex becomes infinitesimally close to the boundary of $M$.
Each edge thus becomes $j^{*}\varphi = \varphi|_{\SFM_{M}(\{1\},\{2\})} = (g_{\partial} \otimes g)(\sigma_{R})$, and we multiply the labels accordingly.
The vertex $u$ is then alone in its connected component and is thus identified with the integral of its label.

For each finite set of external vertices $V$, there is a map $\nu_{R} : \SGraphs^{c_{M}, \szphi}_{R, R_{\partial}}(\varnothing, V) \to \Graphs^{\zphi}_{R}(V)$ defined as follows.
Given some graph $\Gamma \in \SGraphs^{c_{M}, \szphi}_{R, R_{\partial}}(\varnothing, V)$, we remove all the edges from an aerial vertex to a terrestrial vertex and we multiply their endpoints by $\sigma_{R} \in R_{\partial} \otimes R$.
We then obtain a graph where all the terrestrial vertices are zero-valent, and we identify those with the real number given by the integral over $\partial M$ of their labels.
In this way, we obtain an element of $\Graphs^{\zphi}_{R}(V)$.
This is a chain map, the part of the differential described in Figure~\ref{cnfbnd.fig.third-part-diff} being induced by Figure~\ref{cnfbnd.fig.collapse-aerial-terrestrial}.

\begin{proposition}
  The symmetric collection $\Graphs^{\varphi}_{R}$ is a Hopf right co\-mo\-dule over $\Graphs_{n}$.
  The maps
  \[ \nu_{R} : \SGraphs^{c_{M},\szphi}_{R}(\varnothing, -) \qiso \Graphs^{\zphi}_{R} \]
  define a quasi-iso\-mor\-phism of Hopf right $\Graphs_{n}$-comodules.
\end{proposition}
\begin{proof}
  We use an argument similar to the one of Section~\ref{cnf.sec.morphisms-are-quasi}.
  Filter both complexes by the number of edges minus the number of vertices.
  The only part of the differential which remains on the $\EE^{0}$ page is the contraction of aerial edges and the contraction of a subgraph into a terrestrial vertex induced by $c_{M}$, which is nonzero on trees only (see Definition~\ref{cnfbnd.def.c0} and Lemma~\ref{cnfbnd.lem.cphi-cm}, the summand induced by $\ell_1 \otimes e_{\partial M}$ vanishes because it strictly decreases the filtration).

  It follows that both complexes split as a direct sum in terms of connected components, just like in Lemma~\ref{cnf.lem.partition-graph}.
  We can now reuse the same ``trick'' we used for the proof of Lemma~\ref{cnf.lemma.same-cohomology}, to show that in cohomology, there is only one label for each connected component.
  We thus reduce to the morphism $\nu : \SGraphs^{c_{0}}_{n}(\varnothing, V) \to \Graphs_{n}(V)$ from Proposition~\ref{cnfbnd.prop.qiso-sc} (tensored with the identity of $R$ for each connected component).
  Since that morphism is a quasi-isomorphism, we obtain that the induced morphism on the $\EE^{1}$ page is an isomorphism.
  It follows by standard spectral sequence arguments (the filtration is bounded below for a fixed $V$) that the morphism of the proposition is a quasi-isomorphism.
\end{proof}

We would now like to connect $\Graphs^{\zphi}_{R}(V)$ with $\GGt{A}(V)$.

\begin{definition}
  The \textbf{full labeled graph complex} $\fGC_{R}$\index{fGCR@$\fGC_{R}$, full graph complex} is the CDGA of aerial graphs with only internal vertices, the differential being defined in a manner similar to Definition~\ref{cnfbnd.def.graphs-phi}.
  The \textbf{partition function} $\Zphi : \fGC_{R}[n] \to \R$\index{partition function} is induced by $\SZphi$.
  Let $\R_{\varphi}$\index{Rphi@$\R_{\varphi}$} be the one-dimensional $\fGC_{R}[n]$-module induced by $\Zphi$.
\end{definition}

The (shifted) commutative algebra $\fGC_{R}$ is free, generated by the submodule $\GC_{R}$ of connected graphs.
The differential of $\fGC_{R}$ turns $\GC_{R}^{\vee}$ into a $\Lie_{n}$-algebra up to homotopy (see Remark~\ref{cnfbnd.rmk.maurer-cartan}, and note that the part of the differential that removes one vertex may produce more than two connected components, so this is not a $\Lie_{n}$-algebra), and $\Zphi$ defines a Maurer--Cartan element $\zphi \in \GC_{R}^{\vee}$.

We can reuse the counting argument from Proposition~\ref{cnf.prop.strong-vanishing} (and the last property of the propagator in Proposition~\ref{cnfbnd.prop.propagator}) to show that the partition function vanishes on any graph with no bivalent vertex labeled by $1_{R}$.
We can then define $\Ze : \fGC_{R} \to \R$\index{Ze@$\Ze$} to be the algebra morphism which vanishes on all connected graphs $\gamma$ except those with a single vertex.
If that vertex is labeled by $x \in R$ we set $\Ze(\gamma) = \int_{M} g(x)$.

Unfortunately, we still cannot prove directly that $\Zphi = \Ze$; however, we can follow the discussion at the beginning of Section~\ref{cnf.sec.proof-theorem}.
Let $\Omega^{*}(\Delta^{1}) = S(t,dt)$ be the algebra of polynomials forms on $\Delta^{1}$, a path object in the category of CDGAs.
We then obtain:

\begin{proposition}
  There exists a homotopy $h : \fGC_{R}[n] \to \Omega^{*}(\Delta^{1})$ such that the following diagram commutes:
  \[ \begin{tikzcd}
      {} & \fGC_{R}[n] \dlar[swap]{\Zphi} \drar{\Ze} \dar{h} \\
      \R & \Omega^{*}(\Delta^{1}) \lar{\sim}[swap]{d_{1}} \rar{d_{0}}[swap]{\sim} & \R
    \end{tikzcd}
  \qed \]
\end{proposition}

The reduced labeled graph comodule can be viewed as a tensor product $\R_{\varphi} \otimes_{\fGC_{R}[n]} \Tw \Gra_{R}(V)$.
We can then define $\Graphs'_{R} \coloneqq \Omega^{*}(\Delta^{1})_{h} \otimes_{\fGC_{R}[n]} \Tw \Gra_{R}$ and $\Graphs_{R}^{\ze} \coloneqq \R_{\varepsilon} \otimes_{\fGC_{R}[n]} \Tw \Gra_{R}$\index{GraphsRe@$\Graphs_{R}^{\ze}$}.
Both symmetric collections define Hopf right $\Graphs_{n}$-comodules.

\begin{proposition}
  \label{cnfbnd.prop.zigzag-homotopy}
  We have a zigzag of quasi-isomorphisms of comodules:
  \[ \Graphs_{R}^{\ze} \qiso* \Graphs'_{R} \qiso \Graphs_{R}^{\zphi}. \]
\end{proposition}
\begin{proof}
  The $\fGC_{R}[n]$-module $\Tw \Gra_{R}(V)$ is quasi-free and is equipped with a good filtration, hence $\Tw \Gra_{R}(V) \otimes_{\fGC_{R}[n]} (-)$ preserves quasi-isomorphisms.
  The claim now follows from the previous proposition.
\end{proof}

We can now do the same construction as $\Graphs^{\ze}_{R}$ but with $B$ replacing $R$, $\Delta_{B}$ (from Definition~\ref{cnfbnd.def.delta-b}) replacing $\Delta_{R}$, $\sigma_{B}$ replacing $\sigma_{R}$, and taking as ``partition function'' the map which sends a graph with a single vertex labeled by $b$ to $\varepsilon_{B}(b)$ and all other connected graphs to zero.
We obtain a Hopf right $\Graphs_{n}$-comodule $\Graphs_{B}$, where graphs with internal components containing at least two vertices vanish, and an isolated internal vertex labeled by $b \in B$ is identified with $\varepsilon_{B}(b)$.

Finally, we can consider a quotient $\Graphs_{A}$\index{GraphsA@$\Graphs_{A}$} of $\Graphs_{B}$, where we apply the projection $\pi : B \to A$ to all the remaining labels.

It is not necessarily the case that the two morphisms $\varepsilon, \varepsilon' : \operatorname{cone}(\rho) \to \R[-n+1]$, defined respectively by the composites $(\int_{M} g(-), \int_{\partial M} g_{\partial}(-))$ and $(\varepsilon_{B}f, \varepsilon_{B_{\partial}}f_{\partial})$, are equal.
Nevertheless, up to rescaling $\varepsilon$ (which induces an automorphism of $\Graphs^{\ze}_{R}$), they induce the same map up to quasi-isomorphism, $H^{n-1}(\cone(\rho))$ being one-dimensional.
Since all cochain complexes are fibrant and cofibrant, we obtain a diagram:
\begin{equation}
  \begin{tikzcd}
    {} & \cone(\rho) \dlar[swap]{\varepsilon} \drar{\varepsilon'} \dar{\varepsilon_{h}} \\
    \R[-n+1] & P[-n+1] \lar{\sim}[swap]{d_{1}} \rar{d_{0}}[swap]{\sim} & \R[-n+1]
  \end{tikzcd}
\end{equation}
where $P$ is the standard path object for the base field in the category of cochain complexes.
The chain map $\varepsilon_{h}$ is uniquely determined by a graded map $h : \cone(\rho) \to \R[-n+1]$ satisfying $\varepsilon(x) - \varepsilon'(x) = h(dx)$.

Since the product of $\fGC_{R}$ is merely given by the disjoint union of graphs, this yield a homotopy between the two morphisms $\fGC_{R}[n] \to \R$ induced.
Similarly to Proposition~\ref{cnfbnd.prop.zigzag-homotopy}, we then obtain a zigzag of quasi-isomorphisms:
\begin{equation}
  \Graphs^{\ze}_{R} \qiso* \Graphs''_{R} \qiso \Graphs^{\ze'}_{R}.
\end{equation}

Now $\varepsilon'$ is compatible with $(\varepsilon_{B}, \varepsilon_{B_{\partial}})$, we obtain the following proposition:

\begin{proposition}
  We have quasi-isomorphisms of Hopf right $\Graphs_{n}$-comodules
  \[ \Graphs_{R}^{\ze'} \tkrightarrow{f_{*}} \Graphs_{B} \tkrightarrow{\pi_{*}} \Graphs_{A} \]
  which apply $f : R \to B$ (resp.\ $\pi : B \to A$) to all the labels.
\end{proposition}
\begin{proof}
  Filter the graph comodules by the number of edges.
  On each $\EE^{0}$ page, only the differentials from $R$ (resp.\ $B$, $A$) remain.
  Since $f$ and $\pi$ are a quasi-isomorphism, it follows that the induced morphisms on $\EE^{1}$ pages is an isomorphism.
  Standard spectral sequence arguments then imply that $f_{*}$ and $\pi_{*}$ are quasi-iso\-mor\-phisms.
\end{proof}

We can now define a morphism $\Graphs_{A} \to \GGt{A}$ which maps all graphs with internal vertices to zero, and which sends an edge $e_{vv'}$ between external vertices to $\tilde{\omega}_{vv'}$.
Then we can mimic the proof of Proposition~\ref{cnf.prop.graph-A-G-A} in a straightforward way (note that as soon as we filter by $\# \text{edges} - \# \text{vertices}$ the perturbed relations of $\GGt{A}$ become the usual relations of $\GG{A}$):
\begin{proposition}
  \label{cnfbnd.qiso-graphs-ga}
  This defines a quasi-isomorphism of right Hopf comodules:
  \[ (\Graphs_{A}, \Graphs_{n}) \to (\GGt{A}, \enV). \]
\end{proposition}

We can also copy the proof of Proposition~\ref{cnf.prop.surj-cohom} and use Theorem~\ref{cnfbnd.thm.ga-same-cohom} to obtain:
\begin{proposition}
  The morphism $\omega : \SGraphs^{c_{\varphi}, \szphi}_{R}(\varnothing, -) \to \OmPA^{*}(\SFM_{M}(\varnothing, -))$ is a quasi-isomorphism.
\end{proposition}

Finally we can summarize this section as:
\begin{theorem}[Precise version of Theorem~\ref{cnfbnd.thm.A}]
  \label{cnfbnd.thm.model-ga}
  Let $M$ be a simply connected, smooth, compact manifold with boundary of dimension at least $5$, and assume either that it admits a surjective pretty model or that $\dim M \geq 7$ so that it admits a PLD.
  Let $A$ be the CDGA model built from the resulting PLD model.

  Then the symmetric collection of CDGAs $\GGt{A}$ is quasi-isomorphic to $\OmPA^{*}(\SFM_{M}(\varnothing,-))$.
  If moreover $M$ is framed, then the Hopf right comodule $(\GGt{A}, \enV)$ is quasi-isomorphic to $(\OmPA^{*}(\SFM_{M}(\varnothing, -)), \OmPA^{*}(\FM_{n}))$.
  \qed
\end{theorem}

\begin{corollary}[Corollary~\ref{cnfbnd.cor.I}]
  The real homotopy type of configuration spaces on a simply connected, smooth manifold $M$ with simply connected boundary only depends on the real homotopy type of the manifold as soon as:
  \begin{itemize}
  \item either $\dim M \geq 5$ and $M$ admits a surjective pretty model;
  \item or $\dim M \geq 7$.
  \end{itemize}
\end{corollary}

\begin{example}
  We can apply this to $M = D^{n}$, using the surjective pretty model from Example~\ref{cnfbnd.exa.model-dn}.
  Recall that in this case, $A = \R$, $\Delta_{A} = 0$, $\sigma_{A} = 0$, and $\GGt{A}$ is isomorphic to $\enV$ as a Hopf right comodule over itself (see Example~\ref{cnfbnd.exa.model-dn-trois}).
  We then ``recover'' the already known fact that $\SFM_{D^{n}}(\varnothing,-)$ is (Hopf) formal as a right $\FM_{n}$-module (though that ``proof'' is of course circular).

  The Hopf right $\Graphs_{n}$-comodule $\Graphs_{A}$ is isomorphic to $\Graphs_{n}$ seen as a comodule over itself.
  The augmentation $\varepsilon_{B} : B \to \R[-n+1]$ yields a Maurer--Cartan element $\ze$\index{ze@$\ze$} in the abelian $\hoLie_{n}$-algebra $\GC_{A}^{\vee}$ (see Remark~\ref{cnfbnd.rmk.maurer-cartan}), given in the dual basis by the graph with a single vertex labeled by $\vol_{n}$.
  The twisted $\hoLie_{n}$-algebra $\GC_{A}^{\vee,\ze}$ is then isomorphic to $\GC^{\vee}_{n}$ from Section~\ref{cnfbnd.sec.graph-complexes}.\footnote{The twisting ensures that dead ends are not contractible, see the proof of Proposition~\ref{cnf.prop.mph-2}.}
\end{example}

\subsection{Proof of Theorem~\ref{cnfbnd.thm.B}}

\begin{proposition}
  Assume that $n = \dim M \ge 5$.
  Then for all finite sets $U$ and $V$, $H^{*}(\SGraphs^{c_{\varphi},\szphi}_{R}(U,V))$ has the same dimension in each degree as
  \[H^{*}(\SFM_{M}(U,V)) \cong H^{*}(\Conf_{U}(\partial M)) \otimes H^{*}(\Conf_{V}(M)). \]
\end{proposition}
\begin{proof}
  We use a spectral sequence argument similar to the proof in~\cite[Section~5]{Willwacher2015a}.
  The assumption about the dimension of $M$ allows us to apply Theorem~\ref{cnfbnd.thm.model-ga} to $M$ and Theorem~\ref{cnf.thm.A} to $\partial M$.

  First filter $\SGraphs^{c_{\varphi}, \szphi}_{R}(U,V)$ by the number of edges minus the number of vertices as in the proof of the previous proposition.
  We get as an $\EE^{0}$ page the graded module $\SGraphs_{R}(U,V)$ together with the differential which contracts edges connected to an aerial internal vertex.
  Filtering this new complex by the number of vertices that are not of the type ``bivalent, internal, aerial, and labeled by $1_{R}$'' as in the proof of~\cite[Section~5]{Willwacher2015a}, and using the previous proposition, we obtain that $\EE^{0}$ is quasi-isomorphic to its quotient $\mathcal{Q}(U,V)$ in which:
  \begin{itemize}
  \item graphs containing bivalent internal aerial vertices labeled by $1_{R}$ other than those appearing in a zigzag of the type shown in Figure~\ref{cnfbnd.fig.zigzag-graphs_R} are set to zero;
  \item graphs with edges between aerial and terrestrial vertices other than those appearing in a zigzag of the type shown in Figure~\ref{cnfbnd.fig.zigzag-graphs_R} are set to zero;
  \item an edge $e_{vv'}$ between aerial vertices is identified with $(-1)^{n} e_{v'v}$.
  \end{itemize}

  \begin{figure}[htbp]
    \centering
    \begin{tikzpicture}
      \node (0) {};
      \node (o) [right = 3cm of 0] {};
      \draw (0) to (o);
      \node[extv, label = {\scriptsize $x$}] (1) [right = 0.5cm of 0] {$1$};
      \node[extv, label = {\scriptsize $y$}] (2) [right = 1.3cm of 1] {$2$};
      \node[intv, label = {\scriptsize $1_{R}$}] (i) [right = 0.5cm of 1, yshift=1cm] {};
      \draw[->] (i) edge (1) edge (2);
    \end{tikzpicture}
    \caption{A zigzag which represents an edge in $\Graphs_{R_{\partial}}(U)$.}
    \label{cnfbnd.fig.zigzag-graphs_R}
  \end{figure}

  We may identify the underlying graded module of $\mathcal{Q}(U,V)$ with the tensor product $\Graphs_{R}(V) \otimes \Graphs_{R_{\partial}}(U)$.
  The induced differential $d^{0}$ contracts edges in $\Graphs_{R}(V)$ (but not $\Graphs_{R_{\partial}}(V)$).
  We obtain that the underlying graded module of the $\EE^{1}$ page is isomorphic to
  \[ H^{*}(\Graphs_{R}(V), d^{0}) \otimes \Graphs_{R_{\partial}}(V). \]

  The differential $d^{1}$ is induced by the part of the differential which decreases the filtration by exactly $1$.
  By comparing with the proof of Theorem~\ref{cnfbnd.thm.model-ga} (and by extension with the proof of Theorem~\ref{cnf.thm.A}), we obtain that the complex
  \[\bigl( H^{*}(\Graphs_{R}(V), d^{0}), d^{1} \bigr) \]
  is isomorphic to $\GG{A}(V)$, which has the same Betti numbers as $\Conf_{V}(M)$ by Theorem~\ref{cnfbnd.thm.ga-same-cohom}.
  Moreover, by inspection, $d^{1}$ is given on $\Graphs_{R_{\partial}}(U)$ by the contraction of edges (and the multiplication of labels).
  Applying Theorem~\ref{cnf.thm.A} to $\partial M$, we thus obtain that the $\EE^{2}$ page of the spectral sequence is isomorphic to:
  \[ H^{*}(\GG{A}(V)) \otimes \GG{B_{\partial}}(U) \cong H^{*}(\Conf_{V}(M)) \otimes \GG{B_{\partial}}(U). \]

  Since any remaining cocycle is represented by a cocycle in $\SGraphs^{c_{\varphi}, \szphi}_{R}(U,V)$, we obtain that the spectral sequence abuts at this stage.
\end{proof}

\begin{proposition}
  Assume that $n = \dim M \ge 5$.
  Then the morphism
  \[ \omega : \SGraphs^{c_{\varphi},\szphi}_{R,\Omega^{*}(\partial M)}(U,V) \to \OmPA^{*}(\SFM_{M}(U,V)) \]
  is a quasi-isomorphism of CDGAs for all $U$, $V$.
\end{proposition}
\begin{proof}
  We know that $H^{*}(\SFM_{M}(U,V))$ is isomorphic to the tensor product
  \[ H^{*}(\Conf_{V}(M)) \otimes H^{*}(\Conf_{U}(\partial M)). \]
  We can then use the previous proposition and an inductive argument similar to the proof of Proposition~\ref{cnf.prop.surj-cohom} to obtain the result.
  Indeed we can represent any cocycle in $H^{*}(\Conf_{V}(M))$ by a cocycle in $\Graphs^{\varphi}_{R}(V)$, and each cocycle in $H^{*}(\Conf_{U}(\partial M))$ by a cocycle in $\Graphs^{\varphi_{\partial}}_{R_{\partial}}(U)$, and then take the disjoint union of the resulting graphs.
\end{proof}

We can now bundle everything together into a theorem, with a summary of our hypotheses.

\begin{theorem}[Precise version of Theorem~\ref{cnfbnd.thm.B}]
  \label{cnfbnd.thm.sgraphs-sfm}
  Let $M$ be a smooth, simply connected manifold with a simply connected boundary, of dimension $n \geq 5$.
  Suppose also that $M$ admits a surjective pretty model, or that $n \geq 7$ so that it admits a PLD model as in Equation~\eqref{cnfbnd.eq.nice-diagram}.

  Then for each $U$, $V$, then the CDGA $\SGraphs^{c_{M}, \szphi}_{R,R_{\partial}}(U,V)$ is quasi-isomorphic to $\OmPA^{*}(\SFM_{M}(U,V))$, and this is compatible with the symmetric group actions.

  If $M$ is framed, then together with Willwacher's morphism~\cite{Willwacher2015a}, this defines a quasi-isomorphism of right Hopf comodules
  \[ (\SGraphs^{c_{M}, \szphi}_{R, R_{\partial}}, \SGraphs^{c_{0}}_{n}) \simeq (\OmPA^{*}(\SFM_{M}), \OmPA^{*}(\SFM_{n})). \]
\end{theorem}

It would be an interesting question to know how much $\szphi$ can be simplified.
