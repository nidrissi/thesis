% liminary.tex -*- mode: latex -*-

\cleardoublepage
\phantomsection
\addcontentsline{toc}{chapter}{Résumé / Abstract}
\begingroup
\small
\begin{french}
  \textbf{Résumé.}
  Dans une première partie, nous étudions l'opérade $\SC_{2}$ «Swiss-Cheese» de Voronov, qui gouverne l'action d'une algèbre $\DD_{2}$ sur une algèbre $\DD_{1}$.
  Nous construisons un modèle en groupoïdes de cette opérade et nous décrivons les algèbres sur ce modèle de manière similaire à la description classique des algèbres sur $H_{*}(\SC_{2})$.
  Nous étendons notre modèle en un modèle rationnel dépendant d'un associateur de Drinfeld, et nous le comparons au modèle qui existerait si l'opérade $\SC_{2}$ était formelle.

  Dans une seconde partie, nous étudions les espaces de configurations des variétés compactes, lisses, sans bord et simplement connexes.
  Nous démontrons sur $\R$ une conjecture de Lambrechts--Stanley qui décrit un modèle de tels espaces de configurations, avec comme corollaire leur invariance homotopique réelle.
  En nous fondant sur la preuve par Kontsevich de la formalité des opérades $\DD_{n}$, nous obtenons en outre que ce modèle est compatible avec l'action de l'opérade de Fulton--MacPherson quand la variété est parallélisée.
  Cela nous permet de calculer explicitement l'homologie de factorisation d'une telle variété.

  Enfin, dans une troisième partie, nous élargissons ce résultat à une large classe de variétés à bord.
  Nous utilisons d'abord une dualité de Poincaré--Lefschetz au niveau des chaînes pour calculer l'homologie des espaces de configurations de ces variétés, puis nous reprenons les méthodes du second chapitre pour obtenir le modèle, qui est compatible avec l'action de l'opérade Swiss-Cheese $\SC_{n}$.

  \noindent \textbf{Mots-clés : } opérades, espaces de configuration (topologie), variétés topologiques, topologie algébrique
\end{french}
\disfrpt{}
\vfill

\begin{english}
  \textbf{Abstract.}
  In a first part, we study Voronov's ``Swiss-Cheese'' operad $\SC_{2}$, which governs the action of a $\DD_{2}$-algebra on a $\DD_{1}$-algebra.
  We build a model in groupoids of this operad and we describe algebras over this model in a manner similar to the classical description of algebras over $H_{*}(\SC_{2})$.
  We extend our model into a rational model which depends on a Drinfeld associator, and we compare this new model to the one that we would get if the operad $\SC_{2}$ were formal.

  In a second part, we study configuration spaces of closed smooth simply connected manifolds.
  We prove over $\R$ a conjecture of Lambrechts--Stanley which describes a model of such configuration spaces, and we obtain as corollary their real homotopy invariance.
  Moreover, using Kontsevich's proof of the formality of the operads $\DD_{n}$, we obtain that this model is compatible with the action of the Fulton--MacPherson operad when the manifold is framed.
  This allows us to explicitly compute the factorization homology of such a manifold.

  Finally, in a third part, we expand this result to a large class of manifolds with boundary.
  We first use a chain-level Poincaré--Lefschetz duality result to compute the homology of the configuration spaces of these manifolds, then we reuse the methods of the second chapter to obtain our model, which is compatible with the action of the Swiss-Cheese operad $\SC_{n}$.

  \noindent \textbf{English title:} \emph{Operadic Formality and Configuration Spaces}

  \noindent \textbf{Keywords:} operads, configuration spaces (topology), topological manifolds, algebraic topology
\end{english}
\endgroup

\cleardoublepage
\begin{french}
  \section*{Organisation de cette thèse}
  \addcontentsline{toc}{chapter}{Organisation / Organization}

  Outre le Chapitre~\ref{chap.intro} qui sert d'Introduction, chacun des trois autres chapitres est tirée d'un article :
  \begin{itemize}
  \item le Chapitre~\ref{chap.1} est l'article «Swiss-Cheese Operad and Drinfeld Center», publié au \textit{Israel J. Math} en 2017~\cite{Idrissi2017} ;
  \item le Chapitre~\ref{chap.2} constitue la prépublication «The Lambrechts--Stanley Model of Configuration Spaces»~\cite{Idrissi2016} ;
  \item le Chapitre~\ref{chap.3} est tiré d'un travail en cours, en collaboration avec Pascal Lambrechts.
  \end{itemize}

  Tous ces chapitres sont en anglais.
  Un résumé substantiel en français de cette thèse est disponible sur les pages qui suivent.
  Le Chapitre~\ref{chap.1} est en grande partie indépendant des Chapitres~\ref{chap.2} et~\ref{chap.3} et peut être lu séparément.
\end{french}

\bigskip
\noindent\hrulefill{}

\disfrpt{}
\begin{english}
  \section*{Organization of this thesis}

  Besides the Introduction in Chapter~\ref{chap.intro}, each of the other three chapters is drawn from an article:
  \begin{itemize}
  \item Chapter~\ref{chap.1} is the article ``Swiss-Cheese Operad and Drinfeld Center'', published at the \textit{Israel J. Math} in 2017~\cite{Idrissi2017};
  \item Chapter~\ref{chap.2} is the preprint ``The Lambrechts--Stanley Model of Configuration Spaces''~\cite{Idrissi2016};
  \item Chapter~\ref{chap.3} is based on a current work-in-progress joint with Pascal Lambrechts.
  \end{itemize}

  All of these chapters are in English.
  A substantial summary of this thesis in French can be found in the next pages.
  Chapter~\ref{chap.1} is largely independent from Chapters~\ref{chap.2} and~\ref{chap.3} and can be read separately.
\end{english}
