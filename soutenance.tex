\documentclass{beamer}
\usepackage{polyglossia}
\setmainlanguage{english}

\title[Formalité opéradique \& homotopie esp. conf.]{Formalité opéradique et homotopie des espaces de configuration}
\subtitle{Operadic Formality and Homotopy of Configuration Spaces}
\author{\href{https://operad.fr}{Najib Idrissi}}
\date[Soutenance de thèse]{Soutenance de thèse -- 17 novembre 2017}
\institute[Université Lille 1]{%
  \centering
  \href{http://www.univ-lille1.fr}{\raisebox{-.5\height}{\includegraphics[height=1.7cm]{img/logo_univ.pdf}}}%
  \qquad%
  \href{https://math.univ-lille1.fr}{\raisebox{-.5\height}{\includegraphics[height=1.4cm]{img/logo_labo.pdf}}}%
  \qquad%
  \href{http://www.ens.fr}{\raisebox{-.5\height}{\includegraphics[height=1.5cm]{img/logo_ens.jpg}}}
}

\usepackage{amsmath, amssymb, mathtools, mathrsfs}
\usepackage[only,mapsfrom]{stmaryrd}
\usepackage{xcolor}
\usepackage{graphicx}
\usepackage{tikz}
\usetikzlibrary{calc,positioning,arrows,cd,fit,tikzmark}
\usepackage{animate}
\usepackage{cancel}
\usepackage{import}
\usepackage{array,booktabs}

\usetheme[block=fill, numbering=counter, progressbar=foot, sectionpage=simple]{metropolis}
\setbeamersize{text margin left = 3mm, text margin right = 3mm}

\hypersetup{pdfencoding=auto}

% math.tex -*- mode: latex -*-

% Fields
\newcommand{\K}{\Bbbk}
\newcommand{\Q}{\mathbb{Q}}
\newcommand{\R}{\mathbb{R}}

% quasi-iso ; qiso* = à l'envers
\NewDocumentCommand{\tkextarrow}{sO{}m}{%
  \mathrel{\vcenter{\hbox{%
    \begin{tikzpicture}
      \node[minimum width=0.5cm, align=center, anchor=south] (tmp) {$\scriptstyle #3$\\[-1ex]$\scriptstyle #2$};
      \path[draw,\IfBooleanTF{#1}{<-}{->}] (tmp.west) -- (tmp.east);
    \end{tikzpicture}%
  }}}%
}
\NewDocumentCommand{\tkrightarrow}{O{}m}{\tkextarrow[#1]{#2}}
\NewDocumentCommand{\tkleftarrow}{O{}m}{\tkextarrow*[#1]{#2}}
\NewDocumentCommand{\qiso}{s}{
  \mathrel{%
    \IfBooleanTF #1
    {\tkleftarrow{\sim}}
    {\tkrightarrow{\sim}}%
  }
}

% misc
\newcommand{\EE}{\mathsf{E}}
\newcommand{\vol}{\mathrm{vol}}

% Operads
\newcommand{\operadify}[1]{\mathtt{#1}}
% Anonymous
\newcommand{\PP}{\operadify{P}}
\newcommand{\QQ}{\operadify{Q}}
\newcommand{\CC}{\operadify{C}}
\newcommand{\DD}{\operadify{D}}
\newcommand{\MM}{\operadify{M}}
\newcommand{\NN}{\operadify{N}}
% Standard
\newcommand{\End}{\operadify{End}}
\newcommand{\Ass}{\operadify{Ass}}
\newcommand{\Ger}{\operadify{Ger}}
\newcommand{\Com}{\operadify{Com}}
\newcommand{\SC}{\operadify{SC}}
\newcommand{\Sc}{\operadify{sc}}
\newcommand{\en}[1][n]{\operadify{e}_{#1}}
\newcommand{\enV}[1][n]{\en[#1]^{\vee}}
\newcommand{\Lie}{\operadify{Lie}}
\newcommand{\hoLie}{\operadify{hoLie}}
\newcommand{\magma}{\Omega}
\newcommand{\tMagma}{{\magma\magma}}
\newcommand{\II}{\operadify{I}}
\newcommand{\Ln}{\operadify{L}_{n}}
% Braids & cie
\newcommand{\CoB}{\operadify{CoB}}
\newcommand{\CoP}{\operadify{CoP}}
\newcommand{\PaB}{\operadify{PaB}}
\newcommand{\PaP}{\operadify{PaP}}
\newcommand{\CoPB}{\operadify{CoPB}}
\newcommand{\PaPB}{\operadify{PaPB}}
\newcommand{\CD}{\operadify{CD}}
\newcommand{\Sh}{\operadify{Sh}}
\newcommand{\PaSh}{\operadify{PaSh}}
\newcommand{\PaCD}{\operadify{PaCD}}
% completions
\newcommand{\CDh}{\widehat{\CD}}
\newcommand{\PaBh}{\widehat{\PaB}}
\newcommand{\PaCDhp}{\widehat{\PaCD}_{+}}
\newcommand{\PaPCD}{\operadify{PaP}\widehat{\operadify{CD}}_{+}}
% Two-colored operads
\newcommand{\col}[1]{\mathfrak{#1}}
% Configuration spaces
\newcommand{\GG}[1]{\mathtt{G}_{#1}}
\newcommand{\GGt}[1]{\tilde{\mathtt{G}}_{#1}}
\newcommand{\FM}{\mathtt{FM}}
\newcommand{\ff}{\operadify{f}}
\newcommand{\SFM}{\mathtt{SFM}}
\newcommand{\OmPA}{\Omega_{\mathrm{PA}}}
\newcommand{\Disk}{\mathtt{Disk}}
\newcommand{\Embfr}{\operatorname{Emb}^{\mathrm{fr}}}
% Graphs
\newcommand{\Gra}{\mathtt{Gra}}
\newcommand{\Graphs}{\mathtt{Graphs}}
\newcommand{\SGra}{\mathtt{SGra}}
\newcommand{\SGraphs}{\mathtt{SGraphs}}
\newcommand{\fGC}{\mathrm{fGC}}
\newcommand{\fLoop}{\mathrm{fLoop}}
\newcommand{\GC}{\mathrm{GC}}
\newcommand{\SGC}{\mathrm{SGC}}
\newcommand{\fSGC}{\mathrm{fSGC}}
% Hadamard product
\newcommand{\hadprod}{\boxtimes}
% Partition function
\newcommand{\ZZ}[1]{\mathrm{Z}_{#1}}
\newcommand{\Zphi}{\ZZ{\varphi}}
\newcommand{\Ze}{\ZZ{\varepsilon}}
\newcommand{\SZphi}{\ZZ{\varphi}^{S}}
\newcommand{\SZe}{\ZZ{\varepsilon}^{S}}
\newcommand{\zz}[1]{\mathrm{z}_{#1}}
\newcommand{\zphi}{\zz{\varphi}}
\newcommand{\ze}{\zz{\varepsilon}}
\newcommand{\szz}[1]{\mathrm{z}^{S}_{#1}}
\newcommand{\szphi}{\szz{\varphi}}
\newcommand{\sze}{\szz{\varepsilon}}

% Operators
\DeclareMathOperator{\Conf}{Conf}
\DeclareMathOperator{\ob}{ob}
\DeclareMathOperator{\Hom}{Hom}
\DeclareMathOperator{\id}{id}
\DeclareMathOperator{\im}{im}
\DeclareMathOperator{\Tw}{Tw}
\DeclareMathOperator{\Def}{Def}
\DeclareMathOperator{\cone}{cone}
\DeclareMathOperator{\colim}{colim}
\DeclareMathOperator{\hoker}{hoker}
\DeclareMathOperator{\TotCof}{TotCof}
\DeclareMathOperator{\Emb}{Emb}

\newcommand{\CCE}{\mathrm{C}_{*}^{\mathrm{CE}}}

% Categories
\newcommand{\categorify}[1]{\mathsf{#1}}
\newcommand{\Zd}{\mathcal{Z}}  % centre de Drinfeld
\newcommand{\cC}{\categorify{C}}
\newcommand{\cM}{\categorify{M}}
\newcommand{\cN}{\categorify{N}}
\newcommand{\cdga}{\categorify{CDGA}}
\newcommand{\Set}{\categorify{Set}}
\newcommand{\rlz}[1]{\langle #1 \rangle^{\mathbb{L}}}

% orphans
\newcommand{\OO}{\mathcal{O}}
% boundary faces
\newcommand{\BF}{\mathcal{BF}}

% Pictures
\newcommand{\bdot}[2]{\draw (#1) node [fill=black, circle, inner
  sep=1.5pt] {} node[above] {#2}}
\newcommand{\wdot}[2]{\draw (#1) node [draw = black, fill=white, circle, inner
  sep=1.5pt] {} node[above] {#2}}
% Elements of CoB
\tikzstyle{cob}=[<->]

% Vertices:
% external
\tikzset{extv/.style={circle, fill = white, draw, minimum size=0.5cm, inner sep=0pt}}
% internal
\tikzset{intv/.style={circle, fill = black, draw, minimum size = 0.3cm, inner sep = 0pt}}
% unknown type
\tikzset{unkv/.style={circle, fill = gray, draw, minimum size = 0.3cm, inner sep = 0pt}}
% small point
\tikzset{pt/.style={circle, fill = black, draw, minimum size = 0.1cm, inner sep = 0pt}}

% torus
\newcommand{\mytorus}{(-3.5,0) .. controls (-3.5,1.5) and (-1.5,2) .. (0,2)}


% Include Inkscape graphics
% \inklude{width}{name}
% width is in multiples of \linewidth
\newcommand{\inklude}[2]{
  \begin{center}
    \def\svgwidth{#1\linewidth} \import{fig/}{#2.pdf_tex}
  \end{center}
}

\begin{document}

\maketitle

\begin{frame}
  \frametitle{Introduction}
  \begin{block}{Overall Goal}
    Study configuration spaces of manifolds:
    \[ \Conf_{k}(M) \coloneqq \{ (x_{1}, \dots, x_{k}) \in M^{k} \mid \forall i \neq j, \; x_{i} \neq x_{j} \} \]
  \end{block}

  \inklude{.7}{conf}

  \pause
  \begin{exampleblock}{Idea}
    Use ``formality of the little disks operads'' = results for $\Conf_{k}(\R^{n})$.
  \end{exampleblock}
\end{frame}

\begin{frame}
  \frametitle{Plan}
\tableofcontents
\end{frame}

\section{Little Disks Operads}

\begin{frame}
  \frametitle{Little Disks Operads}
  Boardmann--Vogt, May (70's): \alert{little disks operads $\DD_{n} \uncover<2->{= \{ \DD_{n}(r) \}_{r \geq 0}}$}
  \[
    \uncover<2->{
      \begin{tikzpicture}[scale=0.8, baseline]
        \filldraw[fill=gray!40] (0,0) circle (2);
        \filldraw[fill=gray!60] (0.7,0.3) circle (0.8) node {$1$};
        \filldraw[fill=gray!60] (-0.6,-0.6) circle (0.7) node {$2$};
        \filldraw[fill=gray!60] (-0.7,0.6) circle (0.4) node {$3$};
      \end{tikzpicture}
      \in \DD_{2}(3)
    }
    \uncover<3->{
      \mapsto
      \begin{tikzpicture}[scale=0.8, baseline]
        \filldraw[fill=gray!40] (0,0) circle (2);
        \draw[dotted] (0.7,0.3) circle (0.8) node [solid, pt, label={\tiny 1}] {};
        \draw[dotted] (-0.6,-0.6) circle (0.7) node [solid, pt, label={\tiny 2}] {};
        \draw[dotted] (-0.7,0.6) circle (0.4) node [solid, pt, label={\tiny 3}] {};
      \end{tikzpicture}
      \in \Conf_{3}(\R^{2})
    }
  \]
\end{frame}

\begin{frame}
  \frametitle{New structure: insertion}
  One can insert a configuration into a disk:
    \[
      \begin{tikzpicture}[scale=0.8, baseline]
        \filldraw[fill=gray!40] (0,0) circle (2);
        \filldraw[fill=gray!60] (0.7,0.3) circle (0.8) node {$1$};
        \filldraw[red,fill=gray!60] (-0.6,-0.6) circle (0.7) node{$2$};
      \end{tikzpicture}
      \circ_{2}
      \begin{tikzpicture}[scale=0.8, baseline]
        \filldraw[red,fill=gray!40] (0,0) circle (2);
        \filldraw[fill=blue!40] (-0.4,0.6) circle (0.9) node{$1$};
        \filldraw[fill=blue!40] (1, -0.7) circle (0.7) node{$2$};
      \end{tikzpicture}
      =
      \begin{tikzpicture}[scale=0.8, baseline]
        \filldraw[fill=gray!40] (0,0) circle (2);
        \filldraw[fill=gray!60] (0.7,0.3) circle (0.8) node {$1$};
        \filldraw[dashed, red, fill=gray!50] (-0.6,-0.6) circle (0.7);
        \begin{scope}[shift = {(-0.6,-0.6)}, scale = 0.35]
          \filldraw[fill=blue!40] (-0.4,0.6) circle (0.9) node{$2$};
          \filldraw[fill=blue!40] (1, -0.7) circle (0.7) node{$3$};
        \end{scope}
      \end{tikzpicture}
    \]

    \uncover<2->{$\implies$ \alert{operad structure}, cannot be seen on $\Conf_{\bullet}(\R^{n})$}
 \end{frame}

 \begin{frame}
   \frametitle{Configuration spaces of manifolds}
   If $M$ is ``framed'':
   \[ \alert{\DD_{M}(k)} \coloneqq \Emb^{\mathrm{fr}} \bigl( \mathbb{D}^{n} \sqcup \dots \sqcup \mathbb{D}^{n}, M \bigr) \uncover<2->{\qiso \Conf_{k}(M) } \]

   \inklude{.8}{emb_mfd}

   \pause[3]
   \vspace{-0.7cm}
   $\implies \DD_{M}$ is a ``right module'' over $\DD_{n}$

   \begin{block}{Idea}
     Use this extra structure to study $\Conf_{k}(M)$.
   \end{block}
\end{frame}

\begin{frame}
  \frametitle{Algebras over $\DD_{n}$ in the topological world}
  An \alert{algebra} over $\DD_{n}$ is a space on which $\DD_{n}$ ``acts'':
  \[ \DD_{n}(k) \times X^{k} \to X \]

  \pause
  \begin{theorem}[Boardmann--Vogt, May 1972]
    \begin{itemize}
    \item If $X = \Omega^{n} Y$, then $\DD_{n}$ acts on $X$;
    \item if $\DD_{n}$ acts on $X$ (+ grouplike), then $X \simeq \Omega^{n} Y$ for some $Y$.
  \end{itemize}
  \end{theorem}
\end{frame}

\begin{frame}
  \frametitle{Algebraic world}
  Operad $\DD_{n} \mapsto$ homology $H_{*}(\DD_{n})$

  \pause
  \begin{theorem}[Cohen 1976]
    An algebra over $H_{*}(\DD_{n})$ is:
    \begin{itemize}
    \item<2-> an associative algebra $(A, \cdot)$ for $n = 1$;
    \item<3-> an $n$-Gerstenhaber algebra $(B, \wedge, [,])$ for $n \geq 2$.
    \end{itemize}
  \end{theorem}

  \begin{columns}
    \begin{column}{0.5\linewidth}
      \pause[2]
      Associativity for $n \geq 1$:
      \begin{center}
        \begin{tikzpicture}[xscale=1.2, yscale=0.8]
          \draw[|-|] (0,0) to[edge label=1] (1,0);
          \draw[|-|] (1,0) to[edge label=2] (1.5,0);
          \draw[|-|] (1.5,0) to[edge label=3] (2,0);
          \node at (1,1) {$\simeq$};
          \draw[|-|] (0,2) to[edge label=1] (0.5,2);
          \draw[|-|] (0.5,2) to[edge label=2] (1,2);
          \draw[|-|] (1,2) to[edge label=3] (2,2);
        \end{tikzpicture}
      \end{center}
    \end{column}
    \begin{column}{0.5\linewidth}
      \pause[3]
      Commutativity for $n \geq 2$:
      \inklude{.5}{braid_d2}
    \end{column}
  \end{columns}
\end{frame}

\section{Swiss-Cheese Operad and Drinfeld Center}

\begin{frame}
  \frametitle{Categorical world}
  Operad $\DD_{n} \mapsto$ fundamental groupoid $\pi \DD_{n}$

  \begin{block}{Proposition}
    For $n \in \{ 1, 2 \}$, no loss of information: $\DD_{n} \qiso \operatorname{B} (\pi \DD_{n})$.
  \end{block}

  \pause
  \begin{theorem}[Tamarkin, Fresse]
    $\pi \DD_{n} \simeq$ operad whose algebras are:
    \begin{itemize}
    \item<2-> monoidal categories $(\cM, \otimes)$ for $n = 1$;
    \item<3-> braided monoidal categories $(\cN, \otimes, \tau)$ for $n = 2$.
    \end{itemize}
  \end{theorem}
\end{frame}

\begin{frame}
  \frametitle{Swiss-Cheese operad}
  Swiss-Cheese operad $\SC$: ``$\DD_{2}$-algebras acting on $\DD_{1}$-algebras''
  \pause
  \begin{equation*}
    \begin{tikzpicture}[scale=0.85, baseline=5mm]
      \filldraw[fill=gray!40] (-2,0) -- (2,0) arc (0:180:2);
      \filldraw[fill=gray!60] (-1.5,0) -- (0,0) node[above,midway]{$1$} arc (0:180:0.75);
      \filldraw[fill=gray!60] (0.5,0) -- (1.5,0) node[above,midway]{$2$} arc (0:180:0.5);
      \filldraw[red,fill=gray!60] (0.4,1.4) circle (0.5) node{$3$};
      \draw (0,0) node[below] {$\SC^{\col o}(2,1)$};
    \end{tikzpicture}
    \circ_{3}
    \begin{tikzpicture}[scale=0.4, baseline=-2.5mm]
      \filldraw[red,fill=gray!40] (0,0) circle (2);
      \filldraw[fill=blue!40] (-0.4,0.6) circle (0.9) node{$1$};
      \filldraw[fill=blue!40] (1, -0.7) circle (0.7) node{$2$};
      \draw (0,-2) node[below] {$\SC^{\col c}(0,2) = \DD_{2}(2)$};
    \end{tikzpicture}
    =
    \begin{tikzpicture}[scale=0.85, baseline=5mm]
      \filldraw[fill=gray!40] (-2,0) -- (2,0) arc (0:180:2);
      \filldraw[fill=gray!60] (-1.5,0) -- (0,0) node[above,midway]{$1$} arc (0:180:0.75);
      \filldraw[fill=gray!60] (0.5,0) -- (1.5,0) node[above,midway]{$2$} arc (0:180:0.5);
      \filldraw[red,dashed,fill=gray!60] (0.4,1.4) circle (0.5);
      \begin{scope}[shift={(0.4,1.4)}, scale=0.25]
        \filldraw[fill=blue!40] (-0.4,0.6) circle (0.9) node{$\scriptstyle 3$};
        \filldraw[fill=blue!40] (1, -0.7) circle (0.7) node{$\scriptstyle 4$};
      \end{scope}
      \draw (0,0) node[below] {$\SC^{\col o}(2,2)$};
    \end{tikzpicture}
  \end{equation*}
  \pause
  \begin{equation*}
    \begin{tikzpicture}[scale=0.85, baseline=5mm]
      \filldraw[fill=gray!40] (-2,0) -- (2,0) arc (0:180:2);
      \filldraw[red,fill=gray!60] (-1.5,0) -- (0,0) node[above,midway]{$1$} arc (0:180:0.75);
      \filldraw[fill=gray!60] (0.5,0) -- (1.5,0) node[above,midway]{$2$} arc (0:180:0.5);
      \filldraw[fill=gray!60] (0.4,1.4) circle (0.5) node{$3$};
      \draw (0,0) node[below] {$\SC^{\col o}(2,1)$};
    \end{tikzpicture}
    \circ_{1}
    \begin{tikzpicture}[scale=0.85, baseline=5mm]
      \fill[fill=gray!40] (-2,0) -- (2,0) arc (0:180:2);
      \draw[red] (-2,0) -- (2,0) arc (0:180:2);
      \filldraw[fill=blue!40] (-1,1) circle (0.55) node{$1$};
      \draw (0,0) node[below] {$\SC^{\col o}(0,1)$};
    \end{tikzpicture}
    =
    \begin{tikzpicture}[scale=0.85, baseline=5mm]
      \filldraw[fill=gray!40] (-2,0) -- (2,0) arc (0:180:2);
      \begin{scope}[shift={(-.75,0)}, scale=0.375]
        \filldraw[fill=blue!40] (-1,1) circle (0.55) node{$\scriptstyle 1$};
      \end{scope}
      \draw[red,dashed] (-1.5,0) -- (0,0) arc (0:180:0.75);
      \filldraw[fill=gray!60] (0.5,0) -- (1.5,0) node[above,midway]{$2$} arc (0:180:0.5);
      \filldraw[fill=gray!60] (0.4,1.4) circle (0.5) node{$3$};
      \draw (0,0) node[below] {$\SC^{\col o}(1,2)$};
    \end{tikzpicture}
  \end{equation*}
\end{frame}

\begin{frame}
  \frametitle{Homology vs fundamental groupoid of $\SC$}
  \begin{theorem}[Voronov 1999, Hoefel 2009]
    An algebra over $H_{*}(\SC)$ is a triplet $(A, B, f)$ where:
    \begin{itemize}
    \item $(A, \cdot)$ is an associative algebra;
    \item $(B, \wedge, [,])$ is a Gerstenhaber algebra;
    \item $f : B \to Z(A)$ is a central morphism of algebras.
    \end{itemize}
  \end{theorem}

  \pause
  \begin{theorem}
    $\pi\SC \simeq$ an operad whose algebras are triplets $(\cM, \cN, F)$ where:
    \begin{itemize}
    \item $(\cM, \otimes)$ is a monoidal category;
    \item $(\cN, \otimes, \tau)$ is a braided monoidal category;
    \item $F : \cN \to \Zd(\cM)$ is a braided functor to the ``Drinfeld center''
    \end{itemize}
  \end{theorem}
\end{frame}

\begin{frame}
  \frametitle{Recap}
  \begin{tabular}{rlll}
    \toprule
    {}
    & \tikz[remember picture, baseline]{\node (x1) {Topological};}
    & \tikz[remember picture, baseline]{\node (x2) {Algebraical $H_{*}(-)$};}
    & \tikz[remember picture, baseline]{\node (x3) {Categorical $\pi(-)$};} \\
    \midrule
    $\DD_{1}$
    & \raisebox{-.5\height}{
      \begin{tikzpicture}
        \draw[|-|] (0,0) to[edge label=1] (1,0);
        \draw[-|] (1,0) to[edge label=3] (1.5,0);
        \draw[-|] (1.5, 0) to[edge label=2] (2,0);
      \end{tikzpicture}
      }
    & associative $(A, \cdot)$
    & monoidal $(\cM, \otimes)$ \\
    \addlinespace[1mm]
    $\DD_{2}$
    & \raisebox{-.5\height}{\tikz[scale=0.8]{\filldraw[fill=gray!40] (0,0) circle (1); \filldraw[fill=gray!60] (0.3,0.5) circle (0.4) node{1}; \filldraw[fill=gray!60] (-0.2,-0.6) circle (0.3) node{2};}}
    & Gerstenhaber $(B, \wedge, [,])$
    & braided $(\cN, \otimes, \tau)$ \\
    \addlinespace[1mm]
    $\SC$
    & \raisebox{-.5\height}{\tikz{\filldraw[fill=gray!40] (0,0) -- (2,0) arc (0:180:1); \filldraw[fill=gray!60] (1,0) -- (2,0) node[above,midway] {\tiny 1} arc (0:180:0.5); \filldraw[fill=gray!60] (0.2,0) -- (0.7,0) node[above,midway] {\tiny 2} arc (0:180:0.25); \filldraw[fill=gray!60] (0.8,0.7) circle (0.25) node{\tiny 1};}}
    & $(B, \wedge, [,]) \xrightarrow{f} Z(A, \cdot)$
    & $(\cN, \otimes, \tau) \xrightarrow{F} \Zd(\cM, \otimes)$ \\
    \bottomrule
  \end{tabular}

  \uncover<2->{%
    \begin{tikzpicture}[remember picture, overlay]
      \draw[-implies, double equal sign distance] (x1) to (x2);
      \draw[implies-implies, double equal sign distance] (x1.north east) to[bend left] (x3);
      \draw[-implies, double equal sign distance] (x3) to (x2);
    \end{tikzpicture}
  }
  \begin{block}{Remark}<3>
    I also build a model $\PaPCD^{\phi} = \text{``}\PaP \rtimes_{\phi} \CDh_{+}\text{''}$ out of a Drinfeld associator $\phi$, following Tamarkin's proof of the formality of $\DD_{2}$.
  \end{block}
\end{frame}

\section{The Lambrechts--Stanley Model of Configuration Spaces}

\begin{frame}
  \frametitle{Models}
  We are interested in rational/real models
  \[ A \simeq \Omega^{*}(M) \text{ ``forms on } M \text{'' \tiny (e.g.\ de Rham, piecewise polynomial...)} \]
  where $A$ is an ``explicit'' CDGA {\tiny (= Commutative Differential Graded Algebra)}

  ~

  \pause $M$ nilpotent of finite type $\implies A$ contains all the rational/real homotopy type of $M$

  ~

  \pause We're looking for a CDGA $\simeq \Omega^{*}(\Conf_{k}(M))$ built from $A$
\end{frame}

\begin{frame}
  \frametitle{Formality of $\Conf_{k}(\R^n)$}
  $\Conf_{k}(\R^{n})$ is a formal space, i.e. [Kontsevich]:
  \[H^{*}(\Conf_{k}(\R^{n})) \simeq \Omega^{*}(\Conf_{k}(\R^{n}))\]
  completely determines the rational homotopy type of $\Conf_{k}(\R^{n})$

  \pause
  \begin{theorem}[Arnold 1969, Cohen 1976]
    \begin{itemize}
    \item $H^{*}(\Conf_{k}(\R^{n})) = S(\omega_{ij})_{1 \leq i \neq j \leq k} / I$
    \item $\deg \omega_{ij} = n-1$
    \item $I = \bigl( \omega_{ji} = \pm \omega_{ij},\; \omega_{ij}^{2} = 0,\; \omega_{ij} \omega_{jk} + \omega_{jk} \omega_{ki} + \omega_{ki} \omega_{ij} = 0 \bigr)$
    \end{itemize}
  \end{theorem}
\end{frame}

\begin{frame}
  \frametitle{Poincaré duality models}
  \alert<1-3>{Poincaré duality CDGA} $(A, \varepsilon)$ {\tiny \hfill (example: $M$ is closed \& oriented)}
  \begin{itemize}[<+->]
  \item $A$: finite type connected CDGA; {\tiny \hfill (e.g.\ $(H^{*}(M), d=0)$)}
  \item $\varepsilon : A^{n} \to \Bbbk$ such that $\varepsilon \circ d = 0$; {\tiny \hfill (e.g.\ $\int_{M} (-)$)}
  \item $A^{k} \otimes A^{n-k} \to \Bbbk, \, a \otimes b \mapsto \varepsilon(ab)$ non degenerate. {\tiny \hfill (e.g.\ $H^{k}(M) \otimes H^{n-k}(M) \to \Bbbk$)}
  \end{itemize}

  \pause[\thebeamerpauses]
  \begin{columns}
    \begin{column}{0.6\linewidth}
      \begin{theorem}[Lambrechts--Stanley 2004]
        Any \alert<3>{simply connected} manifold has such a model
      \end{theorem}
    \end{column}
    \begin{column}{0.4\linewidth}
      \[ \begin{tikzcd}[ampersand replacement=\&, row sep = tiny, column sep = small]
          \Omega^{*}(M) \ar[dr, "\int_{M}" swap] \& \cdot \ar[l, "\sim" swap] \ar[r, "\sim"] \& \exists A \ar[dl, "\exists \varepsilon"] \\
          {} \& \Bbbk
        \end{tikzcd} \]
    \end{column}
  \end{columns}

  \pause
  \begin{block}{Remark}
    By a result of Longoni--Salvatore (2005), $\exists$ non simply-connected $L \simeq L'$ but $\Conf_{k}(L) \not\simeq \Conf_{k}(L')$
  \end{block}
\end{frame}

\begin{frame}
  \frametitle{The Lambrechts--Stanley model}
  $\alert{\GG{A}(k)}$ conjectured model of $\Conf_{k}(M) = M^{\times k} \setminus \bigcup_{i \neq j} \tikzmark{delta}\Delta_{ij}$
  \tikz[remember picture, overlay]
    \draw[->] ($(pic cs:delta)+(0.2,-0.2)$) to[bend right] +(0.5,-0.3) node[right] {\scriptsize $\coloneqq \{x_{i} = x_{j}\}$};
  \begin{itemize}
  \item<2-> ``Generators'': $A^{\otimes k} \otimes S(\omega_{ij})_{1 \leq i \neq j \leq k}$
  \item<3-> Relations:
    \begin{itemize}
    \item Arnold relations \hfill{} {\tiny ($\omega_{ji} = \pm \omega_{ij}, \omega_{ij}^{2} = \omega_{ij} \omega_{jk} + \omega_{jk} \omega_{ki} + \omega_{ki} \omega_{ij} = 0$)}
    \item $p_{i}^{*}(a) \cdot \omega_{ij} = p_{j}^{*}(a) \cdot \omega_{ij}$. \hfill{} {\tiny $(p_{i}^{*}(a) = 1 \otimes \dots \otimes 1 \otimes a \otimes 1 \otimes \dots \otimes 1)$}
    \end{itemize}
  \item<4-> $d \omega_{ij} = (p_{i}^{*} \cdot p_{j}^{*})(\Delta_{A})$ kills the dual of $[ \Delta_{ij} ]$.
  \end{itemize}

  \pause[5]
  \begin{theorem}[Lambrechts--Stanley 2008]
    \[ \dim_{\Q} H^{i}(\Conf_{k}(M)) = \dim_{\Q} H^{i}(\GG{A}(k)) \]
  \end{theorem}
\end{frame}

\begin{frame}
  \frametitle{First part of the theorem}
  $\GG{A}(k)$ was known to be a rational model of $\Conf_{k}(M)$ in a few cases:
  \begin{itemize}
  \item $M$ smooth projective complex variety [Kriz];
  \item $k = 2$ and $M$ is $2$-connected [Lambrechts--Stanley];
  \item $k = 2$ and $\dim M$ is even [Cordova Bulens]\dots
\end{itemize}


  \pause
  \begin{theorem}
    Let $M$ be a smooth, closed, simply connected manifold \alert{of dimension $\geq 4$}. Then $\GG{A}(k)$ is a model \alert{over $\R$} of $\Conf_{k}(M)$ for all $k \geq 0$.
  \end{theorem}

  \pause
  \begin{block}{Corollary}
    The real homotopy type of $\Conf_{k}(M)$ only depends on the real homotopy type of $M$:
    \[ M \simeq_{\R} N \implies \Conf_{k}(M) \simeq_{\R} \Conf_{k}(N). \]
  \end{block}
\end{frame}

\begin{frame}
  \frametitle{Operads}
  \begin{block}{Ideas \& Goals}
    Adapt the construction for $\DD_{n}$ \& keep track of the $\DD_{n}$-action whenever it exists
  \end{block}

  \pause
  Fulton--MacPherson compactification $\Conf_{k}(M) \xhookrightarrow{\sim} \alert{\FM_{M}(k)}$
  \begin{center}
    \tikzset{every picture/.append style={scale=0.9}}
    \begin{tikzpicture}
  \draw \mytorus;
  \draw[xscale=-1] \mytorus;
  \draw[rotate=180] \mytorus;
  \draw[yscale=-1] \mytorus;
  \draw (-2,.2) .. controls (-1.5,-0.3) and (-1,-0.5) .. (0,-.5) .. controls (1,-0.5) and (1.5,-0.3) .. (2,0.2);
  \draw (-1.75,0) .. controls (-1.5,0.3) and (-1,0.5) .. (0,.5) .. controls (1,0.5) and (1.5,0.3) .. (1.75,0);

  \filldraw [densely dotted, fill = gray!20] (-3.5, 1) to[bend left] (-2.5,-0.5) to[bend left] (-1.5,1);
  \filldraw [fill = gray!20] (-2.5, 1) ellipse (1 and 0.5);
  \draw (-3,0.9) node[pt, label = {\tiny 1}] {};
  \draw (-2, 1.1) node[pt, label = {\tiny 7}] {};

  \draw (-1,-1.5) node[pt, label = 2] {};

  \filldraw [densely dotted, fill = gray!20] (0.8, 2) to[bend left] (2.3,0) to[bend left] (3.8,2);
  \filldraw [fill = gray!20] (2.3,2) ellipse (1.5 and 0.75);

  \draw (1.3,1.6) node[pt, label = 3] {};
  \draw (3.3,1.9) node[pt, label = 4] {};

  \filldraw [densely dotted, fill = gray!40] (1.3,3) to[bend left] (2.2,1.7) to[bend left] (3.3,3);
  \filldraw [fill = gray!40] (2.3,3) ellipse (1 and 0.5);
  \draw (1.8,3) node[pt, label = {\tiny 5}] {};
  \draw (2.8,3) node[pt, label = {\tiny 6}] {};
\end{tikzpicture}

  \end{center}
\end{frame}

\begin{frame}
  \frametitle{Understanding $\FM_M$ (\#1)}
  \begin{center}
    \animategraphics[width=0.8\linewidth]{10}{fig/anim}{1}{16}
  \end{center}
\end{frame}

\begin{frame}
  \frametitle{Understanding $\FM_M$ (\#2)}
  \begin{center}
    \animategraphics[width=0.7\linewidth]{10}{fig/anim}{17}{33}
  \end{center}
\end{frame}

\begin{frame}
  \frametitle{Understanding $\FM_M$ (\#3)}
  \begin{center}
    \animategraphics[width=0.7\linewidth]{10}{fig/anim}{34}{51}
  \end{center}
\end{frame}

\begin{frame}
  \frametitle{Compactifying $\Conf_{k}(\R^{n})$}
  Can also compactify $\Conf_{k}(\R^{n}) \xrightarrow{\sim} \only<2->{\Conf_{k}(\R^{n}) / (\R^{n} \rtimes \R_{+}^{*}) \xhookrightarrow{\sim}} \alert{\FM_{n}(k)}$
  \begin{center}
    \begin{tikzpicture}[font = \tiny, scale = 1.5]
  \filldraw [fill = gray!20] (0,0) ellipse (2 and 1);

  \filldraw [densely dotted, fill = gray!40] (1.8, 1) to[bend right] (1,0) to[bend right] (0.2, 1);
  \filldraw [fill = gray!40] (1,1) ellipse (0.8 and 0.4);

  \filldraw [densely dotted, fill = gray!40] (-1.8, 1) to[bend left] (-1,0) to[bend left] (-0.2, 1);
  \filldraw [fill = gray!40] (-1,1) ellipse (0.8 and 0.5);

  \filldraw [densely dotted, fill = gray!60] (1.9, 1.5) to[bend right] (1.5, 1) to[bend
  right] (1.1, 1.5);
  \filldraw [fill=gray!60] (1.5, 1.5) ellipse (0.4 and 0.2);

  \draw (0.3, -0.4) node [pt, label = 1] {};
  \draw (-0.5, 0.2) node [pt, label = 2] {};
  \draw (-1.25, 1) node [pt, label = 3] {};
  \draw (0.8, 0.9) node [pt, label = 4] {};
  \draw (-0.7, 0.9) node [pt, label = 5] {};
  \draw (1.7, 1.42) node [pt, label = 6] {};
  \draw (1.5, 1.37) node [pt, label = 7] {};
  \draw (-1, 1.2) node [pt, label = 8] {};
\end{tikzpicture}


    \uncover<2->{\scriptsize (+ normalization to deal with $\R^{n}$ being noncompact)}
  \end{center}
\end{frame}

\begin{frame}
  \frametitle{Operads}
  $\FM_{n} = \{ \FM_{n}(k) \}_{k \geq 0}$ is an operad $\simeq \DD_{n}$

  \[
  \begin{gathered}
    \begin{tikzpicture}[scale = 0.7]
      \filldraw [fill = gray!20] (0,0) ellipse (2 and 1);
      \draw (-1,0) node [pt, label = 1] {};
      \draw (1,0) node [pt, fill = blue, label = {\color{blue} 2}] {};
    \end{tikzpicture}
  \end{gathered}
  \circ_{\color{blue} 2}
  \begin{gathered}
    \begin{tikzpicture}[scale = 0.7]
      \filldraw [fill = white!50!blue] (0,0) ellipse (2 and 1);
      \draw (-1,0) node [pt, label = 1] {};
      \draw (1,0) node [pt, label = 2] {};
    \end{tikzpicture}
  \end{gathered}
  =
  \begin{gathered}
    \begin{tikzpicture}[scale = 0.7]
      \filldraw [fill = gray!20] (0,0) ellipse (2 and 1);
      \draw (-1,0) node [pt, label = 1] {};
      \filldraw [densely dotted, fill = white!50!blue] (2, 1.5) to[bend right] (1,0) to[bend right] (0, 1.5);
      \filldraw [fill = white!60!blue] (1,1.5) ellipse (1 and 0.6);
      \draw (0.5,1.5) node [pt, label = {\tiny 2}] {};
      \draw (1.5,1.5) node [pt, label = {\tiny 3}] {};
    \end{tikzpicture}
  \end{gathered}
\]

\[ \FM_{n}(k) \times \FM_{n}(l) \xrightarrow{\circ_{i}} \FM_{n}(k+l-1), \quad 1 \leq i \leq k \]

\end{frame}

\begin{frame}
  \frametitle{Modules over operads}
  $M$ framed $\implies \FM_{M} = \{\FM_{M}(k)\}_{k \geq 0}$ is a right $\FM_{n}$-module $\simeq \DD_{M}$

  \[
  \begin{gathered}
    \begin{tikzpicture}[scale = 0.5]
      \draw \mytorus;
      \draw[xscale=-1] \mytorus;
      \draw[rotate=180] \mytorus;
      \draw[yscale=-1] \mytorus;
      \draw (-2,.2) .. controls (-1.5,-0.3) and (-1,-0.5) .. (0,-.5) .. controls (1,-0.5) and (1.5,-0.3) .. (2,0.2);
      \draw (-1.75,0) .. controls (-1.5,0.3) and (-1,0.5) .. (0,.5) .. controls (1,0.5) and (1.5,0.3) .. (1.75,0);

      \draw (-2.5,0.5) node[pt, label = {\tiny 1}] {};
      \draw (-1,-1.5) node[pt, label = {\tiny 2}] {};
      \draw (2,1) node[pt, fill = blue, label = {\color{blue} \tiny 3}] {};
    \end{tikzpicture}
  \end{gathered}
  \circ_{\color{blue} 3}
  \begin{gathered}
    \begin{tikzpicture}[scale = 0.6]
      \filldraw[fill = white!50!blue] (2,3) ellipse (2 and 1);
      \draw (1,2.5) node[pt, label = {\tiny 1}] {};
      \draw (3,3.3) node[pt, label = {\tiny 2}] {};
      \draw (2,2.7) node[pt, label = {\tiny 3}] {};
    \end{tikzpicture}
  \end{gathered}
  =
  \begin{gathered}
    \begin{tikzpicture}[scale = 0.5]
      \draw \mytorus;
      \draw[xscale=-1] \mytorus;
      \draw[rotate=180] \mytorus;
      \draw[yscale=-1] \mytorus;
      \draw (-2,.2) .. controls (-1.5,-0.3) and (-1,-0.5) .. (0,-.5) .. controls (1,-0.5) and (1.5,-0.3) .. (2,0.2);
      \draw (-1.75,0) .. controls (-1.5,0.3) and (-1,0.5) .. (0,.5) .. controls (1,0.5) and (1.5,0.3) .. (1.75,0);

      \draw (-2.5,0.5) node[pt, label = {\tiny 1}] {};
      \draw (-1,-1.5) node[pt, label = {\tiny 2}] {};

      \filldraw [densely dotted, fill = white!50!blue] (0, 3) to[bend left] (2,1) to[bend left] (4,3);
      \filldraw[fill = white!60!blue] (2,3) ellipse (2 and 1);

      \draw (1,2.5) node[pt, label = {\tiny 3}] {};
      \draw (3,3.3) node[pt, label = {\tiny 4}] {};
      \draw (2,2.7) node[pt, label = {\tiny 5}] {};
    \end{tikzpicture}
  \end{gathered}
\]

\[ \FM_{M}(k) \times \FM_{\R^{n}}(l) \xrightarrow{\circ_{i}} \FM_{M}(k+l-1), \quad 1 \leq i \leq k \]

\end{frame}

\begin{frame}
  \frametitle{Cohomology of $\FM_n$ and coaction on $\GG{A}$}
  $H^{*}(\FM_{n})$ inherits a Hopf cooperad structure

  \pause One can rewrite:
  \[ \GG{A}(k) = (A^{\otimes k} \otimes H^{*}(\FM_{n}(k)) / \text{relations}, d) \]

  \pause
  \begin{block}{Proposition}
    $\chi(M) = 0 \implies \GG{A} = \{ \GG{A}(k) \}_{k \geq 0}$ is a Hopf right $H^{*}(\FM_{n})$-comodule
  \end{block}
\end{frame}

\begin{frame}
  \frametitle{Motivation}
  We are looking for something to put here:
  \[ \GG{A}(k) \qiso* \alert<1>{?} \qiso \Omega^{*}(\FM_{M}(k)) \]

  \pause If true, then hopefully it fits in a diagram like this:
  \[ \begin{tikzcd}[ampersand replacement=\&, row sep = 0, remember picture]
      \GG{A} \& \ar[l, "\sim" swap] \alert<2>{?} \ar[r, "\sim"] \& \Omega^{*}(\FM_{M}) \\
      \circlearrowleft \& \alert<2>{\circlearrowleft} \& \circlearrowleft \\
      |[alias=y1]|
      H^{*}(\FM_{n}) \&
      \ar[l, "\sim" swap] \alert<2>{?} \ar[r, "\sim"] \&
      |[alias=y2]|
      \Omega^{*}(\FM_{n})
    \end{tikzcd} \]

  \pause
  \begin{tikzpicture}[remember picture, overlay]
    \node[draw,rectangle,fit=(y1.north west) (y2.south east), inner sep=0pt] (box) {};
    \draw[->] (box.south) -- ++(0,-0.5) node[below] {Already known: \alert<3>{formality of the little disks operads}};
  \end{tikzpicture}
\end{frame}

\begin{frame}
  \frametitle{Kontsevich's graph complexes}
  [Kontsevich] Hopf cooperad $\Graphs_{n} = \{ \Graphs_{n}(k) \}_{k \geq 0}$
  \[
    d \left(
      \begin{tikzpicture}[scale = 0.5, baseline]
        \node[intv] (int) {};
        \node[extv] (1) [above = 0.3cm of int] {$1$};
        \node[extv] (2) [below left = 0.4cm of int] {$2$};
        \node[extv] (3) [below right = 0.4cm of int] {$3$};
        \draw (1) -- (int); \draw (2) -- (int); \draw (3) -- (int);
      \end{tikzpicture}
    \right) =
    \pm
    \begin{tikzpicture}[scale = 0.5, baseline]
      \node (int) {};
      \node[extv] (1) [above = 0.3cm of int] {$1$};
      \node[extv] (2) [below left = 0.4cm of int] {$2$};
      \node[extv] (3) [below right = 0.4cm of int] {$3$};
      \draw (1) -- (2); \draw (2) -- (3);
    \end{tikzpicture}
    \pm
    \begin{tikzpicture}[scale = 0.5, baseline]
      \node (int) {};
      \node[extv] (1) [above = 0.3cm of int] {$1$};
      \node[extv] (2) [below left = 0.4cm of int] {$2$};
      \node[extv] (3) [below right = 0.4cm of int] {$3$};
      \draw (2) -- (3); \draw (3) -- (1);
    \end{tikzpicture}
    \pm
    \begin{tikzpicture}[scale = 0.5, baseline]
      \node (int) {};
      \node[extv] (1) [above = 0.3cm of int] {$1$};
      \node[extv] (2) [below left = 0.4cm of int] {$2$};
      \node[extv] (3) [below right = 0.4cm of int] {$3$};
      \draw (3) -- (1); \draw (1) -- (2);
    \end{tikzpicture}
  \]

  \uncover<2->{
    \begin{theorem}[Kontsevich 1999, Lambrechts--Volić 2014]
      \begin{equation*}
        \begin{tikzcd}[ampersand replacement=\&, row sep = tiny]
          H^{*}(\FM_{n}; \R) \&
          \Graphs_{n} \ar[l, "\sim" swap] \ar[r, "\sim"] \&
          \OmPA^{*}(\FM_{n}) \\
          \omega_{ij} \&
          \begin{tikzpicture}[scale=0.3]
            \node[circle, draw, minimum size = 0.5cm, inner sep=0pt] (1) {$i$};
            \node[circle, draw, minimum size = 0.5cm, inner sep=0pt] (2) [right = 0.5cm of 1] {$j$};
            \draw (1) -- (2);
          \end{tikzpicture}
          \ar[l, mapsto] \ar[r, mapsto] \&
          \text{explicit representatives}\\
          0 \&
          \tikz[scale=0.5]{\node[intv] (1) {};}
          \ar[l, mapsto] \ar[r, mapsto] \&
          \text{``explicit'' integrals}
        \end{tikzcd}
      \end{equation*}
    \end{theorem}
  }
\end{frame}

\begin{frame}
  \frametitle{Complete version of the theorem}
  \begin{block}{Idea}
    Build $\Graphs^{\ze}_{R}$ from $\Graphs_{n}$ similar to how $\GG{A}$ is built from $H^{*}(\FM_{n})$
  \end{block}

  \pause
  \begin{theorem}[Complete version]
    $M$: closed, simply connected, smooth manifold with $\dim \ge 4$
    \[ \begin{tikzcd}[ampersand replacement=\&, row sep = 0]
        \GG{A} \& \ar[l, "\sim" swap] \Graphs^{\ze}_{R} \ar[r, dashed, "\sim"] \& \OmPA^{*}(\FM_{M}) \\
        \circlearrowleft^{\dagger} \& \circlearrowleft^{\dagger} \& \circlearrowleft^{\ddagger} \\
        H^{*}(\FM_{n}) \& \ar[l, "\sim" swap] \Graphs_{n} \ar[r, "\sim"] \& \OmPA^{*}(\FM_{n})
      \end{tikzcd} \]
    $^{\dagger}$ When $\chi(M) = 0$ \\
    $^{\ddagger}$ When $M$ is framed \hfill $A \qiso* R \qiso \OmPA^{*}(M)$
  \end{theorem}
\end{frame}

\section{Configuration Spaces of Manifolds with Boundary}

\begin{frame}
  \frametitle{Poincaré--Lefschetz duality models}
  Now: $\partial M \neq \varnothing \implies H^{*}(M) \cong H_{n-*}(M, \partial M)$ for $M$ oriented

  \pause \alert{Poincaré--Lefschetz duality pair} $(B \xrightarrow{\lambda} B_{\partial})$:
  \begin{itemize}[<+->]
  \item $(B_{\partial}, \varepsilon_{\partial})$ Poincaré duality CDGA of dimension $n-1$; {\tiny \hfill (models $\partial M$, $\int_{\partial M}$)}
  \item $B$: fin. type connected CDGA; {\tiny \hfill (models $M$)}
  \item $\lambda : B \twoheadrightarrow B_{\partial}$: surjective CDGA morphism; {\tiny \hfill (models $\partial M \hookrightarrow M$)}
  \item $\varepsilon : B^{n} \to \R$ s.t.\ $\varepsilon(dy) = \varepsilon_{\partial}(\lambda(y))$; {\tiny \hfill (models $\int_{M}(-)$ \& Stokes formula)}
  \item if $K = \ker \lambda$, then $\theta : B \to K^{\vee}[-n]$, $b \mapsto \varepsilon(b \cdot -)$ is a surjective quasi-isomorphism. {\tiny \hfill ($K \simeq \Omega^{*}(M, \partial M)$)}
  \end{itemize}

  \pause[\thebeamerpauses]
  In this case, $A \coloneqq B / \ker \theta$ is a model of $M$, and $\theta : A \xrightarrow{\cong} K^{\vee}[-n]$
\end{frame}
\begin{frame}
  \frametitle{Existence \& example of PLD models}
  \begin{exampleblock}{Example}
    If $M = N \setminus \{ * \}$ with $N$ closed:
    \pause
    take $P$ a Poincaré duality model of $N$
    \[ B = (P \oplus \R v_{n-1}, d v = \vol_{P}) \twoheadrightarrow B_{\partial} = H^{*}(S^{n-1}) = (\R \oplus \R v_{n-1}, d = 0) \]
  \end{exampleblock}

  \pause
  \begin{block}{Proposition}
    If $M$ is simply connected, $\partial M$ is simply connected, and $\dim M \geq 7$, then $(M, \partial M)$ admits a PLD model.
  \end{block}

  \pause
  \begin{block}{Remark}
    Also true if $M$ admits a ``surjective pretty model'', cf.\ theorems of Cordova Bulens and Cordova Bulens--Lambrechts--Stanley.
  \end{block}
\end{frame}

\begin{frame}
  \frametitle{The ``naïve'' dg-module $\GG{A}$}
  Given a PLD model $(B, B_{\partial})$ and $A = B / \ker \theta$, can build \alert<1>{$\GG{A}(k)$} as before.

  \pause
  \begin{theorem}
    \[ \dim H^{i}(\Conf_{k}(M)) = \dim H^{i}(\GG{A}(k)) \]
  \end{theorem}

  \pause
  \begin{block}{Idea of proof}
    Combine:
    \begin{itemize}
    \item Techniques of Lambrechts--Stanley to compute homology of spaces of the type $M^{k} \setminus \bigcup_{i \neq j} \Delta_{ij}$;
    \item Techniques of Cordova Bulens--L--S to compute homology of $M = N \setminus X$ where $N$ is a closed manifold and $X \subset N$ is a sub-polyhedron.
    \end{itemize}
  \end{block}
\end{frame}

\begin{frame}
  \frametitle{The actual model}
  In general, $\GG{A}(k)$ is not actually a CDGA model for $\Conf_{k}(M)$.

  \pause
  \begin{block}{Motivation}
    $M = S^{1} \times (0,1) \cong \R^{2} \setminus \{ 0 \} \implies \Conf_{2}(M) \simeq \Conf_{3}(\R^{2})$

    \pause
    Then $A = H^{*}(M) = \R \oplus \R \eta$.
    In $\GG{A}(2)$, relation $(1 \otimes \eta) \omega_{12} = (\eta \otimes 1) \omega_{12}$.
    \pause
    But in $\Conf_{3}(\R^{2})$, Arnold relation: $(1 \otimes \eta) \omega_{12} = (\eta \otimes 1) \omega_{12} \pm (\eta \otimes \eta)$.
  \end{block}

  \pause
  $\implies$ must define a ``perturbed model'' $\GGt{A}(k)$

  \begin{block}{Proposition}
    Isomorphism of dg-modules $\GG{A}(k) \cong \GGt{A}(k)$.
  \end{block}
\end{frame}

\begin{frame}
  \frametitle{Swiss-Cheese \& graphs}
  $M$ looks like $\mathbb{H}^{n}$ (locally) $\implies$ Swiss-Cheese operad

  \pause
  \begin{equation*}
    \begin{tikzpicture}[scale=0.85, baseline = 5mm]
      \filldraw[fill=gray!40] (-2,0) -- (2,0) arc (0:180:2);
      \filldraw[fill=gray!60] (-1.5,0) -- (0,0) node[above,midway]{$1$} arc (0:180:0.75);
      \filldraw[fill=gray!60] (0.5,0) -- (1.5,0) node[above,midway]{$2$} arc (0:180:0.5);
      \filldraw[fill=gray!60] (0.4,1.4) circle (0.5) node{$3$};
      \draw (0,0) node[below] {$\SC_{2}^{\col o}(2,1)$};
    \end{tikzpicture}
    \mapsto
    \begin{tikzpicture}[scale=0.85, baseline=5mm]
      \filldraw[fill=gray!40] (-2,0) -- (2,0) arc (0:180:2);
      \draw[dashed] (-1.5,0) -- (0,0) node[midway,solid,pt,label={\tiny 1}]{} arc (0:180:0.75);
      \draw[dashed] (0.5,0) -- (1.5,0) node[midway,solid,pt,label={\tiny 2}]{} arc (0:180:0.5);
      \draw[dashed] (0.4,1.4) circle (0.5) node[solid,pt,label={\tiny 3}]{};
      \draw (0,0) node[below] {$\Conf_{2,1}(\mathbb{H}^{2})$};
    \end{tikzpicture}
  \end{equation*}

  \pause
  \begin{theorem}[Willwacher 2015]
    Model $\SGraphs_{n}$ for $\SFM_{n} = \overline{\Conf_{\bullet,\bullet}(\mathbb{H}^{n})} \simeq \SC_{n}$:
    \[
  \begin{gathered}
    \begin{tikzpicture}[scale=0.3]
      \node (0) {};
      \node (o) [right = 3cm of 0] {};
      \draw[dotted] (0) to (o);
      \node[extv] (u) [right = 0.5cm of 0] {$1$};
      \node[intv] (i1) [right = 1cm of u] {};
      \node[extv] (v1) [above = 1cm of u] {$1$};
      \node[extv] (v2) [right = 1cm of v1] {$2$};
      \node[intv] (i2) [right = 1cm of v2] {};
      \draw[->] (v1) edge (u) edge (i1) edge (v2);
      \draw[->] (v1) to (v2);
      \draw[->] (i2) edge (v2) edge (i1);
    \end{tikzpicture}
  \end{gathered}
  \in
  \SGraphs_n(1,2)
\]

  \end{theorem}
\end{frame}

\begin{frame}
  \frametitle{Theorem for manifolds with boundary}
  Using similar techniques:
  \begin{theorem}
    For $M$ a smooth, compact manifold of dimension at least $\geq 7$, $M$ and $\partial M$ simply connected:
    \[ \begin{tikzcd}[ampersand replacement=\&, row sep = 0]
        \GGt{A} \& \ar[l, "\sim" swap] \Graphs^{\ze}_{R} \ar[r, dashed, "\sim"] \& \OmPA^{*}(\SFM_{M}(\varnothing, -)) \\
        \circlearrowleft \& \circlearrowleft \& \circlearrowleft^{\dagger} \\
        H^{*}(\FM_{n}) \& \ar[l, "\sim" swap] \Graphs_{n} \ar[r, "\sim"] \& \OmPA^{*}(\FM_{n})
      \end{tikzcd} \]

    Moreover: model $\SGraphs^{c_{M}, \szphi}_{R,R_{\partial}}(k,l)$ of $\SFM_{M}(k,l)$, compatible with the (co)action of $\SGraphs_{n}$ / $\SFM_{n}$
  \end{theorem}
\end{frame}

\begin{frame}[standout]
  \small Fin de la présentation
\end{frame}

\end{document}

% Local Variables:
% TeX-engine: luatex
% TeX-master: t
% End:
