\documentclass[12pt,bibliography=totoc,index=totoc]{scrbook}


\usepackage{polyglossia}
\setmainlanguage{english}
\setotherlanguage{french}
% https://github.com/reutenauer/polyglossia/issues/182
\makeatletter
\newcommand{\disfrpt}{\nofrench@punctuation}
\makeatother

\usepackage{amssymb, mathtools, mathrsfs}
\usepackage{graphicx}
\usepackage{xcolor}
\usepackage{array,tabularx,booktabs}
\usepackage{geometry}
\usepackage{import}
\usepackage{hyphenat}

\usepackage[final]{microtype}
\usepackage{unicode-math}
\defaultfontfeatures{Ligatures=TeX}
\setmainfont{TeX Gyre Pagella}
\setsansfont{Linux Biolinum O}[Scale=MatchLowercase]
\setmonofont{Inconsolata}[Scale=MatchLowercase]
\setmathfont{TeX Gyre Pagella Math}[Scale=MatchLowercase]
\setmathfont{Latin Modern Math}[range=\symcal,Scale=MatchLowercase]

\usepackage[autostyle]{csquotes}
\usepackage[style=alphabetic, sorting=nyt, maxnames=10]{biblatex}

\usepackage{makeidx}
\makeindex

\usepackage{tikz}
\usepackage{tikz-cd}
\usetikzlibrary{arrows,positioning,calc}
\usepackage{todonotes}

\usepackage[numbered]{bookmark}
\usepackage{hyperref}
\usepackage{amsthm, thmtools}
\usepackage[xspace]{ellipsis}
% so that links to figures go to the top of the figure
\usepackage[all]{hypcap}
\usepackage{wrapfig}

\hypersetup{
  pdfencoding=auto,      % https://tex.stackexchange.com/a/26193
  colorlinks,
  urlcolor = {blue!80!black},
  linkcolor = {red!70!black},
  citecolor = {green!50!black}
}
\setcounter{tocdepth}{2}

% bibliography
\addbibresource{thesis.bib}
% let "publisher" and "location" change place
\renewbibmacro*{publisher+location+date}{%
  \printlist{publisher}%
  \setunit*{\addcomma\space}%
  \printlist{location}%
  \setunit*{\addcomma\space}%
  \usebibmacro{date}%
  \newunit}
% numdam
\DeclareFieldFormat{eprint:numdam}{%
  \textsc{Numdam}\addcolon\space%
  \ifhyperref
  {\href{http://www.numdam.org/item?id=#1}{\nolinkurl{#1}}}
  {\nolinkurl{#1}}%
}

% Theorems
\numberwithin{figure}{section}
\numberwithin{equation}{section}
\NewDocumentCommand{\declthm}{m m O{plain}}{% env, name, style
  \declaretheorem[
  name = #2,
  sibling = equation,
  style = #3
  ]
  {#1}
}
\declthm{theorem}{Theorem}
\declthm{proposition}{Proposition}
\declthm{corollary}{Corollary}
\declthm{lemma}{Lemma}
\declthm{definition}{Definition}[definition]
\declthm{example}{Example}[remark]
\declthm{remark}{Remark}[remark]
\declthm{conjecture}{Conjecture}

% in the intro
\declaretheorem[name = Theorem]{theoremintro}
\renewcommand{\thetheoremintro}{\Alph{theoremintro}}
\declaretheorem[name = Corollary, sibling = theoremintro]{corollaryintro}
\renewcommand{\thecorollaryintro}{\Alph{corollaryintro}}

% Blank footnote (for chapters)
\newcommand{\blkftnt}[1]{{\renewcommand{\thefootnote}{\fnsymbol{footnote}}\footnotetext[0]{#1}}}

% VC information
% \usepackage{shellesc}
% \immediate\write18{vc.bat}
% \input{vc}
% \usepackage{scrlayer-scrpage}
% \makeatletter
% \ifoot*{\scriptsize\texttt{\VCinfo{}}}
% \makeatother
% \ihead*{DRAFT}

% Metadata
\newcommand{\mytitle}{Formalité opéradique et homotopie des espaces de configuration}
\newcommand{\myauthor}{Najib \textsc{Idrissi Kaïtouni}}
\title{\mytitle}
\author{\myauthor}
\hypersetup{pdftitle={\mytitle},pdfauthor={\myauthor}}

% Math
% Fields
\newcommand{\K}{\Bbbk}
\newcommand{\Q}{\mathbb{Q}}
\newcommand{\R}{\mathbb{R}}

% quasi-iso ; qiso* = à l'envers
\NewDocumentCommand{\tkextarrow}{sO{}m}{%
  \mathrel{\vcenter{\hbox{%
    \begin{tikzpicture}
      \node[minimum width=0.5cm, align=center, anchor=south] (tmp) {$\scriptstyle #3$\\[-1ex]$\scriptstyle #2$};
      \path[draw,\IfBooleanTF{#1}{<-}{->}] (tmp.west) -- (tmp.east);
    \end{tikzpicture}%
  }}}%
}
\NewDocumentCommand{\tkrightarrow}{O{}m}{\tkextarrow[#1]{#2}}
\NewDocumentCommand{\tkleftarrow}{O{}m}{\tkextarrow*[#1]{#2}}
\NewDocumentCommand{\qiso}{s}{
  \mathrel{%
    \IfBooleanTF #1
    {\tkleftarrow{\sim}}
    {\tkrightarrow{\sim}}%
  }
}

% misc
\newcommand{\EE}{\mathsf{E}}
\newcommand{\vol}{\mathrm{vol}}

% Operads
\newcommand{\operadify}[1]{\mathtt{#1}}
% Anonymous
\newcommand{\PP}{\operadify{P}}
\newcommand{\QQ}{\operadify{Q}}
\newcommand{\CC}{\operadify{C}}
\newcommand{\DD}{\operadify{D}}
\newcommand{\MM}{\operadify{M}}
\newcommand{\NN}{\operadify{N}}
% Standard
\newcommand{\End}{\operadify{End}}
\newcommand{\Ass}{\operadify{Ass}}
\newcommand{\Ger}{\operadify{Ger}}
\newcommand{\Com}{\operadify{Com}}
\newcommand{\SC}{\operadify{SC}}
\newcommand{\Sc}{\operadify{sc}}
\newcommand{\en}[1][n]{\operadify{e}_{#1}}
\newcommand{\enV}[1][n]{\en[#1]^{\vee}}
\newcommand{\Lie}{\operadify{Lie}}
\newcommand{\hoLie}{\operadify{hoLie}}
\newcommand{\magma}{\Omega}
\newcommand{\tMagma}{{\magma\magma}}
\newcommand{\II}{\operadify{I}}
\newcommand{\Ln}{\operadify{L}_{n}}
% Braids & cie
\newcommand{\CoB}{\operadify{CoB}}
\newcommand{\CoP}{\operadify{CoP}}
\newcommand{\PaB}{\operadify{PaB}}
\newcommand{\PaP}{\operadify{PaP}}
\newcommand{\CoPB}{\operadify{CoPB}}
\newcommand{\PaPB}{\operadify{PaPB}}
\newcommand{\CD}{\operadify{CD}}
\newcommand{\Sh}{\operadify{Sh}}
\newcommand{\PaSh}{\operadify{PaSh}}
\newcommand{\PaCD}{\operadify{PaCD}}
% completions
\newcommand{\CDh}{\widehat{\CD}}
\newcommand{\PaBh}{\widehat{\PaB}}
\newcommand{\PaCDhp}{\widehat{\PaCD}_{+}}
\newcommand{\PaPCD}{\operadify{PaP}\widehat{\operadify{CD}}_{+}}
% Two-colored operads
\newcommand{\col}[1]{\mathfrak{#1}}
% Configuration spaces
\newcommand{\GG}[1]{\mathtt{G}_{#1}}
\newcommand{\GGt}[1]{\tilde{\mathtt{G}}_{#1}}
\newcommand{\FM}{\mathtt{FM}}
\newcommand{\ff}{\operadify{f}}
\newcommand{\SFM}{\mathtt{SFM}}
\newcommand{\OmPA}{\Omega_{\mathrm{PA}}}
\newcommand{\Disk}{\mathtt{Disk}}
\newcommand{\Embfr}{\operatorname{Emb}^{\mathrm{fr}}}
% Graphs
\newcommand{\Gra}{\mathtt{Gra}}
\newcommand{\Graphs}{\mathtt{Graphs}}
\newcommand{\SGra}{\mathtt{SGra}}
\newcommand{\SGraphs}{\mathtt{SGraphs}}
\newcommand{\fGC}{\mathrm{fGC}}
\newcommand{\fLoop}{\mathrm{fLoop}}
\newcommand{\GC}{\mathrm{GC}}
\newcommand{\SGC}{\mathrm{SGC}}
\newcommand{\fSGC}{\mathrm{fSGC}}
% Hadamard product
\newcommand{\hadprod}{\boxtimes}
% Partition function
\newcommand{\ZZ}[1]{\mathrm{Z}_{#1}}
\newcommand{\Zphi}{\ZZ{\varphi}}
\newcommand{\Ze}{\ZZ{\varepsilon}}
\newcommand{\SZphi}{\ZZ{\varphi}^{S}}
\newcommand{\SZe}{\ZZ{\varepsilon}^{S}}
\newcommand{\zz}[1]{\mathrm{z}_{#1}}
\newcommand{\zphi}{\zz{\varphi}}
\newcommand{\ze}{\zz{\varepsilon}}
\newcommand{\szz}[1]{\mathrm{z}^{S}_{#1}}
\newcommand{\szphi}{\szz{\varphi}}
\newcommand{\sze}{\szz{\varepsilon}}

% Operators
\DeclareMathOperator{\Conf}{Conf}
\DeclareMathOperator{\ob}{ob}
\DeclareMathOperator{\Hom}{Hom}
\DeclareMathOperator{\id}{id}
\DeclareMathOperator{\im}{im}
\DeclareMathOperator{\Tw}{Tw}
\DeclareMathOperator{\Def}{Def}
\DeclareMathOperator{\cone}{cone}
\DeclareMathOperator{\colim}{colim}
\DeclareMathOperator{\hoker}{hoker}
\DeclareMathOperator{\TotCof}{TotCof}
\DeclareMathOperator{\Emb}{Emb}

\newcommand{\CCE}{\mathrm{C}_{*}^{\mathrm{CE}}}

% Categories
\newcommand{\categorify}[1]{\mathsf{#1}}
\newcommand{\Zd}{\mathcal{Z}}  % centre de Drinfeld
\newcommand{\cC}{\categorify{C}}
\newcommand{\cM}{\categorify{M}}
\newcommand{\cN}{\categorify{N}}
\newcommand{\cdga}{\categorify{CDGA}}
\newcommand{\Set}{\categorify{Set}}
\newcommand{\rlz}[1]{\langle #1 \rangle^{\mathbb{L}}}

% orphans
\newcommand{\OO}{\mathcal{O}}
% boundary faces
\newcommand{\BF}{\mathcal{BF}}

% Pictures
\newcommand{\bdot}[2]{\draw (#1) node [fill=black, circle, inner
  sep=1.5pt] {} node[above] {#2}}
\newcommand{\wdot}[2]{\draw (#1) node [draw = black, fill=white, circle, inner
  sep=1.5pt] {} node[above] {#2}}
% Elements of CoB
\tikzstyle{cob}=[<->]

% Vertices:
% external
\tikzset{extv/.style={circle, fill = white, draw, minimum size=0.5cm, inner sep=0pt}}
% internal
\tikzset{intv/.style={circle, fill = black, draw, minimum size = 0.3cm, inner sep = 0pt}}
% unknown type
\tikzset{unkv/.style={circle, fill = gray, draw, minimum size = 0.3cm, inner sep = 0pt}}
% small point
\tikzset{pt/.style={circle, fill = black, draw, minimum size = 0.1cm, inner sep = 0pt}}

% torus
\newcommand{\mytorus}{(-3.5,0) .. controls (-3.5,1.5) and (-1.5,2) .. (0,2)}

